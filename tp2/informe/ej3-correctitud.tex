% ------ headers globales y begin ---------------
\documentclass[11pt, a4paper, twoside]{article}
\usepackage{header_tp2}
\begin{document}{}
% -----------------------------------------------
\nota{no se puede asumir que el lector conozca bfs}
En nuestro algoritmo recorremos primero los nodos a distancia 0, luego a distancia 1, y así sucesivamente.
Para cada nodo vamos guardando su distancia desde la posicion de origen.
Inicializamos la distancia hasta la posicion de origen, contando con k unidades extra de potencia, con 0, y el 
resto de las distancias en INF.

Para cada nodo $'v'$ que recorremos, sabemos que podemos llegar a sus adyacentes saltando hasta $v$ en $v$.distancia 
pasos, y luego saltando al adyacente $'a'$ en un paso más. \\
Luego, podemos decir que $a$.distancia $\le$ $v$.distancia + 1. Si no había recorrido $a$ previamente, 
$v$.distancia + 1 refleja la distancia del camino más corto hasta $a$. Si ya había recorrido $'a'$, su 
distancia ya está calculada y es menor o igual a la de $v$. En algún momento llegamos a la casilla destino, 
ya que el grafo es conexo, y podemos devolver su distancia. \\ \nota{no es conexo, no podes llegar de k=3 a k=5}


\end{document}