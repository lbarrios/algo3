% ------ headers globales y begin ---------------
\documentclass[11pt, a4paper, twoside]{article}
\usepackage{header_tp2}
\begin{document}{}
% -----------------------------------------------

\subsubsection{Problema 1: Robanúmeros}
  
\lstset{caption={ej1.cpp},label=ej1}
\begin{lstlisting}

struct Levante {
  bool direccion;
  int cantidad;
};

struct Jugada {
  int turnosHastaAhora;
  Levante levanteRealizado;
  int mejorPuntaje;
  inline bool operator<(const Jugada& j) {
    return mejorPuntaje < j.mejorPuntaje;
  }
};

int main(int argc, const char *argv[]) {
    
	while (1) {
		int cantCartas;
		vector<int> cartas;

		//Se reciben los datos.
		cin >> cantCartas;

		if (cantCartas == 0) {
			return 0;
		}

		for (int i = 0; i < cantCartas; i++) {
			int nuevaCarta;
			cin >> nuevaCarta;
			cartas.push_back(nuevaCarta);
		}


		//Se calculan las sumas parciales para cada elemento.
		int sumasParciales[cantCartas+1]; // Necesito un espacio extra para poner 0 al principio.
		sumasParciales[0] = 0;
		for (int i = 1; i < cantCartas+1; i++) {
			sumasParciales[i] = sumasParciales[i-1] + cartas[i-1];
		}

		//Se crea la matriz para guardar los datos;
		Jugada** mejoresJugadas = new Jugada*[cantCartas];
		for (int i = 0; i < cantCartas; i++) {
			mejoresJugadas[i] = new Jugada[cantCartas];
		}

		// Se inicializa la matriz en cero para simplificar codigo.
		for (int i = 0; i < cantCartas; i++) {
			for (int j = 0; j < cantCartas; j++) {
				mejoresJugadas[i][j].mejorPuntaje = 0;
			}
		}

		/**
		 *Se inicializa la matriz para guardar los datos.
		 *Se toman subconjuntos de tamano cada vez mas grande.
		 */
         
		for (int i = 0; i < cantCartas; i++) {
			Jugada& unaJugadaTamUno = mejoresJugadas[i][i];
			unaJugadaTamUno.mejorPuntaje = cartas[i];
			unaJugadaTamUno.turnosHastaAhora = 1;
			unaJugadaTamUno.levanteRealizado.direccion = 0;
			unaJugadaTamUno.levanteRealizado.cantidad = 1;
		}


		for (int tamSubconjunto = 2; tamSubconjunto <= cantCartas; tamSubconjunto++) {
			int principio = 0;
			int final = tamSubconjunto;
			while (final <= cantCartas) {
				int sumaParcial = sumasParciales[final] - sumasParciales[principio];
				Jugada minActual;
				minActual.mejorPuntaje = 1 << 30;// Maximo numero posible en int;
				bool direccion;
				int cuantosAgarro;
				for (int i = principio; i < final; i++) {
					Jugada jugadaEnemigo = mejoresJugadas[i][final-1];
					if ( jugadaEnemigo < minActual ) {
						direccion = 0; //Esto significa izquierda
						minActual = jugadaEnemigo;
						cuantosAgarro = (i - principio) ? i-principio : tamSubconjunto;
						
					}
				}
				for (int i = principio+1; i <= final; i++) {
					Jugada jugadaEnemigo = mejoresJugadas[principio][i-1];
					if ( jugadaEnemigo < minActual ) {
						direccion = 1; //Esto significa derecha.
						minActual = jugadaEnemigo;
						cuantosAgarro = final - i;
					}
				}

				Jugada miJugada;
				miJugada.mejorPuntaje = sumaParcial - minActual.mejorPuntaje;
				miJugada.turnosHastaAhora = minActual.turnosHastaAhora +1;
				Levante ultimoMovimiento;
				ultimoMovimiento.direccion = direccion;
				ultimoMovimiento.cantidad = cuantosAgarro;
				miJugada.levanteRealizado = ultimoMovimiento;
				
				mejoresJugadas[principio][final-1] = miJugada;

				principio++;
				final++;
			}
		}

		int fin = cantCartas -1;
		int init = 0;
		int t = 0;
		list<Levante>* listaLev = new list<Levante>();
		while ( init <= fin ) {
			Levante l = mejoresJugadas[init][fin].levanteRealizado;
			listaLev->push_back(l);
			if (l.direccion) {
				fin-= l.cantidad;
			} else {
				init += l.cantidad;
			}
			t++;
		}

		cout << t << " " << mejoresJugadas[0][cantCartas-1].mejorPuntaje << " " << sumasParciales[cantCartas] - mejoresJugadas[0][cantCartas-1].mejorPuntaje << endl;
		for (list<Levante>::iterator i = listaLev->begin() ; i != listaLev->end() ; i++) {
			cout << ((i->direccion)? ("der") : ("izq")) << " " << i->cantidad << endl;
		}

	}
}

\end{lstlisting}
\clearpage  
  
\subsubsection{Problema 2: La central ``ITA'' (de gas)}
  {\scriptsize(AKA: La central...ITA)}
   
\lstset{caption={ej2.cpp},label=ej1}
\begin{lstlisting}
	
struct ComponenteConexa {
    bool esComponente;
    float distancias[2048]; 
    arista aristaMasCortaTotalHacia[2048];
    float menorDistancia;
    arista aristaMasCortaTotalGeneral;
    int ccMasCercana;
    list<arista> aristas;
};

ComponenteConexa componentes[2048];

int main() {
    while(tomarInput()){
    	inicializarComponentes();
    	kruskal_parcial();
    	imprimir();
    }
    return 0;
}

void inicializarComponentes() {
    for (int i = 1; i <= n; i++) {

        ComponenteConexa c;
        c.esComponente = true;
        list<arista> inicial;
        c.aristas = inicial;
        c.menorDistancia = INF;
        for (int j = 1; j <= n; j++) {
            if (j == i) continue;
            c.distancias[j] = distancia(i, j);
            c.aristaMasCortaTotalHacia[j] = make_pair(i, j);
            if (c.distancias[j] < c.menorDistancia) {
                c.menorDistancia = c.distancias[j];
                c.aristaMasCortaTotalGeneral = c.aristaMasCortaTotalHacia[j];
                c.ccMasCercana = j;
            }
        }
        componentes[i] = c;
    }
}


void kruskal_parcial() {
    for (int i = 1; i <= n - k; i++) {
        float disMin = INF;
        int ind_cc1, ind_cc2;
        // enncuentro las cc mas cercanas
        for (int j = 1; j <= n; j++) {
            ComponenteConexa &actual = componentes[j];
            if (actual.esComponente == false) continue;
            if (actual.menorDistancia < disMin) {
                disMin = actual.menorDistancia;
                ind_cc1 = j;
                ind_cc2 = actual.ccMasCercana;
            }
        }
        // uno cc1 y cc2
        ComponenteConexa &cc1 = componentes[ind_cc1];
        ComponenteConexa &cc2 = componentes[ind_cc2];
        // me olvido de cc2
        cc2.esComponente = false;
        // concateno las listas de aristas
        cc1.aristas.push_back(cc1.aristaMasCortaTotalGeneral);
        cc1.aristas.insert(cc1.aristas.end(), cc2.aristas.begin(), cc2.aristas.end());
        cc1.menorDistancia = INF;
        for (int j = 1; j <= n; j++) {
            ComponenteConexa &actual = componentes[j];
            if (j == ind_cc1 || j == ind_cc2) continue;
            if (actual.esComponente == false) continue;
            if (actual.distancias[ind_cc2] < actual.distancias[ind_cc1]) {
                actual.distancias[ind_cc1] = actual.distancias[ind_cc2];
                actual.aristaMasCortaTotalHacia[ind_cc1] = actual.aristaMasCortaTotalHacia[ind_cc2];
            }
            // notar que uso <=, es para que si quedaba apuntando a cc2 que ya no existe, cambie a cc1
            if (actual.distancias[ind_cc1] <= actual.menorDistancia) {
                actual.menorDistancia = actual.distancias[ind_cc1];
                actual.ccMasCercana = ind_cc1;
                actual.aristaMasCortaTotalGeneral = actual.aristaMasCortaTotalHacia[ind_cc1];
            }
            cc1.distancias[j] = actual.distancias[ind_cc1];
            cc1.aristaMasCortaTotalHacia[j] = actual.aristaMasCortaTotalHacia[ind_cc1];
            if (cc1.distancias[j] < cc1.menorDistancia) {
                cc1.menorDistancia = cc1.distancias[j];
                cc1.ccMasCercana = j;
                cc1.aristaMasCortaTotalGeneral = cc1.aristaMasCortaTotalHacia[j];
            }
        }
    }
}

void imprimir() {

    int q = 0, m = 0;
    list<int> compFinales;
    for (int i = 1; i <= n; i++) {
        ComponenteConexa &actual = componentes[i];
        if (actual.esComponente) {
            q++;
            compFinales.push_back(i);
            m += actual.aristas.size();
        }
    }
    cout << q << " " << m << endl;
    for (list<int>::iterator it = compFinales.begin(); it != compFinales.end(); it++)
        cout << *it << endl;
    for (list<int>::iterator it = compFinales.begin(); it != compFinales.end(); it++) {
        ComponenteConexa &actual = componentes[*it];
        for (list<arista>::iterator it_arista = actual.aristas.begin(); it_arista != actual.aristas.end(); it_arista++) {
            cout << it_arista->first << " " << it_arista->second << endl;
        }
    }
}
\end{lstlisting}
\clearpage

\subsubsection{Problema 3: Saltos en La Matrix}

\lstset{caption={ej3.cpp},label=ej1}
\begin{lstlisting}

struct Posicion {
    int fila;
    int columna;
    bool operator==(const Posicion &p1) {
        return fila == p1.fila && columna == p1.columna;
    }
};

struct Nodo {
    Posicion posicion;
    int pot_extra;
    Posicion pos_padre;
    int padre_pot_extra_uso;
    int distancia;
};

int main() {
    inicializar();
    BFS();
    imprimir();
    return 0;
}

void inicializar() {
    cin >> n >> origen.fila >> origen.columna >> destino.fila >> destino.columna >> k;
    for (int i=1; i<=n; i++) {
        for (int j=1; j<=n; j++) {
            cin >> potencias[i][j];
            for(int l=0; l<=k; l++){
                //seria INF
                distancias[i][j][l].distancia = n * n;
            }
        }
    }
}

void BFS() {
    Nodo inicial;
    inicial.posicion = origen;
    inicial.pot_extra = k;
    inicial.distancia = 0;
    distancias[origen.fila][origen.columna][k] = inicial;
    queue<Nodo> cola;
    cola.push(inicial);
    while (! cola.empty()) {
        Nodo expando = cola.front();
        cola.pop();
        Posicion pos_actual = expando.posicion;
        if (pos_actual == destino) {
            potenciaDeLlegada = expando.pot_extra;
            break;
        }
        hace_la_cruz(expando, cola);
    }
}

void hace_la_cruz (Nodo expando, queue<Nodo> &cola) {
    Posicion pos_actual = expando.posicion;
    int pot = potencias[pos_actual.fila][pos_actual.columna];
    int pot_extra = expando.pot_extra;
    int fila_hijo, columna_hijo, padre_pot_extra_uso, pot_extra_hijo, distancia_hijo;
    Nodo hijo;
    //cruz para arriba
    for (int i=1; pos_actual.fila - i >= 1 && i <= pot + pot_extra; i++) {
        fila_hijo =  pos_actual.fila - i;
        columna_hijo = pos_actual.columna;
        padre_pot_extra_uso = max(0, i - pot);
        pot_extra_hijo = pot_extra - padre_pot_extra_uso;
        distancia_hijo = expando.distancia + 1;
        if (distancias[fila_hijo][columna_hijo][pot_extra_hijo].distancia == n * n) {
            pushea_hijo(hijo, fila_hijo, columna_hijo, distancia_hijo, pot_extra_hijo, padre_pot_extra_uso, pos_actual, cola);
        }
    }
    //cruz para abajo
    for (int i=1; pos_actual.fila + i <= n && i <= pot + pot_extra; i++) {
        fila_hijo =  pos_actual.fila + i ;
        columna_hijo = pos_actual.columna;
        padre_pot_extra_uso = max(0, i - pot);
        pot_extra_hijo = pot_extra - padre_pot_extra_uso;
        distancia_hijo = expando.distancia + 1;
        if (distancias[fila_hijo][columna_hijo][pot_extra_hijo].distancia == n * n) {
            pushea_hijo(hijo, fila_hijo, columna_hijo, distancia_hijo, pot_extra_hijo, padre_pot_extra_uso, pos_actual, cola);
        }
    }
    //cruz para izquierda
    for (int i=1; pos_actual.columna - i >= 1 && i <= pot + pot_extra; i++) {
        fila_hijo =  pos_actual.fila;
        columna_hijo = pos_actual.columna - i;
        padre_pot_extra_uso = max(0, i - pot);
        pot_extra_hijo = pot_extra - padre_pot_extra_uso;
        distancia_hijo = expando.distancia + 1;
        if (distancias[fila_hijo][columna_hijo][pot_extra_hijo].distancia == n * n) {
            pushea_hijo(hijo, fila_hijo, columna_hijo, distancia_hijo, pot_extra_hijo, padre_pot_extra_uso, pos_actual, cola);
        }
    }
    //cruz para derecha
    for (int i=1; pos_actual.columna + i <= n && i <= pot + pot_extra; i++) {
        fila_hijo =  pos_actual.fila;
        columna_hijo = pos_actual.columna + i;
        padre_pot_extra_uso = max(0, i - pot);
        pot_extra_hijo = pot_extra - padre_pot_extra_uso;
        distancia_hijo = expando.distancia + 1;
        if (distancias[fila_hijo][columna_hijo][pot_extra_hijo].distancia == n * n) {
            pushea_hijo(hijo, fila_hijo, columna_hijo, distancia_hijo, pot_extra_hijo, padre_pot_extra_uso, pos_actual, cola);
        }
    }
}

void pushea_hijo (Nodo &hijo, int &fila_hijo, int &columna_hijo, int &distancia_hijo, int pot_extra_hijo, int padre_pot_extra_uso, Posicion &pos_actual, queue<Nodo> &cola) {
    hijo.posicion.fila = fila_hijo;
    hijo.posicion.columna = columna_hijo;
    hijo.distancia = distancia_hijo;
    hijo.pot_extra = pot_extra_hijo;
    hijo.padre_pot_extra_uso = padre_pot_extra_uso;
    hijo.pos_padre = pos_actual;
    distancias[fila_hijo][columna_hijo][pot_extra_hijo] = hijo;
    cola.push(hijo);
}

void imprimir() {
    Nodo fin = distancias[destino.fila][destino.columna][potenciaDeLlegada];
    cout << fin.distancia << endl;
    list<Nodo> camino;
    Nodo recorro = fin;

    for (int i=0; i<fin.distancia; i++) {
        camino.push_front(recorro);
        int fila_padre = recorro.pos_padre.fila;
        int col_padre = recorro.pos_padre.columna;
        int pot_padre = recorro.pot_extra + recorro.padre_pot_extra_uso;
        recorro = distancias[fila_padre][col_padre][pot_padre];
    }
    for (list<Nodo>::iterator it=camino.begin(); it != camino.end(); it++) {
        cout << it->posicion.fila << " " << it->posicion.columna << " " << it->padre_pot_extra_uso << endl;
    }
}

   \end{lstlisting}

% -----------------------------------------------
\end{document}