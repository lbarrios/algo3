% ------ headers globales y begin ---------------
\documentclass[11pt, a4paper, twoside]{article}
\usepackage{header_tp2}

\begin{document}{}
% -----------------------------------------------
A continuación presentamos distintas instancias que sirven para verificar que el programa funciona correctamente.\\

\begin{itemize}
\item Caso celda origen = celda destino \\
	 \begin{minipage}{0.4\textwidth}
		\begin{tabular}{cccccc}
		   Input \\
		   \hline
		   2 & 1 & 1& 1 & 1 & 0\\
		   1 & 1 &  &   &   &  \\
		   1 & 1 &  &   &   &  \\
		\end{tabular}
	\end{minipage} 
		\begin{minipage}{0.2\textwidth}
			\begin{tabular}{c}
			   Output \\
			   \hline
			   0 \\
				 \\
				 \\
		\end{tabular}
	\end{minipage} 

\item Caso celda origen $\ne$ celda destino 
	\begin{itemize}
		\item Potencia extra = 0 \\
			\begin{itemize}
				\item Potencia máxima del resorte igual para todas las celdas: \\
				\\
					\begin{minipage}{0.4\textwidth}
							\begin{tabular}{cccccc}
							 & Input \\
							   \hline
							   3 & 1 & 1 & 3 & 3 & 0\\
							   1 & 1 & 1 &   &   &  \\
							   1 & 1 & 1 &   &   &  \\
							   1 & 1 & 1 &   &   &  \\
							   \\
							\end{tabular}
						\end{minipage} 
							\begin{minipage}{0.3\textwidth}
								\begin{tabular}{ccc}
								  & Output \\
								   \hline
								   4 &   &   \\
								   2 & 1 & 0 \\
								   3 & 1 & 0 \\
								   3 & 2 & 0 \\
								   3 & 3 & 0 \\
							\end{tabular}
					\end{minipage} 	\\
					\\ \\
					\begin{minipage}{0.4\textwidth}
							\begin{tabular}{cccccc}
							   & Input  \\
							   \hline
							   3 & 1 & 1 & 3 & 3 & 0\\
							   5 & 5 & 5 &   &   &  \\
							   5 & 5 & 5 &   &   &  \\
							   5 & 5 & 5 &   &   &  \\
							\end{tabular}
					\end{minipage} 
					\begin{minipage}{0.3\textwidth}
							\begin{tabular}{ccc}
								   & Output \\
								   \hline
								   2 &   &   \\
								   3 & 1 & 0 \\
								   3 & 3 & 0 \\
								   \\
							\end{tabular}
					\end{minipage} 	\\

				\item Potencia máxima del resorte distinta para las celdas: \\
				\\
				    \begin{minipage}{0.4\textwidth}
							\begin{tabular}{cccccc}
							 & Input \\
							   \hline
							   3 & 1 & 1 & 3 & 3 & 0\\
							   1 & 1 & 2 &   &   &  \\
							   1 & 3 & 1 &   &   &  \\
							   1 & 1 & 1 &   &   &  \\
							   \\
							\end{tabular}
						\end{minipage} 
							\begin{minipage}{0.3\textwidth}
								\begin{tabular}{ccc}
								  & Output \\
								   \hline
								   3 &   &   \\
								   1 & 2 & 0 \\
								   1 & 3 & 0 \\
								   3 & 3 & 0 \\
								    \\
							\end{tabular}
					\end{minipage} 	\\
					\\
					
					\begin{minipage}{0.4\textwidth}
							\begin{tabular}{cccccc}
							 & Input \\
							   \hline
							   4 & 1 & 1 & 4 & 4 & 0\\
							   1 & 1 & 3 & 1 &   &  \\
							   3 & 1 & 1 & 2 &   &  \\
							   1 & 1 & 1 & 1 &   &  \\
							   2 & 1 & 1 & 1 &   &  \\
							\end{tabular}
						\end{minipage} 
							\begin{minipage}{0.3\textwidth}
								\begin{tabular}{ccc}
								  & Output \\
								   \hline
								   3 &   &   \\
								   2 & 1 & 0 \\
								   2 & 4 & 0 \\
								   4 & 4 & 0 \\
								    \\
							\end{tabular}
					\end{minipage} 	\\
			\end{itemize} 
		\item Potencia extra $\ne$ 0 \\
			\begin{itemize}
				\item Potencia máxima del resorte igual para todas las celdas:\\
				\\
					\begin{minipage}{0.4\textwidth}
							\begin{tabular}{cccccc}
							 & Input \\
							   \hline
							   3 & 1 & 1 & 3 & 3 & 5\\
							   1 & 1 & 1 &   &   &  \\
							   1 & 1 & 1 &   &   &  \\
							   1 & 1 & 1 &   &   &  \\
							\end{tabular}
						\end{minipage} 
						\begin{minipage}{0.3\textwidth}
							\begin{tabular}{ccc}
								   & Output \\
								   \hline
								   2 &   &   \\
								   3 & 1 & 1 \\
								   3 & 3 & 1 \\
								   \\
							\end{tabular}
					    \end{minipage} 	\\
				\item Potencia máxima del resorte distinta para las celdas: \\
				\\
					\begin{minipage}{0.4\textwidth}
								\begin{tabular}{cccccc}
								 & Input \\
								   \hline
								   3 & 1 & 1 & 3 & 3 & 1\\
								   1 & 1 & 2 &   &   &  \\
								   1 & 3 & 1 &   &   &  \\
								   1 & 1 & 1 &   &   &  \\
								\end{tabular}
							\end{minipage} 
								\begin{minipage}{0.3\textwidth}
									\begin{tabular}{ccc}
									  & Output \\
									   \hline
									   2 &   &   \\
									   1 & 3 & 1 \\
									   3 & 3 & 0 \\
									   \\
								\end{tabular}
					\end{minipage} \\
					\\
					
					\begin{minipage}{0.4\textwidth}
							\begin{tabular}{cccccc}
							 & Input \\
							   \hline
							   4 & 1 & 1 & 4 & 4 & 3\\
							   1 & 1 & 3 & 1 &   &  \\
							   3 & 1 & 1 & 2 &   &  \\
							   1 & 1 & 1 & 1 &   &  \\
							   2 & 1 & 1 & 1 &   &  \\
							\end{tabular}
						\end{minipage} 
							\begin{minipage}{0.3\textwidth}
								\begin{tabular}{ccc}
								  & Output \\
								   \hline
								   2 &   &   \\
								   4 & 1 & 2 \\
								   4 & 4 & 1 \\
								    \\
								    \\
							\end{tabular}
					\end{minipage} 						
				
			\end{itemize}
	\end{itemize}
\end{itemize}

Ejecutamos el programa con los distintos ejemplos y se llegó a la solución esperada. 
Por lo tanto, podemos concluir que el comportamiento del programa es correcto. 

\end{document}
