% ------ headers globales y begin ---------------
\documentclass[11pt, a4paper, twoside]{article}
\usepackage{header_tp2}

\begin{document}{}
% -----------------------------------------------
Para una mayor claridad, vamos a describir nuestro algoritmo reduciendo el problema a encontrar la distancia
hasta la casilla destino. Encontrar la secuencia de saltos es algo secundario y está detallado en el código fuente.

Vamos a pensar el problema como un grafo, siendo cada nodo una posición del tablero con una cantidad de unidades 
de potencia extra restantes. Los adyacentes a cada nodo son las casillas (y las unidades extra que quedan para 
cada caso) a las que puedo llegar usando mi resorte y mis unidades extra. Es importante destacar que desde cualquier casilla puedo ir a cualquier otra casilla de la matriz, ya que las potencias de los resortes son positivas, es decir, siempre me puedo mover hacia las casillas vecinas, y así hacia la siguiente, sucesivamente hasta llegar a cualquier posición del tablero. Más explicitamente, desde la posición (a, b) hasta la (c, d) un camino es el siguiente. Supongamos que a $<$ c y b $<$ d, se puede extender a los otros casos muy facilmente.


(a, b) $->$ (a+1, b) $->$ ... $->$ (c, b) $->$ (c, b+1) $->$ (c, b+2) $->$ ... $->$ (c, d)
\\\\es un camino que no itiliza ninguna unidad extra de potencia. 

Se implementó un algoritmo utilizando la técnica de Breadth-First Search. \\

\centerbf{Pseudocódigo}
\begin{algorithm}[H]
\caption{Saltos en la Matrix}\label{alg:ej3-matrix}
\begin{algorithmic}[1]

\Statex \texttt{// Inicializo}
	\For{$(i=1..n, j=1..n, l=0..k)$}  \Comment{\bigO{n^2 * k}}
		\State distancia hasta el casillero[i][j], sobrando l unidades extra de potencia = $\infty$
	\EndFor \\

\State distancia hasta el casillero origen, sobrando k unidades extra de potencia = 0 \\
\State cola$<(int, int, int)>$ colaBFS
\State colaBFS.push(origen, k) \\

\While{(colaBFS no esté vacía)}  \Comment{\bigO{n^3 * k}} 
\State actual = colaBFS.pop()
	\For{cada casillero (x,y) al que puedo llegar desde actual}
		    \State l = unidades extra que quedan tras ir a (x,y) \\
			\Statex \hspace{1cm} \texttt{// si no recorrí ya ese casillero, quedando esas unidades extra}
			\If{distancia a (x, y), quedando l unidades extra es menor a $\infty$}
				\State   la sobreescribo como: 'distancia desde actual' + 1
				\If {es el casillero de destino} 
					\State break
				\EndIf
				\State colaBFS.push((x,y), unidades extra que quedan)
			\EndIf	
	\EndFor		
\EndWhile
\\
\State \Return distancia al destino

\end{algorithmic}
\end{algorithm}

\end{document}
