% ------ headers globales y begin ---------------
\documentclass[11pt, a4paper, twoside]{article}
\usepackage{header_tp2}

\begin{document}{}
% -----------------------------------------------
% Para una mayor claridad, vamos a describir nuestro algoritmo reduciendo el problema a encontrar la distancia hasta la casilla destino.\\
% Encontrar la secuencia de saltos es algo secundario y está detallado en el código fuente.
% Vamos a pensar el problema como un grafo, siendo cada nodo una posición del tablero con una cantidad de unidades de potencia extra \\
% restantes. Los adyacentes a cada nodo son las casillas (y las unidades extra que quedan para cada caso) a las que puedo llegar usando \\
% mi resorte y mis unidades extra. Se implementó un Breadth-first search. Recorremos primero los nodos a distancia 0, luego a distancia 1, y \\
% así sucesivamente.
% Para cada nodo vamos guardando su distancia desde la posicion de origen. \\
% Inicializamos la distancia hasta la posicion de origen, contando con k unidades extra de potencia, con 0, y el resto de las distancias en INF.\\
% Para cada nodo 'v' que recorremos, sabemos que podemos llegar a sus adyacentes saltando hasta v en v.distancia pasos, y luego saltando al \\
% adyacente 'a' en un paso mas. Luego, podemos decir que a.distancia <= v.distancia + 1. Si no habia recorrido a previamente, \\
% v.distancia + 1 refleja la distancia del camino más corto hasta a. Si ya habia recorrido 'a', su distancia ya está calculada y es menor o \\
% o igual a la de v.\\
% En algun momento llegamos a la casilla destino, ya que el grafo es conexo, y podemos devolver su distancia.
% \nota{Esto de que es conexo, es decir, que cualquier casillero podes llegar a cualquier otro, se puede mencionar en la descripcion del
% problema}


% Pseudocódigo:

% for i=1..n, j=1..n, l=0..k :
%     distancia desde el casillero[i][j], sobrando l unidades extra de potencia = infinito

% distancia hasta el casillero origen, sobrando k unidades extra de potencia = 0

% cola<(int, int, int)> colaBFS

% colaBFS.push(origen, k)

% while ! colaBFS.empty():
%     actual = colaBFS.pop()
%     para cada casillero (x, y) al que puedo llegar desde actual
%         l = unidades extra que quedan tras ir a (x, y)
%         // si no recorri ya ese casillero, quedando esas unidades extra
%         if distancia al casillero (x, y), quedando l unidades extra es menor a infinito :
%             la pongo en 'distancia desde actual' + 1
%             if (es el casillero de destino) break
%             colaBFS.push((x,y), unidades extra que quedan)

% return distancia al destino
\end{document}
