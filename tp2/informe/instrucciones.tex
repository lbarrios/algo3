% ------ headers globales y begin ---------------
\documentclass[11pt, a4paper, twoside]{article}
\usepackage{header_tp2}
\begin{document}{}
% -----------------------------------------------

\subsection{Herramientas utilizadas}\label{subsec:instrucciones-herramientas}

Para la realización de este trabajo se utilizaron un conjunto de herramientas, las cuales se enumeran a continuación:

\begin{itemize}
  \item \texttt{C++} como lenguaje de programación
    \begin{itemize}
      \item \texttt{gcc} como compilador de C++
    \end{itemize}
  \item \texttt{python} y \texttt{bash} para la realización de scripts
    \begin{itemize}
      \item \texttt{python} para generar casos de prueba
      \item \texttt{bash} para automatizar las mediciones
      \item \texttt{python/matplotlib} para plotear los gráficos
    \end{itemize}
  \item \LaTeX\ para la redacción de este documento
  \item Se testeó bajo los siguientes Sistemas Operativos \hfill
    \begin{itemize}
      \item \texttt{Debian GNU/Linux}
      \item \texttt{Ubuntu}
      \item \texttt{FreeBSD}, compilando a través de \texttt{gmake}
      \item \texttt{Windows}, a través de \texttt{cygwin}
    \end{itemize}
\end{itemize}

% -----------------------------------------------
\end{document}