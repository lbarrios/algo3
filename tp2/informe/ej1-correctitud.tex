%
%
%
% ESTA SECCION SE ENCUENTRA LEÍDA, CORREGIDA Y "CERRADA".
% PARA EVITAR INCONVENIENTES, POR FAVOR, 
% NO MODIFICAR A ULTIMO MOMENTO SIN AVISAR AL RESTO DEL GRUPO.
%
%
%

% ------ headers globales y begin ---------------
\documentclass[11pt, a4paper, twoside]{article}
\usepackage{header_tp2}

\begin{document}{}
% -----------------------------------------------

Se procede a demostrar que nustro algoritmo es correcto. Vamos a concentrarnos
en probar que devuelve un puntaje máximo correcto, que obtenemos de
$matrizJugadas[1][n]$.

\centerbf{Premisa inductiva}

P(k): En la iteración k del ciclo principal se cumple
\begin{align*}
\forall (i, j) : j - i \leq k \text{, sea C} = \text{cartas que venían en el orden de i a j} \\
matrizJugadas[i][j] = MaxPuntajePosible( C ) 
\end{align*}

Hagamos inducción en k, necesitamos probar P(n).

Nos vamos a apoyar en la \propref{ej1-mpp}
\centerbf{Caso base} 

P(1): Se corresponde al caso base de la proposición. $matrizJugadas[i][i]$ vale
el valor de la carta $i$, según la operación en \algline{alg:ej1}{alg:ej1-trivial}

\centerbf{Paso inductivo} 

Supongo que vale P(k - 1), y quiero ver que vale P(k) $\forall$ $2 \leq k \leq n$.

Ahora, los subconjuntos de las cartas que venian en el orden de i a j que le
puedo dejar al oponente son:

\begin{itemize}

  \item Ninguna carta\footnote{Este caso está representado en
$matrizJugadas[x][y]$ con $y < x$, inicializado al comienzo del algoritmo con 0,
que representa el puntaje máximo que se puede sacar sin cartas sobre la mesa.}

  \item Las cartas 
    de <<$i$ a $j-1$>>, 
    de <<$i$ a $j-2$>>,
    <<$\dots$>>, 
    de <<$i$ a $i$>>
    \footnote{Corresponde a sacar cartas del extremo derecho.}

  \item Las cartas 
    de <<$i+1$ a $j$>>, 
    de <<$i+2$ a $j$>>,
    <<$\dots$>>,
    de <<$j$ a $j$>>
    \footnote{Corresponde a sacar cartas del extremo izquierdo.}

\end{itemize}

Ahora, todos estos subconjuntos de cartas cumplen que:\\
\mbox{\texttt{su extremo derecho} - \texttt{su extremo izquierdo} $\leq k$},
luego por HI su $puntajeMaximo$ ya está calculado en
$matrizJugadas[extremo izquierdo][extremo derecho]$.

Nuestro algoritmo recorre la matriz en todas esas posiciones, quedándonse con el
puntaje más bajo (el que saca el oponente), y guardandolo en
$matrizJugadas[i][j]$. Luego por el resultado 1.1 contiene
$MaxPuntajePosible(\text{cartas de i a j})$. En la <<k-esima>> iteración del
ciclo principal, lo hace para todos los $(i, j)$ tales que $i - j = k$. Luego se
cumple P(k).

\end{document}