% ------ headers globales y begin ---------------
\documentclass[11pt, a4paper, twoside]{article}
\usepackage{header_tp2}

\begin{document}{}
% -----------------------------------------------
%-
%- PLANTEO DEL PROBLEMA
%-
\centerbf{Planteo del Problema}

Existe una región del país en la que \emph{un grupo de pueblos no cuenta con red de gas natural}.
Luego de una fuerte campaña política se lograron recaudar los fondos necesarios para \emph{emprender
una obra que provea del servicio a todos los pueblos de la región}.

El sistema consistirá en \textbf{una red de \underline{tuberías interconectadas} de un pueblo al
otro}, de forma tal que \emph{la distribución de la misma asegure el abastecimiento} de cada uno de
los pueblos, y en donde además \textbf{se seleccionarán determinados pueblos para establecer
\underline{centralitas}}, que serán las encargadas de \emph{proveer gas hacia todo pueblo con el que
cuenten con conexión}, ya sea mediante una tubería o a través de un camino de tuberías.

Debido a que el costo de las cañerías es significativamente menor que el de las centralitas,
\textbf{el presupuesto final estará regido por la cantidad de centralitas} que se construyan (y
viceversa, según quién lo mire). Por lo antes expuesto se conoce el cálculo que permite predecir,
habiendo reunido una cantidad determinada de presupuesto, cual es la \textbf{cantidad máxima de
centralitas} correspondientes que el mismo permite construir.

Se entiende como ``\underline{\textbf{riesgo}}'' de una determinada distribución de tuberías a la
\textbf{máxima de sus longitudes}.

El codicioso Fontanero Jefe, Mario Toti Segale\footnote{``No es más que un italojaponés americano...
gordo y bigotudo'' - descripción anónima de un empleado.}, desea conocer, dado un determinado
``presupuesto máximo'', es decir, una ``cantidad máxima de centralitas'', cuál es la distribución
óptima de tuberías y centralitas de forma tal que el gasto sea menor al presupuestado. Para ello
encomendó la tarea a su tímido hermano Luigi. Dado que Luigi es precavido y miedoso por naturaleza,
decidió que el gasto no importaba realmente, siempre que fuera menor al presupuestado, y que una
distribución óptima era más bien aquella en la que el riesgo resultase mínimo.

%-
%- REQUERIMIENTOS TÉCNICOS
%-
\centerbf{Requerimientos técnicos}

El problema requiere encontrar una distribución insuperable de gaseoductos, recibiendo como
\textbf{datos de entrada} la cantidad de ciudades ($n$), la cantidad máxima de núcleos gaseosos
($k$), y $n$ pares de números enteros ($x$,$y$) representando las coordenadas euclídeas de cada una
de las borneras (hay una por pueblo), y devolviendo como \textbf{datos de salida} la cantidad de
núcleos gaseosos ($q$) y caños ($m$) construídos, junto con un listado en donde se detallen cada uno
de los $q$ pueblos ($p$) en donde se emplazará una central, y un segundo listado en donde a través
de $m$ pares de números enteros ($v$,$w$) se detallen cada una de las tuberías a construir. El
algoritmo implementado debe respetar una complejidad temporal de peor caso de \bigO{n^{2}}.

%-
%- FORMATO DE LOS DATOS
%-
\centerbf{Formato de los datos}
% primera columna
\begin{minipage}[t]{0.5\textwidth}
  \centerbf{Formato de entrada}
  \setlength{\leftmargini}{6.5em}
  \begin{quote}
  \texttt{n k}\\
  \texttt{x$_{1}$ y$_{1}$}\\
  \texttt{.}\\
  \texttt{.}\\
  \texttt{.}\\
  \texttt{x$_{n}$ y$_{n}$}
  \end{quote}
\end{minipage}%
% segunda columna
\begin{minipage}[t]{0.5\textwidth}
  \centerbf{Formato de salida}
  \setlength{\leftmargini}{6.5em}
  \begin{quote}
  \texttt{q m}\\
  \texttt{p$_{1}$ ... p$_{q}$}\\
  \texttt{v$_{1}$ w$_{1}$}\\
  \texttt{.}\\
  \texttt{.}\\
  \texttt{.}\\
  \texttt{v$_{m}$ w$_{m}$}
  \end{quote}
\end{minipage}

\end{document}