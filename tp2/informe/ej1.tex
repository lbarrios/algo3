% ------ headers globales y begin ---------------
\documentclass[11pt, a4paper, twoside]{article}
\usepackage{header_tp1}

\begin{document}{}
% -----------------------------------------------

\subsubsection{Descripción} 

En este problema se tiene que crear un algoritmo que juegue al \textit{Robanúmeros} de forma tal 
que el jugador realice la mejor jugada posible a su favor durante su turno. Es decir que los dos jugadores que
participan, estarán jugando de manera óptima todo el tiempo. El algoritmo tiene que tener una complejidad temporal
de peor caso de \bigO{n^3}, con n la cantidad de cartas iniciales.\\
\\
Reglas del \textit{Robanúmeros}: 
\begin{itemize}
	\item Comienzo del juego: \\
	Se tiene una cantidad n ($n \in \mathbb{N}$) de cartas con valores enteros alineadas
	horizontalmente (c1, c2, ..., cn) sobre la mesa. Las cartas tienen que estar boca arriba. 
	\item Turnos: \\
	Participan 2 jugadores, cada uno va alternando un turno.(Total de turnos t: 1,..,n).  
	\item Elección de cartas: \\
	En cada turno el jugador tiene que elegir un extremo, el izquierdo (izq) o el derecho (der), de 
	la secuencia de cartas desde el que irá tomando de 1 a n de las cartas adyacentes que están en la mesa. 
	La cantidad de cartas elegidas variará según le sea conveniente al jugador, pero por lo menos tiene que tomar una 
	carta en su turno. 
	\item Fin del juego: \\
	El juego finaliza cuando no hay más cartas sobre la mesa. Se suman las cartas de cada 
	jugador (p1: Ptos. Jug1, p2: Ptos. Jug2). Gana el que obtiene el mayor puntaje.
\end{itemize}

\begin{ejemplo}\hspace{0em}

	\begin{itemize}
		\item	Cartas iniciales:
			\begin{center}
				  \begin{tabular}{|c|c|c|c|c|}
					  \hline
					   2 & -3 & -2 & 5 & 5 \\
					  \hline
				  \end{tabular}
			\end{center} 
		 
		\item   Turno1 (Jug1): Elige el extremo derecho y toma las 2 últimas cartas. 
			\begin{tabular}{|c|c|}
				  \hline
				   5 & 5 \\
				  \hline
			\end{tabular} \\

		\item	Quedan sobre la mesa: 	
			\begin{center}
				  \begin{tabular}{|c|c|c|}
					  \hline
					   2 & -3 & -2 \\
					  \hline
				  \end{tabular}
			\end{center} 	

		\item	Turno2 (Jug2): Elige el extremo izquierdo y toma 1 carta. 
			\begin{tabular}{|c|}
				  \hline
				   2\\
				  \hline
			\end{tabular} \\

		\item Quedan sobre la mesa: 	
			\begin{center}
			  \begin{tabular}{|c|c|}
				  \hline
				   -3 & -2 \\
				  \hline
			  \end{tabular}
			\end{center} 		
			
		\item Turno3 (Jug1): Elige el extremo derecho y toma 1 carta. 
			  \begin{tabular}{|c|}
				  \hline
				   -2 \\
				  \hline
			  \end{tabular} \\

		\item Quedan sobre la mesa: 	
			\begin{center}
			  \begin{tabular}{|c|}
				  \hline
				   -3 \\
				  \hline
			  \end{tabular}
			\end{center} 		

		\item Turno4 (Jug2): Sólo queda una carta, por lo que elige ésta. Es indistinto para este caso 
			  si el extremo elegido es el izquierdo o el derecho. 
			  \begin{tabular}{|c|}
				  \hline
				   -3 \\
				  \hline
			  \end{tabular} \\

		\item Finaliza el juego porque no hay más cartas. Se suman los puntajes de cada jugador.\\
			\begin{center}
			  \begin{tabular}{|c|c|}
				  \hline
				  Ptos. Jug1 & Ptos. Jug2 \\
				  \hline
				  5 + 5 + (-2) = 8 & 2 + (-3) = -1 \\
				  \hline
			  \end{tabular} \\ 
			\end{center} 	
			
		\item Formato de entrada y salida: 
			
			\begin{minipage}{0.5\textwidth}
				  \begin{tabular}{ccccccc}
					   Input: 5 & 2 & -3 & -2 & 5 & 5 \\
					   \\
					   \\
					   \\
					   \\
				  \end{tabular}
			  \end{minipage} 
			  \begin{minipage}{0.3\textwidth}
				  \begin{tabular}{cccc}
					  Output:  & 4   & 8  & -1 \\
							   & der & 2  & \\
							   & izq & 1  & \\
							   & der & 1  & \\
							   & izq & 1  & \\
				  \end{tabular}
			  \end{minipage}
	
	\end{itemize} 	
		
\end{ejemplo}

\begin{ejemplo}\hspace{0em}

	\begin{center}
	  \begin{tabular}{|c|c|c|}
		  \hline
		   2 & -1 & 6 \\
		  \hline
	  \end{tabular} 
	\end{center} 

El Jug1 toma todas las cartas porque de esta manera obtiene el mayor puntaje. \\
Finaliza el juego en 1 turno porque no hay más cartas. Se suman los puntajes de cada jugador.\\
Este tipo de caso también se daría si todas las cartas tuvieran números positivos, sólo llegaría a jugar el Jug1.\\

	\begin{center}
	  \begin{tabular}{|c|c|}
		  \hline
		  Ptos. Jug1 & Ptos. Jug2 \\
		  \hline
		  2 + (-1) + 6 = 7 & 0 \\
		  \hline
	  \end{tabular}
	\end{center} 	
	
\end{ejemplo}



\subsubsection{Hipótesis de resolución}
Para una mayor claridad, vamos a reducir el problema a encontrar la mayor cantidad de puntos que se pueden sacar con el juego de cartas dado.\\
Sea f(i,j) = "maxima cantidad de puntos que se pueden sacar en el juego que consiste en las cartas que estaban desde la posición i \\
hasta la j en el juego de cartas original".\\
Sea suma(i,j) la suma de las cartas que estaban entre la posición i y j en el juego de cartas original.\\
Nuestro algoritmo se basa en que f(i,j) =   | valor de la carta i      si i=j\\
                                            | suma(i, j) - min(f(i', j')) para todo i' j' que representen un juego que le dejo al oponente\\
                                              usando una movida válida        sino\\
\\
\nota{habra que justificar esto?}





\subsubsection{Justificación formal de correctitud}

\subsubsection{Cota de complejidad temporal}

\subsubsection{Verificación mediante casos de prueba}

\end{document}
