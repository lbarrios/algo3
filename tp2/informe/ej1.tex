% ------ headers globales y begin ---------------
\documentclass[11pt, a4paper, twoside]{article}
\usepackage{header_tp1}

\begin{document}{}
% -----------------------------------------------

\subsubsection{Descripción} 

\subsubsection{Hipótesis de resolución}
Para una mayor claridad, vamos a reducir el problema a encontrar la mayor cantidad de puntos que se pueden sacar con el juego de cartas dado.\\
Sea f(i,j) = "maxima cantidad de puntos que se pueden sacar en el juego que consiste en las cartas que estaban desde la posición i \\
hasta la j en el juego de cartas original".\\
Sea suma(i,j) la suma de las cartas que estaban entre la posición i y j en el juego de cartas original.\\
Nuestro algoritmo se basa en que f(i,j) =   | valor de la carta i      si i=j\\
                                            | suma(i, j) - min(f(i', j')) para todo i' j' que representen un juego que le dejo al oponente\\
                                              usando una movida válida        sino\\
\\
\nota{habra que justificar esto?}





\subsubsection{Justificación formal de correctitud}

\subsubsection{Cota de complejidad temporal}

\subsubsection{Verificación mediante casos de prueba}

\end{document}
