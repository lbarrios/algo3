% ------ headers globales y begin ---------------
\documentclass[11pt, a4paper, twoside]{article}
\usepackage{header_tp2}

\begin{document}{}
% -----------------------------------------------

A continuación presentamos distintas instancias que sirven para verificar que el programa funciona correctamente.\\

\begin{minipage}{0.4\textwidth}
      \begin{tabular}{cccc}
         Input \\
         \hline
         n & $c_1$ & \dots & $c_n$ \\
         \\
         \\
         \\
      \end{tabular}
\end{minipage} 
\begin{minipage}{0.3\textwidth}
      \begin{tabular}{cccc}
        Output\\
        \hline
        & t   & p1  & p2 \\
        & $e_1$ & $c_1$  & \\
        & \vdots & \vdots & \\
        & $e_t$  & $c_t$ & 
      \end{tabular}
\end{minipage} \\

\begin{itemize}
  \item n: \#cartas iniciales. 
  \item $c_i$ con $1 \le i \le n$: $c_i$ valor de la carta $i$. 
  \item t: \#turnos del juego. 
  \item p1: puntaje total Jug1.
  \item p2: puntaje total Jug2.
  \item $e_i$ con $1 \le i \le t$: $e_i$ extremo elegido por el jugador en el turno i (izq o der). 
  \item $c_i$ con $1 \le i \le t$: $c_i$ \#cartas tomadas por el jugador en el turno i.   
\end{itemize}
        
Según los valores de p1 y p2, podemos separar en 3 casos posibles: 
        
\begin{enumerate}
  \item Caso Empate entre Jug1 y Jug2:

    \begin{itemize}
      \item Cartas con valor cero \\
      \\
        \begin{minipage}{0.4\textwidth}
            \begin{tabular}{cccc}
               Input \\
               \hline
               3 & 0 & 0 & 0 \\
               \\
            \end{tabular}
        \end{minipage} 
        \begin{minipage}{0.3\textwidth}
            \begin{tabular}{cccc}
              Output\\
              \hline
              & 1   & 0  & 0 \\
              & izq & 3  & \\
            \end{tabular}
        \end{minipage} 
        
      \item Cartas con valores negativos \\
      \\
        \begin{minipage}{0.4\textwidth}
            \begin{tabular}{cccc}
               Input\\
               \hline
               3 & -1 & -2 & -3 \\
               \\
               \\
            \end{tabular}
        \end{minipage} 
        \begin{minipage}{0.3\textwidth}
            \begin{tabular}{cccc}
              Output \\
                          \hline
                 & 2   & -3  & -3 \\
               & izq & 2  & \\
               & izq & 1  & \\
            \end{tabular}
        \end{minipage}
    \end{itemize}
    
  \item Caso Perdedor Jug1: 
    \begin{itemize}
      \item Cartas con valores negativos \\
      \\
        \begin{minipage}{0.4\textwidth}
            \begin{tabular}{cccc}
               Input \\
                           \hline
               3 & -2 & -3 & -1 \\
               \\
               \\
            \end{tabular}
        \end{minipage} 
        \begin{minipage}{0.3\textwidth}
            \begin{tabular}{cccc}
              Output\\
                \hline
               & 2   & -4  & -2 \\
               & der & 2  & \\
               & izq & 1  & \\
            \end{tabular}
        \end{minipage} 
    \end{itemize}

  \item Caso Ganador Jug1: 
  
    \begin{itemize}
      \item Cartas con valores positivos \\
      \\
        \begin{minipage}{0.4\textwidth}
            \begin{tabular}{cccc}
               Input \\
                           \hline   
               3 & 1 & 2 & 3 \\
               \\
            \end{tabular}
        \end{minipage} 
        \begin{minipage}{0.3\textwidth}
            \begin{tabular}{cccc}
              Output\\
                          \hline
                       & 1   & 6  & 0 \\
               & izq & 3  & \\
            \end{tabular}
        \end{minipage} 
        
      \item Cartas con valores negativos \\ 
      \\
        \begin{minipage}{0.4\textwidth}
            \begin{tabular}{cccc}
               Input \\
                           \hline
                          3 & -5 & -1 & -3 \\
               \\
               \\
            \end{tabular}
        \end{minipage} 
        \begin{minipage}{0.3\textwidth}
            \begin{tabular}{cccc}
              Output \\
                          \hline
                           & 2   & -4  & -5 \\
               & der & 2  & \\
               & izq & 1  & \\
            \end{tabular}
        \end{minipage}
      
            \item Cartas con valores positivos y negativos \\
      \\
        \begin{minipage}{0.4\textwidth}
            \begin{tabular}{ccccc}
               Input\\
                           \hline
               4 & 2 & -8 & -8 & 3 \\
               \\
               \\
               \\
               \\
            \end{tabular}
        \end{minipage} 
        \begin{minipage}{0.3\textwidth}
            \begin{tabular}{cccc}
               Output \\
             \hline 
               & 4   & -5  & -6 \\
               & der & 1  & \\
               & izq & 1  & \\
               & izq & 1  & \\
               & izq & 1  & \\
            \end{tabular}
        \end{minipage}        
        
    \end{itemize}
  
Ejecutamos el programa con los distintos ejemplos y se llegó a la solución esperada. 
Por lo tanto, podemos concluir que el comportamiento del programa es correcto. 
  
\end{enumerate}

\end{document}