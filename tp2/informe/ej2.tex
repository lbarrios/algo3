% ------ headers globales y begin ---------------
\documentclass[11pt, a4paper, twoside]{article}
\usepackage{header_tp1}

\begin{document}{}
% -----------------------------------------------

\subsubsection{Descripción} 

\subsubsection{Hipótesis de resolución}

Pseudocódigo: \\

ComponenteConexa componentesConexas[n] \\

estructura ComponenteConexa {
    float distancias[n]
    list<arista> aristas
    arista aristaMasCortaHacia[n]
    float distanciaMasCorta
    int indiceCCMasCerca
}

for i in 1 to n: comentario: O(n^2) \\
    componentesConexas[i] = i-esimo pueblo  comentario: O(1)\\
    for j in 1 to n:  comentario: O(n)\\
        aristaMasCorta entre la CC i y la CC j = (i, j) comentario: O(1) \\
        distancia entre componentesConexas[i] y la componente conexa j = distanciaEuclidea(pueblo i, pueblo j) comentario: O(1) \\
        me voy fijando cual de estas distancias es mas corta y la guardo junto con el indice j  comentario: O(1)\\

for i in 1 to n - k: comentario: O(n * (n-k)) = O(n^2)\\
    comentario: me fijo la distancia mas corta entre dos componentes \\
    for i in 1 to n:  comentario: O(n) \\ 
        observacion: si componentes[i] ya no representa mas una componente continuo \\
        me voy fijando que componente tiene la menor 'menor distancia hacia otra componente' y la guardo en CCAUnir1,\\
        su componente mas cercana en CCAUnir2 comentario: O(1) \\

    comentario: uno CCAUnir1 y CCAUnir2 \\
    CCAUnir1.aristas = CCAUnir1.aristas \union CCAUnir2.aristas + aristaMasCorta entre CCAUnir1 y CCAUnir2 comentario: O(1)\\
    marco CCAUnir2 como que ya no representa una componente conexa. toda su información pasa a CCAUnir1  comentario: O(1)\\
    
    comentario: actualizo distancias \\
    for i in 1 to n:  comentario: O(n)\\
        si componentesConexas[i] ya no representa una componente conexa, continuo comentario: O(1) \\
        distancia entre componentesConexas[i] y CCAUnir1 = min (distancia a CCAUnir1, distancia a CCAUnir2) comentario: O(1) \\
        si CCAUnir2 esta mas cerca que CCAUnir1, aristaMasCortaHacia CCAUnir1 = aristaMasCortaHacia CCAUnir2 comentario: O(1) \\
        y lo mismo actualizo para la distancia y arista mas corta desde CCAUnir1 hacia la CC i comentario: O(1) \\
        voy guardando la distanciaMasCorta e indiceCCMasCerca de CCAUnir1 comentario: O(1) \\
        voy viendo si tengo que actualizar la distanciaMasCorta e indiceCCMasCerca de la CC i comentario: O(1)\\

al final recorro componentesConexas fijandome que indices representan componentes conexas, en estos indices de pueblo coloco una centra \\
y las tuberías son las aristas de la CC representada por estos indices comentario: O(n) \\
    
Total O(n ^ 2) \\


\subsubsection{Justificación formal de correctitud}

\subsubsection{Cota de complejidad temporal}

\subsubsection{Verificación mediante casos de prueba}

\end{document}
