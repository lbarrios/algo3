% ------ headers globales y begin ---------------
\documentclass[11pt, a4paper, twoside]{article}
\usepackage{header_tp2}

\begin{document}{}
% -----------------------------------------------
A continuación presentamos distintas instancias que sirven para 
verificar que el programa funciona correctamente.

Según la distribución de los pueblos en el mapa, y la relación entre cantidad total de pueblos y la cantidad 
máxima de centrales, podemos separar el conjunto de soluciones en 2 grandes casos: 

\begin{itemize}
	\item Todas las tuberías tienen la misma longitud: 
			\begin{itemize}
				\item Caso \#pueblos $\le$ \#centrales: \\
					En estos casos se coloca en todos los pueblos una central y se va a tener un riesgo mínimo 
					porque no hay tuberías (longitud de tuberías: 0). \\
					
					 \begin{minipage}{0.2\textwidth}
						\begin{tabular}{cc}
						   Input \\
						   \hline
						   3 & 4 \\
						   1 & 1 \\
						   2 & 2 \\
						   3 & 3 \\
						\end{tabular}
					\end{minipage} 
					\begin{minipage}{0.2\textwidth}
						\begin{tabular}{cc}
						   Output \\
						   \hline
						   3 & 0 \\
						   1 &  \\
						   2 &  \\
						   3 &  \\
						\end{tabular}
					\end{minipage} 
					
				\item Caso \#pueblos  $>$ \#centrales: \\
					Todas las tuberías tienen longitud 1 en este ejemplo. \\
					
					\begin{minipage}{0.2\textwidth}
						\begin{tabular}{cc}
						   Input \\
						   \hline
						   6 & 2 \\
						   1 & 1 \\
						   2 & 1 \\
						   1 & 2 \\
						   3 & 3 \\
						   3 & 2 \\
						   4 & 2 \\
						\end{tabular}
					\end{minipage} 
					\begin{minipage}{0.2\textwidth}
						\begin{tabular}{cc}
						   Output \\
						   \hline
						   2 & 4 \\
						   1 &  \\
						   4 &  \\
						   1 & 2 \\
						   3 & 1 \\
						   4 & 5 \\
						   6 & 5 \\
						\end{tabular}
					\end{minipage} 
					
			\end{itemize}
		
	\item Las tuberías tienen longitudes diferentes: 
		    \begin{itemize}
				\item Caso \#pueblos $>$ \#centrales: \\
				   Longitudes de las tuberías: 0, 1, 2. \\
				   \begin{minipage}{0.2\textwidth}
						\begin{tabular}{cc}
							   Input \\
							   \hline
							   6 & 2 \\
							   1 & 1 \\
							   1 & 2 \\
							   2 & 5 \\
							   3 & 1 \\
							   3 & 3 \\
							   4 & 1 \\
						\end{tabular}
					\end{minipage} 
					\begin{minipage}{0.2\textwidth}
						\begin{tabular}{cc}
						   Output \\
						   \hline
						   2 & 4 \\
						   1 &  \\
						   3 &  \\
						   1 & 2 \\
						   4 & 1 \\
						   4 & 6 \\
						   5 & 4 \\
						\end{tabular}
				   \end{minipage}  \\
			
				Misma distribución de los pueblos pero sólo teniendo una central: \\
				Longitudes de las tuberías: 1, 2, $\sqrt[] {5}$. \\
				   \begin{minipage}{0.2\textwidth}
						\begin{tabular}{cc}
						   Input \\
						   \hline
						   6 & 1 \\
						   1 & 1 \\
						   1 & 2 \\
						   2 & 5 \\
						   3 & 1 \\
						   3 & 3 \\
						   4 & 1 \\
						\end{tabular}
				   \end{minipage} 
				   \begin{minipage}{0.2\textwidth}
						\begin{tabular}{cc}
						   Output \\
						   \hline
						   1 & 5 \\
						   1 &   \\
						   1 & 2 \\
						   4 & 1 \\
						   4 & 6 \\
						   5 & 4 \\
						   3 & 5 \\
						\end{tabular}
				   \end{minipage} 
			\end{itemize}
		
\end{itemize}

Ejecutamos el programa con los distintos ejemplos y se llegó a la solución esperada. 
Por lo tanto, podemos concluir que el comportamiento del programa es correcto. 
\end{document}
