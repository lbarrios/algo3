% ------ headers globales y begin ---------------
\documentclass[11pt, a4paper, twoside]{article}
\usepackage{header_tp2}

\begin{document}{}
% -----------------------------------------------

Queremos demostrar que, partiendo de un grafo inicial $S_{n}$, compuesto por $n$ componentes
triviales, el resultante de aplicar $k$ iteraciones de Kruskal es una solución óptima. 
Para ello, aplicaremos inducción.

\textbf{Aviso:} Se cometerá un abuso de notación al indicar que se
``suma/resta una arista a una solución''; lo que se está efectuando es
realmente agregar o quitar la arista del conjunto de aristas de la solución.

\centerbf{Hipótesis Inductiva}
\texttt{P(i):} La solución $S_{n-i}$, obtenida a luego de aplicar $i$ veces Kruskal,
minimiza la \textbf{función objetivo} $f$
(\ref{ej2-funcionobjetivo}, pág. \pageref{ej2-funcionobjetivo})
frente a cualquier otra solución compuesta por $i$ componentes conexas.

\centerbf{Caso Inicial}
\texttt{P(1):} Es trivial, ya que cualquier grafo de $n$ nodos que contiene
${n-1}$ componentes conexas, contiene a lo sumo una sola arista. \nota{No sé si es tan trivial eso..}
Y ya que Kruskal elige en cada iteración la menor arista,
denotémosla $e_{1}$, el grafo $S_{n-1} = S{n} + e_{1}$ resultante es mínimo.

\centerbf{Paso Inductivo}
\texttt{P(i) $\rightarrow$ P(i+1):}
Sea $S_{n-(i+1)}$ la solución obtenida en el paso $(i+1)$ de \textbf{Kruskal},
podemos reescribir la misma de la forma $S_{n-(i+1)} = S_{n-i} + e_{i+1}$, en
donde $e_{i+1}$ es la arista agregada en este paso, y en donde \red{$S_{n-i}$ es
una solución óptima} según la \textbf{Hipótesis Inductiva}.

\nota{Después sigo...}


\end{document}