% ------ headers globales y begin ---------------
\documentclass[11pt, a4paper, twoside]{article}
\usepackage{header_tp2}

\begin{document}{}
% -----------------------------------------------
\nota{asegurarse de que el pseudocodigo este mas arriba que esto}

Veamos que nuestra implementación es una correcta versión del algoritmo de Kruskal. En cada iteración, Kruskal agrega a sus aristas la arista con menor peso
entre las que no forman ciclo con las que ya tiene. Es decir, que una dos nodos que no estaban conectados por ningún camino, o lo que es lo mismo, que pertenezcan a distintas componentes conexas.

Veamos que nuestro algoritmo elige la misma arista que Kruskal. Iteramos sobre todas las componentes y elegimos las dos que tienen la menor distancia hacia otra componente. Ahora veamos que las distancias de una componente hacia otra están bien calculadas.

En la etapa de inicialización, cuando tenemos n componentes conexas triviales, la distancia entre cualquier par de ellas es la distancia euclidea entre sus únicos nodos. Ahora la distancia entre dos componentes conexas no triviales, es, afín a la noción de distancia en conjuntos, la distancia más corta entre un nodo de una componente conexa y un nodo de la otra. Supongamos que conocemos la distancia de la componte conexa A hacia la B y la C. Luego la distancia entre A y $B \cup C$ es la distancia entre un nodo de A y un nodo de B o C, es decir el mínimo de la distancia mínima entre un nodo de A y un nodo de B, y la distancia mínima entre un nodo de B y un nodo de C. Se concluye esta relación: distancia entre A y $B \cup C$ = min(distancia entre A y B, distancia entre A y C).



\nota{aca poner algun titulo separador}

Queremos demostrar que, partiendo de un grafo inicial $S_{n}$, compuesto por $n$ componentes
triviales, el resultante de aplicar $k$ iteraciones de Kruskal es una solución óptima. 
Para ello, aplicaremos inducción.

\textbf{Aviso:} Se cometerá un abuso de notación al indicar que se
``suma/resta una arista a una solución''; lo que se está efectuando es
realmente agregar o quitar la arista del conjunto de aristas de la solución.

\centerbf{Hipótesis Inductiva}
\texttt{P(i):} La solución $S_{n-i}$, obtenida a luego de aplicar $i$ veces Kruskal,
minimiza la \textbf{función objetivo} $f$
(\ref{ej2-funcionobjetivo}, pág. \pageref{ej2-funcionobjetivo})
frente a cualquier otra solución compuesta por $i$ componentes conexas.

\centerbf{Caso Inicial}
\texttt{P(1):} Es trivial, ya que cualquier grafo de $n$ nodos que contiene
${n-1}$ componentes conexas, contiene a lo sumo una sola arista. \nota{No sé si es tan trivial eso..}
Y ya que Kruskal elige en cada iteración la menor arista,
denotémosla $e_{1}$, el grafo $S_{n-1} = S{n} + e_{1}$ resultante es mínimo.

\centerbf{Paso Inductivo}
\texttt{P(i) $\rightarrow$ P(i+1):}
Sea $S_{n-(i+1)}$ la solución obtenida en el paso $(i+1)$ de \textbf{Kruskal},
podemos reescribir la misma de la forma $S_{n-(i+1)} = S_{n-i} + e_{i+1}$, en
donde $e_{i+1}$ es la arista agregada en este paso, y en donde \red{$S_{n-i}$ es
una solución óptima} según la \textbf{Hipótesis Inductiva}.

\nota{Después sigo...}


\end{document}