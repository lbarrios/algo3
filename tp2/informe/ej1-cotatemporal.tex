% ------ headers globales y begin ---------------
\documentclass[11pt, a4paper, twoside]{article}
\usepackage{header_tp2}

\begin{document}{}
% -----------------------------------------------

Todas las operaciones y ciclos del pseudocódigo están debidamente anotados. Faltaría justificar \fixme{} que el ciclo que comienza en la línea 22, ``para subjuego en subjuegosPosibles'' es verdaderamente O(n).
El resto de los ciclos están explícitamente acotados por n, la cantidad de cartas.
La cantidad de subjuegos, es decir subconjuntos de cartas, que le podemos dejar al oponente, son siempre cartas contiguas. Es decir las cartas que vinieron en el orden de i a j, con 1 $\leq$ i $\leq$ j $\leq$ n. Ésto es siempre menor o igual a 2 $*$ n, que es O(n).

\end{document}