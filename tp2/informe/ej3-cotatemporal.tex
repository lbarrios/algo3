% ------ headers globales y begin ---------------
\documentclass[11pt, a4paper, twoside]{article}
\usepackage{header_tp2}

\begin{document}{}
% -----------------------------------------------

\centerbf{Pseudocódigo}
\begin{algorithm}[H]
\caption{La Centralita}
\begin{algorithmic}[1]

\Statex \texttt{Inicializo}
	\For{$(i=1..n, j=1..n, l=0..k)$}  \Comment{\bigO{n^2 * k}}
		\State distancia desde el casillero[i][j], sobrando l unidades extra de potencia = INF
	\EndFor \\

\State distancia hasta el casillero origen, sobrando k unidades extra de potencia = 0 \\
\State cola$<(int, int, int)>$ colaBFS
\State colaBFS.push(origen, k) \\

\While{(colaBFS no esté vacía)}  \Comment{\bigO{n^3 * k}} 
\State actual = colaBFS.pop()
	\For{cada casillero (x,y) al que puedo llegar desde actual}
		    \State l = unidades extra que quedan tras ir a (x,y) \\
			\Statex \hspace{1cm} \texttt{si no recorrí ya ese casillero, quedando esas unidades extra}
			\If{distancia a (x, y), quedando l unidades extra es menor a INF}
				\State   la pongo en 'distancia desde actual' + 1
				\If {es el casillero de destino} 
					\State break
				\EndIf
				\State colaBFS.push((x,y), unidades extra que quedan)
			\EndIf	
	\EndFor		
\EndWhile
\\
\State \Return distancia al destino

\end{algorithmic}
\end{algorithm}

Inicializar las distancias lleva $\bigO{n^2 * k}$.\\
En el ciclo \texttt{mientras} recorremos a lo sumo $n^2 * k$ nodos, ya que no recorremos dos veces el mismo nodo.\\
Para cada nodo miramos todos sus adyacentes, que son a lo sumo todas las casillas en la misma fila, o todas
las casillas en la misma columna, cada casilla con una cantidad de unidades de potencia extra única. \\
En peor caso miramos $2*n$ nodos, es decir $\bigO{n}$ nodos.\\
Luego la complejidad del ciclo es $\bigO{n^2 * k} * \bigO{n} = \bigO{n^3 * k}$.\\

\end{document}
