% ------ headers globales y begin ---------------
\documentclass[11pt, a4paper, twoside]{article}
\usepackage{header_tp2}

\begin{document}{}
% -----------------------------------------------

Como se pudo observar en el pseudocódigo (sección Planteamiento de Resolución), inicializar las distancias 
lleva $\bigO{n^2 * k}$.

En el ciclo \texttt{mientras} recorremos a lo sumo $n^2 * k$ nodos, ya que sólo agregamos a la cola nodos que no hayan sido recorridos previamente. Luego en cada iteración miramos un nodo distinto.

Para cada nodo miramos todos sus adyacentes, que son a lo sumo todas las casillas en la misma fila, o todas
las casillas en la misma columna. Y cada casilla a la que me muevo determina una única cantidad de potencia extra que me va a quedar, las que tenía menos las que usé en el salto.
En peor caso miramos $2*n$ nodos, es decir $\bigO{n}$ nodos.

Luego la complejidad del ciclo es $\bigO{n^2 * k} * \bigO{n} = \bigO{n^3 * k}$. La complejidad final queda en $\bigO{n^3 * k}$.

\end{document}
