%
%
%
% ESTA SECCION SE ENCUENTRA LEÍDA, CORREGIDA Y "CERRADA".
% PARA EVITAR INCONVENIENTES, POR FAVOR, 
% NO MODIFICAR A ULTIMO MOMENTO SIN AVISAR AL RESTO DEL GRUPO.
%
%
%

% ------ headers globales y begin ---------------
\documentclass[11pt, a4paper, twoside]{article}
\usepackage{header_tp2}
\begin{document}{}
% -----------------------------------------------

La cota temporal según anotado en el pseudocódigo es $O(n^2)$. Todos los ciclos,
excepto el último, son del estilo <<$for$>>, con la cantidad de iteraciones
claramente definida.

Faltaría entonces ver la cantidad de iteraciones que hace el ciclo que
reconstruye, a partir de las componentes conexas, el grafo resultante con sus
centrales y tuberías:

\hspace{0.03\linewidth}
\parbox{0.89\linewidth}{
  Se recorre el arreglo de $ComponentesConexas$ de tamaño $n$. \\

  Para cada componente conexa, representada por una posición del arreglo, voy
  agregando las aristas de la componente a las aristas del grafo. \\

  Son exactamente n - k aristas. Recorro a lo sumo $n$ componentes conexas, y
  entre todas agrego n - k aristas en O(1). \\

  Finalmente, la complejidad de este ciclo es $O(n)$. \\
}


\end{document}
