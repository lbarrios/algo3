%
%
%
% ESTA SECCION SE ENCUENTRA LEÍDA, CORREGIDA Y "CERRADA".
% PARA EVITAR INCONVENIENTES, POR FAVOR, 
% NO MODIFICAR A ULTIMO MOMENTO SIN AVISAR AL RESTO DEL GRUPO.
%
%
%
% ------ headers globales y begin ---------------
\documentclass[11pt, a4paper, twoside]{article}
\usepackage{header_tp2}

\begin{document}{}
% -----------------------------------------------

%-
%- MODELADO DEL PROBLEMA
%-
\centerbf{Modelado del problema}

Este es un problema típico en que una \textbf{estructura de grafos} permite
fácilmente realizar una visualización simple y práctica del escenario, como
así también encaminar el análisis del universo de soluciones hacia un posible
recorrido, construcción o \emph{deconstrucción}\footnote{<<Deshacer
analíticamente los elementos que constituyen una estructura conceptual>>, es
decir: desarmar, analizar, modificar y/o reconstruir una estructura, en este
caso un grafo, la cual se encuentra ya previamente armada, ya sea de forma
explícita o implícita.} de grafos.

En este caso en particular, representamos una \textbf{solución a una instancia} determinada del
problema como un \textbf{subgrafo generador ponderado no dirigido}, en donde cada pueblo está
representado por un vértice $v \in V$, y en donde cada cañería es una arista $e \in E_{S}$, siendo
$E_{S}$ subconjunto del conjunto de aristas del grafo completo de $n$ vértices. Refiriendonos a este
último detalle, y en un \emph{abuso de notación}, afirmamos que $S$ es subconjunto de $K_{n}$:
%
    \[
    S = (V,E_{S}) \subseteq K_{n} = (V,E_{K})
    \]
%

\begin{aclaracion}

De aquí en adelante se utilizará la letra K para referirse tanto al grafo
completo como a la cantidad de centralitas. Interpretar según el contexto.

\end{aclaracion}

%-
%- FUNCION PESO
%-
\centerbf{Función peso}

Establecemos además la \textbf{función peso}, $p: E \rightarrow \real$, siendo $e \in E$ una tupla
($inicio$,$fin$), en donde $inicio$ y $fin$ son dos vértices, como la distancia euclídea entre el
pueblo representado por el vértice de inicio y el pueblo representado por el vértice de fin, es
decir, el módulo del vector que resulta de la resta de ambos:
%
\[
  p(e) = dist(e.inicio, e.fin) =~\parallel inicio - fin \parallel ~=~ \sqrt[2]{(x_{fin}-x_{inicio})^{2}+(y_{fin}-y_{inicio})^{2}}
\]

\begin{paragraph}{}

%-
%- FUNCION OBJETIVO
%-
\centerbf{Función objetivo}\label{ej2-funcionobjetivo}

Se pide seleccionar una \textbf{solución óptima} a partir de un \textbf{conjunto de soluciones
factibles}, estableciendo la \textbf{función objetivo} $f: S \rightarrow \real$ como aquella que
dada una solución devuelve el peso de la mayor arista, la cual deberá ser minimizada:
%
\[
f(s) = max(\{p(e) : e \in E_{s}\})
\]

\end{paragraph}

\begin{samepage}

%-
%- CARACTERIZACIÓN DE LA SOLUCIÓN
%-
\centerbf{Caracterización de la solución}
Dado un conjunto de \textbf{soluciones factibles}, es evidente que las \textbf{soluciones óptimas}
son un subconjunto del mismo. Dada cualquier solución, sin importar si esta es factible o no, y
debido a que cada solución es un subgrafo generador de $K_{n}$, es posible determinar que la misma
contendrá una cantidad de componentes conexas mayor o igual a 1, y menor o igual a n. \nopagebreak

Dado que \textbf{se disponen de a lo sumo $k$ centralitas}, diremos que una solución es factible
cuando la misma se compone de \textbf{a lo sumo $k$ componentes conexas}.\nopagebreak

\textbf{Abuso de notación}: Representamos $S_{i}$ como ``una solución de \emph{i} componentes conexas''.\nopagebreak

Dada una solución factible $S_{x} = (V, E_{x}) \subseteq K_{n}$ (sin importar si esta es o no
óptima), si la misma está compuesta por una cantidad $x$ de \textbf{componentes conexas,
estrictamente menor a $k$}, entonces podemos afirmar que existe una solución factible $S_{k} (V,
E_{k}) \subseteq S_{x} \subseteq K_{n}$, de tal forma que el conjunto $E_{k}$ se forma a partir de
quitarle, ya sea una o sucesivas veces, la arista de mayor tamaño al conjunto $E_{x}$, repitiendo el
proceso siempre y cuando la solución resultante de cada eliminación de arista esté compuesta por a
lo sumo $k$ componentes conexas.

En otras palabras, \textbf{dada una solución de menos de $k$ componentes conexas}, siempre es
posible encontrar a partir de la misma \textbf{otra solución de exactamente $k$ componentes
conexas}. \darkred{\textbf{Decimos entonces que $f(s_{k}) <= f(s_{x})$}}, y la demostración de esto
es trivial ya que al depender f del peso de la máxima arista, es imposible que la valuación de la
misma aumente al retirar una o más aristas.

De esta forma, y a efectos de simplificar el análisis, podemos acotar el conjunto de soluciones
factibles por el de \textbf{todas las soluciones de exactamente $k$ componentes conexas}.

\end{samepage}

%-
%- IDEA DE RESOLUCIÓN
%-
\begin{paragraph}\
\centerbf{Idea de resolución}

Se utilizará de forma parcial el \textbf{Algoritmo de Kruskal}, es decir que partiendo de un grafo solución
inicial $S_{n}$, conformado por $n$ subgrafos triviales, y siendo $k$ el número de centralitas, el
ciclo será frenado al llegar a la <<$``n-k''~iteracion$>> o, dicho de otro modo, al formar un
subgrafo de $k$ componentes conexas. Decimos, entonces, que \textbf{el bosque resultante pertenece al
conjunto de soluciones óptimas}.
\end{paragraph}

%-
%- Pseudocódigo
%-
\begin{figure}
\centerbf{Pseudocódigo}
\begin{algorithm}[H]
\caption{Kruskal}\label{alg:ej2-kruskal}
\footnotesize\begin{algorithmic}[1]

\State ComponenteConexa componentesConexas[n] 
\\
\State estructura ComponenteConexa: 
\State \hspace{0.15cm} float distancias[n]
\State \hspace{0.15cm} list$<arista>$ aristas
\State \hspace{0.15cm} arista aristaMasCortaHacia[n]
\State \hspace{0.15cm} float distanciaMasCorta
\State \hspace{0.15cm} int indiceCCMasCerca
\\

\ForAll{i de 1 a n}  \Comment{$\bigO{n^2}$}
	\Statex \hspace{0.5cm} componentesConexas[i] = i-esimo pueblo \Comment{ \bigO{1}}
    \ForAll{j de 1 a n}    \Comment{ \bigO{n}}
        \State aristaMasCorta entre la CC i y la CC j = (i, j) \Comment{ \bigO{1}}
        \State distancia entre componentesConexas[i] y la componente conexa j = distanciaEuclídea(pueblo i, pueblo j) \Comment{ \bigO{1}} 
        \State me voy fijando cual de estas distancias es mas corta y la guardo junto con el indice j  \Comment{ \bigO{1}}
	\EndFor
\EndFor
\\

\ForAll{i de 1 a n - k} \Comment{ \bigO{(n * (n-k)) = \bigO{(n^2)}}} 

	\Statex \hspace{0.5cm} \texttt{// me fijo la distancia mas corta entre dos componentes}
    \ForAll{i in 1 to n}  \Comment{ \bigO{n}} 
        \Statex \hspace{0.9cm}si componentes[i] ya no representa más una componente \texttt{continúo}
        \State me voy fijando qué componente tiene la menor 'menor distancia hacia otra componente' 
		y la guardo en CCAUnir1, su componente mas cercana en CCAUnir2 \Comment{ \bigO{1}}
	\EndFor	
	\\
	
	\Statex \hspace{0.5cm} \texttt{// uno CCAUnir1 y CCAUnir2}
	\State CCAUnir1.aristas = CCAUnir1.aristas $\cup$ CCAUnir2.aristas + aristaMasCorta entre CCAUnir1 y CCAUnir2 \Comment{ \bigO{1}}
    \State marco CCAUnir2 como que ya no representa una componente conexa. (toda su información pasa a CCAUnir1)  \Comment{ \bigO{1}}
	\\
	
	\Statex \hspace{0.5cm} \texttt{// actualizo distancias}
	\ForAll{i de 1 a n} \Comment{ \bigO{n}} 
		\State si componentesConexas[i] ya no representa una componente conexa, \texttt{continúo} \Comment{ \bigO{1}} 
		\State distancia entre componentesConexas[i] y CCAUnir1 = min (distancia a CCAUnir1, distancia a CCAUnir2) \Comment{ \bigO{1}} 
        \State si CCAUnir2 esta mas cerca que CCAUnir1, aristaMasCortaHacia CCAUnir1 = aristaMasCortaHacia CCAUnir2 \Comment{ \bigO{1}} 
        \State con los mismos valores actualizo la distancia y arista mas corta desde CCAUnir1 hacia la CC i \Comment{ \bigO{1}} 
        \State voy guardando la distanciaMasCorta e indiceCCMasCerca de CCAUnir1 \Comment{ \bigO{1}} 
        \State voy viendo si tengo que actualizar la distanciaMasCorta e indiceCCMasCerca de la CC i \Comment{ \bigO{1}} 
	\EndFor
\EndFor 

\State al final recorro componentesConexas fijandome qué índices representan componentes conexas, 
	   en estos índices de pueblo coloco una central y las tuberías son la unión de las las aristas 
	   de las CC representadas por estos índices \Comment{ \bigO{n}} 


\Comment{Total \bigO{(n^2)}}
	
\end{algorithmic}
\end{algorithm}
\end{figure}

\end{document}
