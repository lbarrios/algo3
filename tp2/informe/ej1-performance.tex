% ------ headers globales y begin ---------------
\documentclass[11pt, a4paper, twoside]{article}
\usepackage{header_tp2}

\begin{document}{}

Para la medición de la performance del programa comparamos los tiempos en un caso aleatorio, un peor caso y 
un mejor caso, variando el tamaño de entrada ($n$ cantidad inicial de cartas). Además, quisimos comprobar que se 
cumpliera la cota teórica $\bigO{n^3}$ calculada. 

\begin{itemize}
\item El caso $aleatorio$ se armó tomando valores random para cada una de las $n$ cartas iniciales. 
\item El caso $mejor$ se produce cuando el juego dura sólo 1 turno. Esta situación se daría cuando las cartas son 
todas positivas. Esto se debe a que para lograr el mejor juego, sólo basta tomar todas las cartas en el primer y 
único turno. La suma de los valores de estas cartas siempre será el mejor puntaje que podrá obtener 
el Jugador1, frente a los 0 puntos que obtendrá el Jugador2. 
\item El caso $peor$ se produce cuando la cantidad de turnos totales durante el juego es máximo. 
Por lo que para los valores de las $n$ cartas iniciales se eligieron números negativos y todos iguales. 
En cada turno lo mejor que se puede hacer es tomar sólo una carta. Al finalizar el juego habrán pasado $n$ turnos, 
siendo este el mayor número de turnos posibles.
\end{itemize}

%grafico1 tiempo vs n (caso aleatorio, peor, mejor)

Por otro lado, para confirmar que el algoritmo tiene una complejidad de peor caso de $\bigO{n^3}$, decidimos 
realizar un gráfico tomando el cociente $\frac{tiempoDeEjecuci\acute{o}n}{n^3}$ y variando el tamaño de entrada $n$. 
Nuevamente elegimos valores negativos iguales para todas las cartas iniciales y medimos el tiempo de ejecución. \\
Se puede apreciar que a medida que crece $n$ la curva se estabiliza muy cerca de una constante, con lo cual 
concluímos que la cota temporal calculada fue correcta. 

%gráfico2  cociente vs n (caso peor)  
  

\end{document}
% -----------------------------------------------
