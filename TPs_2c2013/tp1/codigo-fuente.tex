% ------ headers globales y begin -----
\documentclass[11pt, a4paper, twoside]{article}
\usepackage{header_tp1}
\begin{document}{}
% -------------------------------------


% --- Problema 1
\subsubsection{Problema 1: Pascual y el correo}

\lstset{caption={ej1.cpp},label=ej1}
\begin{lstlisting}
#include "ej1.h"
int main( int argc, char** argv ){
	ParserDeParametros parser( argc, argv );
	if( ! parser.parametrosSonValidos() ) {
		parser.imprimirAyuda();
		return -1;
	}
	Ejercicio1 ejercicio = Ejercicio1();
	while( ejercicio.obtenerInput( parser.input() ) ){
		ejercicio.resolver();
		ejercicio.imprimirOutput( parser.output() );
	}
	return 0;
}
\end{lstlisting}

\lstset{caption={Ejercicio1.cpp},label=ejercicio1}
\begin{lstlisting}
void Ejercicio1::obtenerInput(istream &input){
	char testcase[16];
	input >> testcase;														// Parametro L
	if(testcase[0] != TESTCASE_NULL){
		_cargaMaxima = max(atoi(testcase),0);
		input >> _cantidadTotalDePaquetes;					// Parametro n
		_inicializarEstructurasInternas();					// Reserva espacio
		for(int i=0;i<_cantidadTotalDePaquetes;i++)
			input >> _listaDePesosDePaquetes[i];			// Parametros p1..pn
	} else _cargaMaxima = 0;
	_cantidadDeCamionesOcupados = 0;
}

bool Ejercicio1::inputNoEsNulo( void ){
	return _cargaMaxima > 0;
}

void Ejercicio1::resolver( void ){
	for(int i=0; i<_cantidadTotalDePaquetes; i++){
		int pesoPaquete = _listaDePesosDePaquetes[i];
		if( !_hayLugarParaCargar( pesoPaquete ) )
			_nuevoCamion ( pesoPaquete );
		else
			_agregaCarga ( pesoPaquete );
	}
}

void Ejercicio1::imprimirOutput( ostream &output ){
	output << _cantidadDeCamionesOcupados;
	for(int i=0; i<_cantidadDeCamionesOcupados; i++)
		output << " " << _listaDeCamiones[i].carga;
	output << endl;
}
\end{lstlisting}

\clearpage
\subsubsection{Problema 2: Profesores visitantes}

\lstset{caption={ej2.cpp},label=ej2}
\begin{lstlisting}
#include "ej2.h"
int main( int argc, char** argv ){
  ParserDeParametros parser( argc, argv );
  if( ! parser.parametrosSonValidos() ) {
    parser.imprimirAyuda();
    return -1;
  }
  Ejercicio2 ejercicio = Ejercicio2();
  while( ejercicio.obtenerInput( parser.input() ) ){
    ejercicio.resolver();
    ejercicio.imprimirOutput( parser.output() );
  }
  return 0;
}
\end{lstlisting}

\lstset{caption={Ejercicio2.cpp},label=ejercicio2}
\begin{lstlisting}
/* Inicializo con una instancia 'vacia' del ejercicio */
Ejercicio2::Ejercicio2(){ _cantidadTotalDeCursos = 0; }

bool Ejercicio2::obtenerInput(istream &input){
  char testcase[BASE10_INT_MAXSIZE];
  input >> testcase; 																// Parametro n
  if(testcase[0] != TESTCASE_NULL){
    _cantidadTotalDeCursos = max(atoi(testcase),0); // int n > 0 o invalido
    _inicializarEstructurasInternas(); 							// Reserva espacio
    for(uint i=0;i<_cantidadTotalDeCursos;i++){
    	uint inicio, fin;
      input >> inicio >> fin;												// Parametros ik fk
      _listaDeCursos[i]=Curso(inicio, fin, i+1);
    }
  } else return false; // Testcase == TESTCASE_NULL
	return true;
}

void Ejercicio2::resolver( void ){
  for(uint i=0; i<_cantidadTotalDeCursos; i++){
  	_listaOrdenadaDeCursos.insert( _listaDeCursos[i] );
  }
	set<Curso>::iterator it = _listaOrdenadaDeCursos.begin();
  uint fin = it->fechaDeFin, numeroDeCurso = it->numeroDeCurso;
	_cantidadDeCursosQueSeDictaran = 1;
  while( it != _listaOrdenadaDeCursos.end() ){
    if(it->fechaDeInicio > fin && it->fechaDeFin > fin){
      _listaDeCursosQueSeDictaran[_cantidadDeCursosQueSeDictaran-1] = numeroDeCurso;
      fin = it->fechaDeFin;
      numeroDeCurso = it->numeroDeCurso;
			_cantidadDeCursosQueSeDictaran++;
    }
    it++;
  }
  _listaDeCursosQueSeDictaran[_cantidadDeCursosQueSeDictaran-1] = numeroDeCurso;
}

void Ejercicio2::imprimirOutput( ostream &output ){
	int i=0;
	for( ;i<(int)_cantidadDeCursosQueSeDictaran-1; i++)
    cout << _listaDeCursosQueSeDictaran[i] << " ";
  cout << _listaDeCursosQueSeDictaran[i] << endl;
}
\end{lstlisting}
\clearpage

% --- Problema 3
\subsubsection{Problema 3: Una noche en el museo}

\lstset{caption={ej3.cpp : main},label=ej3-main}
\begin{lstlisting}
#include "ej3.h"
int main(int argc, char **argv)
{
	int filas, columnas;
	LCoord celdas_a_procesar;
	LSensor sensores_solucion;
	LSensor sensores_solucion_parcial;
	Solucion solucion(sensores_solucion, 0);
	Solucion solucion_parcial(sensores_solucion_parcial, 0);
	VectorMascaras mascaras_filas(16, 0);
	VectorMascaras mascaras_columnas(16, 0);

	// Paser de Input & Init
	cin >> filas;
	cin >> columnas;
	Grilla grilla(filas, Vec(columnas, 0));
	VectorOpciones opciones_filas(filas, MASK_OPCIONES_INICIALES);
	VectorOpciones opciones_columnas(columnas, MASK_OPCIONES_INICIALES);  
	cargar(grilla);

	if( analizar(grilla, celdas_a_procesar,
							 opciones_filas, opciones_columnas)
		){
		resolver(grilla, celdas_a_procesar, solucion_parcial,
						 opciones_filas, opciones_columnas, mascaras_filas,
						 mascaras_columnas, solucion);
	}

	// Imprimir resultado
	if( solucion.second == 0 ) cout << "-1" << endl;
	else {
	 cout << solucion.first.size() << " " << solucion.second << endl;
	 for( int j = 0; j < solucion.first.size();j++) {
		cout << solucion.first[j].second << " ";
		cout << solucion.first[j].first.first+1 << " ";
		cout << solucion.first[j].first.second+1 << endl;
	 }
	}
 return 0; 
}
\end{lstlisting}

\lstset{caption={ej3.cpp : analizar},label=ej3-analizar}
\begin{lstlisting}
bool analizar(Grilla &grilla,
							LCoord &celdas_a_procesar,
							VectorOpciones &opciones_filas,
							VectorOpciones &opciones_columnas)
{
  int filas = grilla.size();
  int columnas = grilla[0].size(); 
  vector<uint8_t> columna_valida(columnas, 0); 
  vector<uint8_t> fila_valida(filas, 0); 
  VCoord pared_previa_columna(columnas, celdas_a_procesar.end());
  LCoord::iterator pared_previa_fila = celdas_a_procesar.end();
  VCoord ultima_celda_libre_columna(columnas, celdas_a_procesar.end()); 
  LCoord::iterator ultima_celda_libre_fila = celdas_a_procesar.end();
  LCoord::iterator itr_celdas_a_procesar;
  
  // Se recorre grilla
  for( int i = 0; i < filas ;i++){
		for(int j = 0; j < columnas ; j++){
			if( grilla[i][j] == CELDA_IMPORTANTE ){
				grilla[i][j] = LASER_CRUZ;
				columna_valida[j] = columna_valida[j] | TIENE_CELDAS_IMPORTANTES;
				fila_valida[i] = fila_valida[i]  | TIENE_CELDAS_IMPORTANTES;
						// Dado que en la fila y columna hay una celda
						// de estado LaserC, tengo que modificar opciones.
				marcarFilaComoImportante(grilla, celdas_a_procesar, i,
																			 opciones_filas, pared_previa_fila);
				marcarColumnaComoImportante(grilla, celdas_a_procesar, j,
															opciones_columnas, pared_previa_columna[j]);
				// Si la Reliquia esta rodeada por paredes,
				// la grilla no tiene solucion. 
				if(estaRodeadaPorParedes(grilla, i, j)){
					return false;
				}
			
			} else if( grilla[i][j] == CELDA_LIBRE  ){
				grilla[i][j] = VACIA;
				columna_valida[j] = columna_valida[j] | TIENE_CELDAS_VACIAS;
				fila_valida[i] = fila_valida[i]  | TIENE_CELDAS_VACIAS;
				celdas_a_procesar.push_back(Coord(i, j));
				itr_celdas_a_procesar = celdas_a_procesar.end();
				itr_celdas_a_procesar--;
				ultima_celda_libre_fila  = itr_celdas_a_procesar;
				ultima_celda_libre_columna[j] = itr_celdas_a_procesar;
			
			} else if( grilla[i][j] == PARED  ){
				grilla[i][j] = grilla[i][j] | ( MASK_PARED << 4);
				columna_valida[j] = columna_valida[j] | TIENE_PAREDES;
				fila_valida[i] = fila_valida[i]  | TIENE_PAREDES;
				celdas_a_procesar.push_back(Coord(i, j)); // Iterador al ultimo
				itr_celdas_a_procesar = celdas_a_procesar.end();
				itr_celdas_a_procesar--;
				
				//Analizo longitud de sub-fila y sub-columna y modifico opciones 
				procesarFinDeSegmento(grilla, celdas_a_procesar,
									ultima_celda_libre_fila, ultima_celda_libre_columna[j]);
				analizarLargoSegmentoHorizontal(grilla, celdas_a_procesar, i, j,
																			 opciones_filas, pared_previa_fila);
				analizarLargoSegmentoVertical(grilla, celdas_a_procesar, i, j,
															opciones_columnas, pared_previa_columna[j]); 
				pared_previa_fila = itr_celdas_a_procesar;
				pared_previa_columna[j] = itr_celdas_a_procesar;
			}
		}
		
		// Aplico Mascara MASK_ULTIMA_CELDA_LIBRE_FILA
		// a la utlima celda libre de la fila
		if( ultima_celda_libre_fila != celdas_a_procesar.end() ){
			grilla[(*ultima_celda_libre_fila).first]
						[(*ultima_celda_libre_fila).second]
						= ( grilla[(*ultima_celda_libre_fila).first]
											[(*ultima_celda_libre_fila).second] )
							| ( MASK_ULTIMA_CELDA_LIBRE_FILA );
			ultima_celda_libre_fila = celdas_a_procesar.end();
		}
		pared_previa_fila = celdas_a_procesar.end();
  }
  
  // Aplico mascara MASK_ULTIMA_CELDA_LIBRE_COL
  // a la ultima celda de cada columna
  for( int j = 0; j<ultima_celda_libre_columna.size();j++){
		if( ultima_celda_libre_columna[j] != celdas_a_procesar.end() ){
			grilla[(*ultima_celda_libre_columna[j]).first]
						[(*ultima_celda_libre_columna[j]).second]
						= ( grilla[(*ultima_celda_libre_columna[j]).first]
											[(*ultima_celda_libre_columna[j]).second] )
							| ( MASK_ULTIMA_CELDA_LIBRE_COL );
			ultima_celda_libre_columna[j] = celdas_a_procesar.end();
		}
	}
	
	// Analizamos mascaras de estados para ver si las filas son validas,
	// es decir, si tienen solucion
	for( int i = 0; i < filas;i++){
		if( fila_valida[i] == 2 || fila_valida[i] == 3 ) return false;
	}
	// Analizamos mascaras de estados para ver si las columnas son validas,
	// es decir, si tienen solucion
	for( int i = 0; i < columnas;i++){
		if( columna_valida[i] == 2 || columna_valida[i] == 3 ) return false;
	}
	
	// Detectamos Sub-columnas de 1 Celda de longitud en la
	// ultima fila de la grilla
	if( filas > 1 ){
		for( int i = 0;i < columnas;i++){
			if( esVacia(grilla[filas-1][i]) == true
					&& esPared(grilla[filas-2][i]) == true ){
				grilla[filas-2][i]
					= grilla[filas-2][i] & ( MASK_PARED_INICIO_COL_1_CELDA << 4 );
			}
		}
	} else {
		for( int i = 0; i < columnas; i++){
			opciones_columnas[i]
				= opciones_columnas[i] & ( MASK_PARED_INICIO_COL_1_CELDA >> 9 );
			if( grilla[0][i] == LASER_CRUZ ) return false;
		}
	}
	
	// Detectamos Sub-filas de 1 Celda de longitud en la
	// ultima columna de la grilla
	if( columnas > 1 ){
		for( int i = 0;i < filas;i++){
			if( esVacia(grilla[i][columnas-1])
					&& esPared(grilla[i][columnas-2]) == true ){
				grilla[i][columnas-2]
					= grilla[i][columnas-2] & (MASK_PARED_INICIO_FILA_1_CELDA << 4);
			}
		}
	} else {
		for( int i = 0; i < filas; i++){
			opciones_filas[i]
				= opciones_filas[i] & MASK_PARED_INICIO_FILA_1_CELDA;
			if( grilla[i][0] == LASER_CRUZ ) return false;
		}
	}
	
	return true;
}
\end{lstlisting}

\clearpage
\lstset{caption={ej3.cpp : resolver},label=ej3-resolver}
\begin{lstlisting}
void resolver(Grilla &g,
							LCoord &celdas_a_procesar,
							Solucion &solucion_parcial,
							VectorOpciones &opciones_filas,
							VectorOpciones &opciones_columnas,
							VectorMascaras &mascaras_filas,
							VectorMascaras &mascaras_columnas,
							Solucion &solucion)
{
	LCoord::iterator itr = celdas_a_procesar.begin();
	uint32_t opciones_originales_fila; 
	uint32_t opciones_originales_columna; 
	int fila, columna;
	int tipo = 0;
	int valor_original;
	
	if( itr != celdas_a_procesar.end() ){
		//Calculo Interseccion Opciones Sensores entre Fila y columna
		Coord celda = (*itr);
		fila = celda.first;
		celdas_a_procesar.pop_front();
		columna = celda.second;
		//Guardo estados de opciones/requerimientos para fila/columna
		opciones_originales_fila = opciones_filas[fila];
		opciones_originales_columna = opciones_columnas[columna];
		valor_original = g[fila][columna];
		if( esPared(g[fila][columna]) ){
			opciones_filas[fila] = ( g[fila][columna] >> 4 ) & 0x1FF;
			opciones_columnas[columna] = ( g[fila][columna] >> 13 ) & 0x1FF;
			resolver(g, celdas_a_procesar, solucion_parcial, opciones_filas,
					opciones_columnas, mascaras_filas, mascaras_columnas, solucion);   
			opciones_filas[fila] = opciones_originales_fila;
			opciones_columnas[columna] = opciones_originales_columna;
		} else {
		// En el caso de que no sea una pared, seguro es una celda libre VACIA
			uint32_t requerimiento_fila = opciones_filas[fila] >> 7;
			uint32_t requerimiento_columna = opciones_columnas[columna] >> 7;
			uint32_t opciones_celdas =
				( opciones_originales_fila & opciones_originales_columna )
				& puedeContenerLaser(g, g[fila][columna], requerimiento_fila,
																									 requerimiento_columna);
			uint32_t precio = 0;
			for(int i = 1; i<= 6;i++){
				if( ( opciones_celdas >> i ) & 1 ){
					if( esSensor(i))
						solucion_parcial.first.push_back(Sensor(celda, i));
					g[fila][columna] = i;
					opciones_filas[fila] =
						( opciones_filas[fila] & mascaras_filas[i] & 0x7E )
						| nuevoRequerimientoFila(i, requerimiento_fila);
					opciones_columnas[columna] =
						( opciones_columnas[columna] & mascaras_columnas[i] & 0x7E )
						| nuevoRequerimientoCol(i, requerimiento_columna);
					precio = obtenerPrecio(i);
					if( solucion_parcial.second + precio < solucion.second
							|| solucion.second == 0){
						// Si solucion parcial sigue por debajo de la mejor
						// solucion encontrada, sumo precio sensor a solucion_parcial
						solucion_parcial.second += precio; 
						
						//lamo recursivamente
						resolver
						(g, celdas_a_procesar, solucion_parcial, opciones_filas,
						opciones_columnas, mascaras_filas, mascaras_columnas,solucion);
						//restauro precio
						solucion_parcial.second -= precio;
					}
					//Restauro estado previo a la seleccion de la opcion
					if( esSensor(i))
						solucion_parcial.first.pop_back();
					g[fila][columna] = valor_original;
					opciones_filas[fila] = opciones_originales_fila;
					opciones_columnas[columna] = opciones_originales_columna;
				}
			}
		}
		celdas_a_procesar.push_front(celda);
	} else {
	// Si ya no quedan celdas a procesar, tenemos una solucion
		
		// Si la nueva solucion es menos costosa que la actual, reemplazamos
		if( solucion_parcial.second < solucion.second || solucion.second == 0){
			solucion.second = solucion_parcial.second;
			solucion.first.clear();
			for( int j = 0; j < solucion_parcial.first.size();j++ )
				solucion.first.push_back(solucion_parcial.first[j]);
		}
	}
	return;
}
\end{lstlisting}

% -----------------------------------------------
\end{document}
