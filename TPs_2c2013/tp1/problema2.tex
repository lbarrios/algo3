% ------ headers globales y begin ---------------
\documentclass[11pt, a4paper, twoside]{article}
\usepackage{header_tp1}
\begin{document}{}
% -----------------------------------------------

\subsubsection{Descripción}

En este problema debemos maximizar la cantidad de cursos a dar a través del programa de Profesores
Visitantes que organiza el Departamento de Computación de la facultad.
Se invitó a un grupo de prestigiosos profesores extranjeros a que dicten
cursos en la facultad. Como es natural, cada profesor que puede concurrir
tiene una agenda bastante cargada, con lo cual son ellos quienes establecen
la disponibilidad. Esto tiene la desventaja de que seguramente existirán
cursos que se solapen en el tiempo, produciendo que si algún alumno
quisiera participar en todos ellos, no lo va a poder hacer. Debemos
garantizar que se ofrecerán la mayor cantidad de cursos sin que se solapen.

Los datos de los cursos vienen dados como pares $[inicio, fin]$ sin ningún tipo de orden.
Una primera aproximación podría ser analizar el universos de soluciones y quedarse con
la óptima. Este enfoque resultaría muy caro. Dado que el departamento
solicita una cota menor que \bigO{n^2}, tendremos que mejorar el enfoque.

Este tipo de problemas suelen tener buenas resoluciones mediantes técnicas
de algoritmos golosos. Algunas de las posibles estrategias serían: menor fecha de inicio,
menor fecha de fin, menor duración. Luego vamos a demostrar que tomar el curso
con menor fecha de fin cada vez es la estrategia que maximiza la cantidad de cursos 
a dictar con complejidad temporal en peor caso \bigO{n \log{n}}. Un ejemplo:

\[
[1,6] \quad [4,5] \quad [2,3] \quad [7,9] \quad [9,10] \quad [10,12]
\]

\begin{center}
	\begin{tabular}{ l l|c }
		\multicolumn{3}{ c }{Elección de cursos} \\
		\hline
		Menor fecha de inicio & $[1,6] \quad [7,9] \quad [10,12]$        & \textbf{\#3} \\
		Menor fecha de fin    & $[2,3] \quad [4,5] \quad [7,9] \quad [10,12]$ & \textbf{\#4} \\
		Menor duración        & $[2,3] \quad [4,5] \quad [9,10]$         & \textbf{\#3}
	\end{tabular}
\end{center}

\subsubsection{Hipótesis de resolución}

Para resolver el problema se plantea un algoritmo que a cada momento elije el curso
cuya fecha de finalización es menor que todos los demás cursos aún no seleccionados.
Si el curso no se solapa con el anterior (con lo cual tampoco se solapa con los anteriores
al anterior) entonces es el curso que mejor encaja en ese momento y se lo agrega
a los cursos ya seleccionados. Si se solapa, se lo descarta. El procedimiento se repite
hasta que no quedan cursos para analizar. En caso de haber dos cursos que finalicen
en la misma fecha, se elije aquel que comience antes~\footnote{al estar ordenados por fecha de fin,
la elección desempate según fecha de inicio es indistinta.
Si elijiendo por menor fecha de inicio en el desempate hubiese otro curso más que \enquote{entre},
entonces éste debiera tener menor fecha de fin, pues sino se solaparía.
El desempate por menor fecha de inicio se basa en el hecho de que sea un curso con mayor duración, cubriendo
de esta manera, no sólo la mayor cantidad de cursos a dictar sino también la mayor cantidad
de días ocupados.} (y obviamente no se solape) y
en caso de persistir el empate, se elije aquel con menor \enquote{id}, o sea, menor
en orden de entrada.

Nuestro algoritmo procesa el \enquote{elegir en cada momento} ordenando los
cursos por menor fecha de fin (desempatando como se describió) y luego se los recorre
linealmente. Este cambio no modifica el comportamiento pero logra generar un
mejor orden de complejidad temporal. Dados $n$ cursos se los puede ordenar en
\bigO{\log n} utilizando \textit{heapsort}. Para ello utilizamos la estructura
\texttt{set} de la \texttt{stl} de \texttt{C++} cuya complejidad cumple lo propuesto
ya que internamente está implementado como un árbol binario de búsqueda~\bib{stl:set}.
Se puede ver el comportamiento en el \alg{profes}.

\begin{algorithm}
\caption{Profesores Visitantes}\label{profes}
\begin{algorithmic}[1]
	\Require
		\Statex $cant \gets$ \Call{Cantidad de Cursos Ofrecidos}{} \Comment{$integer$}
		\Statex $listaFechas \gets$ \Call{Lista de Fechas Inicio y Fin}{} \Comment{$std::vector\langle integer \rangle$}
	\Ensure
		\Statex \Call{Lista de Números de Cursos}{} \Comment{$std::list\langle integer \rangle$}
	\Statex
	
	\State $conjuntoCursos \gets$ \Call{Conjunto}{$cant$} \Comment{$std::set\langle Curso \rangle$}
	\Statex
	
	\Function{Procesar Cursos}{$cant, listaFechas$}
		\State \Call{Cargar Cursos}{$cant, listaFechas$} \Comment{\bigO{n\log{n}}}
		\State \Return \Call{Selección de Cursos}{} \Comment{\bigO{n}}
	\EndFunction \Comment{\texttt{final} \bigO{n\log{n}}}
	\Statex
	
	\Procedure{Cargar Cursos}{$cant, listaFechas$}
		\State $i \gets 0$
		\While {$i < 2*cant$}
			\State $num \gets i \div 2$
			\State $inicio \gets listaFechas[i]$
			\State $fin \gets listaFechas[i+1]$
			\State $curso \gets$ \Call{Nuevo Curso}{$num, inicio, fin$} \Comment{\bigO{1}}
			\State \Call{Insertar Ordenado}{$conjuntoCursos, curso$} \Comment{std::set::insert, \bigO{\log n}}
			\State $i \gets i + 2$
		\EndWhile \Comment{\texttt{ciclo} \bigO{n}}
	\EndProcedure \Comment{\texttt{final} \bigO{n\log{n}}}
	\Statex
	
	\Function{Selección de Cursos}{}
		\State $listaNumeros \gets$ \Call{Nueva Lista}{}
		\State $curso \gets$ \Call{Próximo Curso}{$conjuntoCursos$} \Comment{\bigO{1}}
		\State $num \gets$ \Call{Numero}{$curso$}
		\State $fin \gets$ \Call{Fecha de Fin}{$curso$}
		\While {$proxCurso \gets$ \Call{Próximo Curso}{$conjuntoCursos$}}
			\State $proxNum \gets$ \Call{Numero}{$proxCurso$} \Comment{\bigO{1}}
			\State $proxInicio \gets$ \Call{Fecha de Inicio}{$proxCurso$} \Comment{\bigO{1}}
			\State $proxFin \gets$ \Call{Fecha de Fin}{$proxCurso$} \Comment{\bigO{1}}
			\If {$fin < proxInicio \wedge fin < proxFin$}
				\State \Call{Insertar}{$listaNumeros, num$} \Comment{\bigO{1}}
				\State $num \gets proxNum$
				\State $fin \gets proxFin$
			\EndIf
		\EndWhile \Comment{\texttt{ciclo} \bigO{n-1}}
		\State \Call{Insertar}{$listaNumeros, num$} \Comment{el último, \bigO{1}}
		\State \Return $listaNumeros$
	\EndFunction \Comment{\texttt{final} \bigO{n}}
\end{algorithmic}
\end{algorithm}

\clearpage
\subsubsection{Justificación formal de correctitud}

\begin{notac}
\emph{curso} = $\langle fechaInicio, fechaFin, ID \rangle$, tupla tal que $fechaInicio<fechaFin$ y $ID$ es único.
\end{notac}
\begin{definicion}
Considerando a $C$ como \emph{el conjunto de todos los cursos provenientes de la entrada del problema}, notaremos como $F_p$ = $\rho(C)$ al conjunto de todas las posibles soluciones, y a $F_v\subset F_p$ el subconjunto tal que
\[
	F_v = \{f : f \in F_p \wedge f \; \text{representa a una solucion válida}\}.
\]
\end{definicion}
\begin{definicion}
$\forall \ f \in F_v$, caracterizamos a $s(f)$ secuencia conformada por 
los cursos $c \in f$ de forma tal que se cumple que $\forall \ s_{i}(f), s_{i+1}(f)$
\[
	 fechaInicio(s_{i}(f)) < fechaFinal(s_{i}(f)) < fechaInicio(s_{i+1}(f)) < fechaFinal(s_{i+1}(f))
\]
Denotamos como $S(f)$ al conjunto de todas las posibles secuencias de la forma $s(f)$.
\end{definicion}
\begin{aclaracion}
Como la consigna requiere que no existan cursos solapados, asumimos que para todo conjunto de cursos que pueda
representar una \textbf{solución válida} existirá una y solo una secuencia ordenada según el criterio mencionado
anteriormente, de forma tal que para cada curso que se encuentre en la posición $i+1$ de la secuencia, 
la fecha de inicio es mayor a la fecha de fin del \textbf{curso} que se encuentre en la posición anterior 
($i$). Demostrarlo resulta trivial ya que, dado que todos los \textbf{cursos} tienen su propia fecha de finalización
posterior a su propia fecha de inicio, si asumiesemos que el criterio de orden estricto utilizado en la definición 
anterior no fuese satisfacible para alguna \textbf{solución válida}, entonces mediante dicha solución podríamos 
construir una secuencia en donde (al resultar imposible cumplir con el criterio de orden) debería 
existir al menos un par de cursos para el cual la fecha de inicio del primero de ellos sería menor a la fecha 
de finalización del otro, y en donde la fecha de inicio de este último seria menor a la fecha de finalización del
primero, lo cual es absurdo porque los cursos no se pueden solapar.
\end{aclaracion}
\begin{definicion}
Notamos $s(x) \in S(f)$ como la secuencia formada por los \textbf{cursos} de \textbf{nuestra solución}.
\end{definicion}
\begin{prop}
$\forall \ i \in \nat$, $s(f) \in S(f)$,
$\nexists s(f)[1 \dots i]$ subsecuencia de $s(f)$ formada por sus primeros $i$ elementos tal que
\[
	\nexists \ s(x)[1 \dots i] \lor fechaFinal(s_{i}(x)) > fechaFinal(s_{i}(f))
\]
\end{prop}
\begin{demostracion}\ \newline
\indent \textbf{Caso base} Suponemos que existe un $s_{1}(f)$ subsecuencia de $s(f)$ tal que
\[
	fechaFinal(s_{1}(f))<fechaFinal(s_{1}(x))
\]
Sabemos que el criterio de selección de 
nuestra solución toma siempre desde el conjunto de cursos que todavía no han sido 
elegidos o descartados el curso con menor fecha de finalización. Entonces, 
llegamos a un absurdo, puesto que como el primer curso que está tomando nuestra 
solución va a ser necesariamente alguno de los que menor fecha tenga con 
respecto a todo el conjunto de cursos, es contradictorio plantear la existencia de
otra solución cuyo primer curso tenga una fecha menor a esta. 

Suponemos entonces que existe un $s_{1}(f)$ subsecuencia de $s(f)$ tal que
$\nexists \ s(x)[1 \dots 1]$. Esto también es absurdo, porque si sabemos que existe 
al menos \enquote{\textbf{un curso}} (de forma tal que permite la existencia de la subsecuencia $s_{1}(f)$) 
en particular existe también \enquote{\textbf{un curso de menor fecha de finalización}}, y 
por lo tanto existe $s(x)[1 \dots 1]$.


\indent \textbf{Paso inductivo} $P(n) \then P(n+1)$

Suponemos que existe una subsecuencia $z = s(f)[1 \dots n+1]$ de $s(f)$, de forma tal 
que la misma contiene $n+1$ \textbf{cursos} de una \textbf{solucion válida}, y que
además es \textbf{potencialmente} mejor que \textbf{nuestra solución}, es decir que se 
cumple que $fechaFinal(z_{n+1}) < fechaFinal(s_{n+1}(x))$, 
o bien que $\nexists \ s(x)[1..n+1]$. 

En particular, podemos afirmar que al ser $z$ una subsecuencia de $s(f)$,
la subsecuencia $g = z - z_{n+1}$ (puesto que es una subsecuencia de $z$) 
va a seguir siendo también una subsecuencia de $s(f)$, y debido a que para
formarla le estamos quitando el último elemento a $z = s(f)[1 \dots n+1]$, 
va a tener la forma $g = s(f)[1 \dots n+1] - s_{n+1}(f) = s(f)[1 \dots n]$.

Por la \textbf{hipótesis inductiva}, sabemos que vale $\exists \ s(x)[1 \dots n]$, y
que además no existe subsecuencia $s(f)[1 \dots n]$ de $s(f)$ tal que se cumpla
$fechaFinal(s_{n}(f))<fechaFinal(s_{n}(x))$. Esto es es equivalente a afirmar que
\[
	(\forall \ s(f)[1 \dots n]) fechaFinal(s_{n}(x)) \leq fechaFinal(s_{n}(f))
\]
{}
(Esta última afirmación aplica también a $g$, ya que es subsecuencia de $s(f)$)


Finalmente, podemos concluir: por un lado que no es posible que 
se cumpla que dado $z \then \nexists \ s^{x}[1 \dots n+1]$, puesto que sabemos 
que $\exists \ s^{x}[1 \dots n]$, que $\exists \ g = z - z_{n+1}$, 
y que $fechaFinal(s_{n}(x)) \leq fechaFinal(g)$, nuestro algoritmo podría haber
tomado el curso $z_{n+1}$, el cual no se solapa con $g$, y por 
consiguiente no se solapa con $s_{n}(x)$. Por otro lado, tampoco es posible
que se cumpla que dado $z \then \exists \ s(x)[1 \dots n+1] \wedge 
fechaFinal(z_{n+1}) < fechaFinal(s_{n+1}(x))$, puesto que sabemos que
$\exists \ s(x)[1..n]$, que $z = s_{n}(f) + z_{n+1}$, y que
$ \ \forall s(f)[1..n], \; fechaFinal(s_{n}(x)) \leq fechaFinal(s_{n}(f))$,
y como \textbf{nuestra solución} toma siempre el de menor fecha de finalización, 
es absurdo plantear que $z_{n+1}$ pudiese terminar antes que $s_{n+1}(x)$, puesto
que en ese caso, como la fecha de finalización de $s_{n}(x)$ es menor a la de $z_{n}$,
debería haberlo elegido. \qed
\end{demostracion}

\begin{prop}
$s(x) \in S(f)$ (\textbf{nuestra solucion}) es óptima.
\end{prop}

\begin{demostracion}
Suponemos que existe una solución mejor a $s(x)$. En particular, esa solución 
va a ser válida, por lo que va a pertenecer a $S(f)$. Para ser mejor que
nuestra solución, esta debe como condición necesaria pero no suficiente,
cumplir las mismas condiciones expuestas en la proposición anterior, es decir,
volcándola en una secuencia bajo el mismo criterio de ordenamiento ya expuesto,
debería existir un $i$ para el cual, o bien esta nueva solución tenga en su
última posición una fecha de finalización menor a nuestra solución, o bien
nuestra solución disponga tan solo de $i-1$ elementos en su secuencia.
Aplicando el mismo tipo de razonamiento que para el procedimiento anterior, 
se llega al absurdo, lo cual demuestra que no puede existir una solución 
mejor que la planteada.
\end{demostracion}

\begin{comment}
Sean $FechaInicio, FechaFinal, ID$ enteros positivos, sea el tipo Curso una Tupla

\[
\langle FechaInicio, FechaFinal, ID \rangle
\]

tal que para toda Tupla $t$, $FechaInicio(t) < FechaFinal(t)$.
Sea un $c$ un conjunto de cursos. Se pide hallar una secuencia de $c$ sin repetidos tal que
para todo curso $t$ que está en $s$, también está en $c$, y para todo curso $t'$
que está en $s$, $FechaInicio(t) > FechaFinal(t')$ o $FechaFinal(t) < FechaInicio(t')$ o $t == t'$
y no existe $s'$ otra secuencia de cursos tal que sea solución del problema y 
$s'$ tenga mayor cantidad de cursos que $s$.

Dado algoritmo goloso que resuelve el problema, queremos demostrar que en cada paso
tomar el curso de menor $FechaFinal$ que no se superponga con el ninguno la secuencia ya formada
(secuencia ordenada a partir de comparaciones sobre la fecha final),
es el mejor candidato para agregar a nuestra secuencia solución.

Debemos tomar un curso $m$ que pertenezca a $c$ tal que para todo curso $m'$ distinto a $m$
que también esté en $c$ y para el curso $q$ el cual es el último que tomamos anteriormente,
el curso $FechaFinal(m) \leq FecheFinal(m')$ y además no se solapa con $q$.

Veamos el primer caso de una secuencia vacía. Por un lado veamos que $m$ es un buen candidato.
Supongamos que $m$ no fuera el primer elemento de la secuencia solución $s$ y existiera una 
secuencia $s'$ solución del problema donde $m$ no fuera el primer elemento de esa secuencia.
Entonces existiria una $m'$ primer elemento de $s'$ tal que $FechaFinal(m') > FechaFinal(m)$.
Si yo tomara la subsecuencia posterior a $m'$, ésta no se superpondría y le concatenara $m$, al
principio, esta secuencia sería solución, y tendría la misma longitud que $s'$, por lo tanto
es falso que pueda existir.

Para la segunda parte de la hipótesis la secuencia que voy formando es vacía, por lo tanto
no existe ningún Curso con el que $m$ se pueda superponer.

Entonces para el primer paso $m$ es un buen candidato. Ahora veamos que en particular
en cualquier instancia del problema $m$ es un buen candidato.
Supongamos que en alguna instancia del problema, existe el curso $m'$ tal que
$FechaFinal(m) < FechaFinal(m')$ y que $m'$ no pertenece a la solución $s$
y existe $s'$ solución del problema tal que $m'$ pertenece a $s'$ y $long(s) < long(s')$.

Como $s, s'$ son solución del problema, en particular existe una permutación de ellas tal que están
ordenadas. Supongamos que yo tomos esas permutaciones. (por motivos didacticos la voy a seguir llamnando $s, s'$).
Hay dos casos: $principio(s,m) < principio(s',m')$ o $final(s,m) < final(s',m')$.
Final es la subsecuencia desde el elemento que contiene una secuencia hasta el final de la
misma.

Veamos el primero. Llamemos a la subsecuencia $p:principio(m',s')$.
Si $FechaInicio(m') < FechaInicio(m)$, yo le puedo concatenar esta secuencia a $m$, por lo tanto
la $long$ es igual, asi que no puede existir este $m'$.

Si $FechaInicio(m') > FechaInicio(m)$, entonces existe un curso $p'$, tal que
$p'$ no está en $principio(s, m)$ y $FechaFinal(p') < FechaInicial(m)$.
Pero $FechaInicial(m) < FechaFinal(m)$ por definición de curso.
Bueno esto no puede pasar porque $m$ es el menor de todos los $FechaFinal(t)$ tal que 
$t$ pertenece a $c$ y $t$ no pertenece a $principio(s,m)$. Por lo tanto este curso no puede existir.

Veamos el segundo. Supongamos que existe la subsecuencia 
$f':final(s',m')$ tal que $long(f') > long(final(s,m))$ donde $FechaFinal(m) < FechaFinal(m')$.
Si $FechaInicio(m') < FechaInicio(m)$, entonces yo puedo concatenar $m$ con $f'$ y la longitud
quedaría igual, asi que no puede existir ese $m'$.
Si $FechaInicio(m) < FechaInicio(m')$ entonces existe ubn curso $p'$ tal que $p'$
pertenece a $f'$ y $p'$ no está en $final(s, m)$ y $FechaFinal(p') < FechaFinal(m)$.

Si $FechaInicial(p') < FechaInicial(m)$ se podría concatenar al principio
de $final(s, m)$ este elemento y la longitud quedaría igual.
Si $FechaInicial(p') > FechaInicial(m)$ no puede pasar porque $m$ era el elemento con menor
$FechaFinal$, de todos los elemento que no pertenecían a $Principio(s,m)$.
Por lo tanto este elemento $m'$ no puede existir y $m$ es siempre un buen candidato para $s$.

Descartar a todos los elementos que se superponen con $m$ a cada paso
no influye en la solución. Se puede ver que tomar cualquier elemento de un conjunto de
Curso que se superponen es lo mismo, ya que de todos ellos solo se
puedo elegir uno y la $long$ de la solución es la misma.
\end{comment}

\subsubsection{Cota de complejidad temporal}

El algoritmo ejecuta dos pasos. Primero recorre todos los cursos y los
inserta con orden en el conjunto ordenado. La recorrida toma una tiempo
linea \bigO{n} dado que se recorren todos los elementos en orden de entrada,
mientras que la inserción se hace en tiempo logarítmico al tratarse de una
estructura implementada en base a \textit{heapsort}. Este proceso tiene
una complejidad de \bigO{n \log n}.

En segunda instancia se hace la selección de cursos, en donde se recorren
nuevamente todos cursos tomando una complejidad de \bigO{n}. La complejidad
final es de \bigO{n \log n}.

\subsubsection{Verificación mediante casos de prueba}

Para verificar el correcto funcionamiento de nuestro programa tomamos
una serie de instancias que consideramos representativas de casos extremos
o sensibles y analizamos el comportamiento. Definimos los siguientes casos:

\begin{verbatim}
in1: 2 1 1 1 1                    | out1: 1
in2: 5 1 2 2 3 3 4 4 5 5 6        | out2: 1 3 5
in3: 4 1 2 3 4 5 6 7 8            | out3: 1 2 3 4
in4: 4 1 10 2 3 4 5 6 7           | out4: 2 3 4
in5: 6 1 6 4 5 2 3 7 9 9 10 10 12 | out5: 3 2 4 6
\end{verbatim}

La entrada 1 provee dos cursos con mismas fechas, con lo cual devuelve el primero
(en este caso el desempate llega hasta menor \enquote{id}).
En la entrada 2, cada curso se solapa con el siguiente, con lo cual
toma uno sí y otro no. En la entrada 3 son todos cursos contigüos,
entonces se seleccionan todos. La entrada 4 contiene cursos contigüos
salvo el primero que se solapa con todos, entonces, a éste no lo tiene
en cuenta y elije todos los demás. La entrada 6 es aquella que se usó como
ejemplo en la descripción del problema. Se puede ver que la elección
es consecuente con lo mostrado.

\clearpage
\subsubsection{Medición empírica de la performance}

\begin{figure}[H]
   \begin{center}
   \includegraphics[width=1.4\textwidth,angle=90]{img/ejercicio2.png}
   \caption{\textbf{Primer muestreo del Ejercicio 2}}
   \label{fig:ej2}
   \end{center}
\end{figure}

\begin{figure}[H]
   \begin{center}
   \includegraphics[width=1.4\textwidth,angle=90]{img/ejercicio2_final.png}
   \caption{\textbf{Segundo muestreo del Ejercicio 2}}
   \label{fig:ej2-final}
   \end{center}
\end{figure}

% -----------------------------------------------
\end{document}
