% ------ headers globales y begin ---------------
\documentclass[11pt, a4paper, twoside]{article}
\usepackage{header_tp1}
\begin{document}{}
% -----------------------------------------------

\subsection{Objetivos}
Mediante la realización de este trabajo práctico se pretende realizar una introducción a la implementación y el análisis de las técnicas algorítmicas básicas para resolución de problemas. 

Se analizan las técnicas de \textit{algoritmos golosos}, y de \textit{backtracking}.

\subsection{Pautas de trabajo}
Se brindan tres problemas, escritos en términos coloquiales, en donde para cada uno de ellos se requiere \textbf{encontrar un algoritmo} que brinde una \textbf{solución particular}, acotado por una determinada \textbf{complejidad temporal}. El algoritmo debe ser \textbf{implementado} en un lenguaje de programación a elección. Los datos son proporcionados y deben ser devueltos bajo formatos específicos de \textit{input} y de \textit{output}.

Posteriormente se deben realizar análisis teóricos y empíricos tanto de la de \textbf{correctitud} como de la \textbf{complejidad temporal} para cada una de las soluciones propuestas.

\subsection{Metodología utilizada}
Para cada ejercicio, se brinda primeramente una \textbf{descripción} del problema planteado, a partir de la cual se realiza una \textbf{abstracción} hacia un \textbf{modelo formal}, que permite tener un \textbf{entendimiento preciso} de las pautas requeridas. 

Se expone, cuando las hay, una \textbf{enumeración de las características} elementales del problema; estas son aquellas que permiten \textbf{encuadrarlo} dentro de una \textbf{familia de problemas} típicos.

Se desarrolla posteriormente un análisis del conjunto \textbf{universo de posibles soluciones}, caracterizando matemáticamente el concepto de \textbf{solución correcta} y, en los casos en que se solicita optimización, las condiciones que definen a la \textbf{solución particular} (o \textit{mejor solución}) que se encuadra dentro de las pretenciones del problema.

Luego de caracterizar para todo conjunto posible de entradas <<\textit{cómo se compone el conjunto solución}>> correspondiente, se desarrolla un \textbf{pseudocódigo} en el que se expone <<\textit{cómo llegar a ese conjunto}>>\footnote{Una explicación coloquial, obviando detalles puramente implementativos: arquitectura, lenguaje, etc.}.

Habiendo planteado la \textbf{hipótesis de resolución} se demuestra, de manera informal o mediante inferencias matemáticas según sea necesario, que el \textbf{algoritmo propuesto} realmente permite obtener la \textbf{solución correcta}\footnote{En caso de existir más de una solución correcta, se demuestra que el algoritmo obtiene al menos una de ellas o, dicho de otro modo, para problemas de optimización, se demuestra que ninguna del resto de las soluciones correctas es mejor que la solución propuesta por nuestro algoritmo}.

Después de demostrar la \textbf{correctitud de la solución}, y su \textbf{optimalidad} en caso de existir varias soluciones correctas, se realiza un análisis teórico de la \textbf{complejidad temporal} en donde se estima el comportamiento del algoritmo en términos de tiempo. Este análisis en particular se realiza con el objetivo de obtener una \textit{cota superior asintótica}\footnote{Aunque se mencionan, sobre todo en el caso del \textit{Algoritmo de Backtracking}, algunas <<familias de entrada>> particulares bajo las cuales el algoritmo propuesto presenta un comportamiento mucho mejor al peor caso.}.

Luego de calcular la \textbf{cota de complejidad temporal}, se realiza una \textbf{verificación} empírica, junto con una \textbf{exposición gráfica} de los resultados obtenidos, mediante la combinación de técnicas básicas de medición y análisis de datos.

\newpage
\subsection{Herramientas utilizadas}
Para la realización de este trabajo se utilizaron un conjunto de herramientas, las cuales se enumeran a continuación:

\begin{itemize}
  \item \texttt{C++} como lenguaje de programación
    \begin{itemize}
      \item \texttt{gcc} como compilador de C++
    \end{itemize}
  \item \texttt{python} y \texttt{bash} para la realización de scripts
    \begin{itemize}
      \item \texttt{python} para generar casos de prueba
      \item \texttt{bash} para automatizar las mediciones
      \item \texttt{python/matplotlib} para plotear los gráficos
    \end{itemize}
  \item \LaTeX\ para la redacción de este documento
  \item Se testeó Sistemas Operativos \hfill
    \begin{itemize}
      \item \texttt{Debian GNU/Linux}
      \item \texttt{Ubuntu}
      \item \texttt{FreeBSD}, compilando a través de \texttt{gmake}
      \item \texttt{Windows}, a través de \texttt{cygwin}
    \end{itemize}
\end{itemize}


% -----------------------------------------------
\end{document}
