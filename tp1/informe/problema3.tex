% ------ headers globales y begin ---------------
\documentclass[11pt, a4paper, twoside]{article}
\usepackage{header_tp1}
\begin{document}{}
% -----------------------------------------------

\subsubsection{Descripción}

Este problema consiste en ubicar en un tablero la mayor cantidad de piezas posibles siguiendo ciertas reglas.  

\begin{itemize}
	\item El tablero contiene $n \times m$ casilleros cuadrados, $n$ filas y $m$ columnas. 
	\item Piezas existentes: $1,...,n \times m$. (Cantidad total de piezas: $n \times m$).     
	\item Una pieza es cuadrada y puede tener de $1$ a $4$ colores distintos. A cada lado $(sup, izq, der, inf)$ le corresponde un color. 
	\item Las piezas no se pueden rotar.
	\item Colores posibles: $1,...,c$ ($c$ entero positivo). Como una pieza puede tener como mucho $4$ colores distintos y son $n \times m$ piezas en total, $ 1 \le c \le (4 \times n \times m)$.  
	\item 2 piezas pueden ubicarse en casilleros adyacentes sólo si sus lados adyacentes tienen el mismo color. Podría ocurrir que no sea posible llenar completamente el tablero con las piezas existentes. 
	\item El contenido final de una casilla podría ser $1,...,n \times m$, si se pudo colocar alguna ficha, o $0$ si quedara vacía. 
	\item Cantidad mínima de piezas que se pueden colocar en el tablero: $(n \times m)/ 2$. (Se intercalan las piezas en el tablero). \\
	Para un tablero de $3\times 3$ y uno de $2\times 2$, suponiendo un caso donde ninguna ficha puede colocarse adyacente a otra, una de las posibles soluciones sería la siguiente: \\
	\\
	En el tablero de $3\times 3$ se podrían colocar las fichas $1,2,3,4$ y $5$, y en el de $2\times 2$, las fichas $1$ y $2$.  \\

	\begin{minipage}{0.2\textwidth}
		\begin{tabular}{|c|c|c|}
			\hline
			 $1$ & $0$ & $2$ \\
			\hline
			 $0$ & $3$ & $0$  \\
			\hline 
			 $4$ & $0$ & $5$ \\
			\hline
		\end{tabular}
	\end{minipage}
	\begin{minipage}{0.2\textwidth}
		\begin{tabular}{|c|c|}
			\hline
			 $1$ & $0$ \\
			\hline
			 $0$ & $2$ \\
			\hline
		\end{tabular}
	\end{minipage}

\end{itemize} 

El problema se deberá resolver utilizando la técnica de $Backtracking$ eligiendo algunas podas para mejorar los tiempos de ejecución del programa. \\

Por ejemplo: \\ 
\\
Se tiene un tablero de $2\times 2$, los colores $1,2,3$ y las piezas \textbf{1},\textbf{2},\textbf{3},\textbf{4} $:$ \\
\\

\begin{minipage}{0.2\textwidth}
	\begin{tabular}{ |l l l|}
		\hline
			 & $1$     &       \\
		$3$  & \textbf{1} &   $2$ \\ 
			 & $2$     &       \\
		\hline
	\end{tabular}
\end{minipage}
\begin{minipage}{0.2\textwidth}
	\begin{tabular}{ |l l l|}
		\hline
			 & $3$     &       \\
		$2$  & \textbf{2} & $2$ \\ 
			 & $1$     &       \\
		\hline
	\end{tabular}
\end{minipage}
\begin{minipage}{0.2\textwidth}
	\begin{tabular}{ |l l l|}
		\hline
			 & $3$      &       \\
		$1$  & \textbf{2}  & $3$ \\ 
			 & $2$      &       \\
		\hline
	\end{tabular}
\end{minipage}
\begin{minipage}{0.2\textwidth}
	\begin{tabular}{ |l l l|}
		\hline
			 & $1$      &       \\
		$1$  & \textbf{4}  & $2$   \\ 
			 & $2$      &       \\
		\hline
	\end{tabular} 
\end{minipage}  \\

En este caso, la cantidad máxima de piezas que se pueden colocar en el tablero es $3$. Entonces las posibles soluciones serían: \\

\begin{minipage}{0.5\textwidth}
	\centering
	\begin{tabular}{ | l | l |}
		\hline 
		\textbf{1}  & \textbf{2} \\ 
		\hline 
		$0$  & \textbf{4} \\ 
		\hline
	\end{tabular}  \\
\end{minipage}
\begin{minipage}{0.5\textwidth}
	\begin{tabular}{ | l | l |}
		\hline 
		$0$     & \textbf{2} \\ 
		\hline 
		\textbf{3}  & \textbf{1} \\ 
		\hline
	\end{tabular} \\
\end{minipage}  \\

\subsubsection{Hipótesis de resolución}

\subsubsection{Justificación formal de correctitud}

\subsubsection{Cota de complejidad temporal}

\subsubsection{Verificación mediante casos de prueba}

A continuación presentamos distintas instancias que sirven para verificar que el programa funciona correctamente.\\ 

\begin{minipage}{0.5\textwidth}
	\begin{tabular}{llll}
		Input  \\
		\hline
		n       & m       & c       &         \\
		$sup_1$ & $izq_1$ & $der_1$ & $inf_1$ \\
		\vdots & \vdots   & \vdots  & \vdots  \\
		$sup_{nxm}$ & $izq_{nxm}$ & $der_{nxm}$ & $inf_{nxm}$ \\ 
	\end{tabular} \\  
\end{minipage}
\begin{minipage}{0.5\textwidth}	
	\begin{tabular}{lll}
		Output  \\
		\hline
		$x_1$ & \dots & $x_m$ \\
		\vdots&       & \vdots \\
		$x_n$ & \dots & $x_{nxm}$ \\
		 \\
	\end{tabular} \\
\end{minipage} \\
\\
n: cant. filas del tablero. \\
m: cant. columnas del tablero. \\
c: cant. colores distintos en las piezas. \\
$sup_i$, $izq_i$, $der_i$, $inf_i$: colores (entre 1 y c) de los lados de la pieza $i$. \\
$x_i$: número de la pieza en la casilla $i$ del tablero. ($0$ si no hay ninguna pieza). \\
\\

Separamos en casos y mostramos ejemplos para cada uno: 
\begin{itemize}
\item Todas las piezas son iguales:
	\begin{itemize}
		\item El tablero se completa. \\
			\begin{minipage}{0.2\textwidth}
				\begin{tabular}{l}
					Input  \\
					\hline
					$2$ $2$ $1$       \\
					$1$ $1$ $1$ $1$ \\
					$1$ $1$ $1$ $1$ \\
					$1$ $1$ $1$ $1$ \\
					$1$ $1$ $1$ $1$ \\
				\end{tabular} \\  
			\end{minipage}
			\begin{minipage}{0.2\textwidth}	
				\begin{tabular}{l}
					Output  \\
					\hline
					$1$ $2$ \\
					$3$ $4$ \\
					\\
					\\
					\\
				\end{tabular} \\
			\end{minipage} \\
			\begin{minipage}{0.2\textwidth}
				\begin{tabular}{l}
					Input  \\
					\hline
					$2$ $2$ $2$       \\
					$1$ $2$ $2$ $1$ \\
					$1$ $2$ $2$ $1$ \\
					$1$ $2$ $2$ $1$ \\
					$1$ $2$ $2$ $1$ \\
				\end{tabular} \\  
			\end{minipage}
			\begin{minipage}{0.2\textwidth}	
				\begin{tabular}{l}
					Output  \\
					\hline
					$1$ $2$ \\
					$3$ $4$ \\
					\\
					\\
					\\
				\end{tabular} \\
			\end{minipage} \\
			
			En estos ejemplos el tablero se completaría colocando las $4$ piezas de cualquier forma. No hay restricciones. \\

		\item El tablero tiene la mínima cantidad de piezas (($n \times m)/2$).\\
		
		    \begin{minipage}{0.2\textwidth}
				\begin{tabular}{l}
					Input  \\
					\hline
					$2$ $2$ $2$       \\
					$1$ $2$ $1$ $2$ \\
					$1$ $2$ $1$ $2$ \\
					$1$ $2$ $1$ $2$ \\
					$1$ $2$ $1$ $2$ \\
				\end{tabular} \\  
			\end{minipage}
			\begin{minipage}{0.2\textwidth}	
				\begin{tabular}{l}
					Output  \\
					\hline
					$1$ $0$ \\
					$0$ $2$ \\
					\\
					\\
					\\
				\end{tabular} \\
			\end{minipage} 
			\begin{minipage}{0.2\textwidth}	
				\begin{tabular}{l}
					Output  \\
					\hline
					$0$ $1$ \\
					$2$ $0$ \\
					\\
					\\
					\\
				\end{tabular} \\
			\end{minipage} \\
			
			En este caso sólo se pueden colocar las piezas intercaladas. Se podrían tomar $2$ piezas cualesquiera para la solución (Ej: piezas $3$ y $4$). \\
			
	\end{itemize}
\item Todas las piezas son diferentes:
     \begin{itemize}
		\item El tablero se completa. \\
		
		    \begin{minipage}{0.2\textwidth}
				\begin{tabular}{l}
					Input  \\
					\hline
					$3$ $3$ $8$     \\
				    $8$ $2$ $6$ $2$ \\ 
					$1$ $1$ $2$ $4$ \\
					$1$ $2$ $7$ $5$ \\
					$3$ $4$ $4$ $6$ \\
					$4$ $4$ $3$ $7$ \\
					$5$ $3$ $6$ $8$ \\
					$6$ $4$ $1$ $2$ \\
					$7$ $1$ $2$ $7$ \\
					$8$ $8$ $1$ $3$ \\
				\end{tabular} \\  
			\end{minipage}
			\begin{minipage}{0.2\textwidth}	
				\begin{tabular}{l}
					Output  \\
					\hline
					$9$ $2$ $3$ \\
					$4$ $5$ $6$ \\
					$7$ $8$ $1$ \\
					\\
					\\
					\\
					\\
					\\
					\\
					\\
				\end{tabular} \\
			\end{minipage} \\
		
		\item El tablero tiene la mínima cantidad de piezas (($n \times m)/2$). \\
		
			\begin{minipage}{0.2\textwidth}
				\begin{tabular}{l}
					Input  \\
					\hline
					$2$ $2$ $3$     \\
					$2$ $3$ $2$ $2$ \\
					$1$ $1$ $2$ $2$ \\
					$1$ $3$ $2$ $2$ \\
					$2$ $1$ $2$ $2$ \\
				\end{tabular} \\  
			\end{minipage}
			\begin{minipage}{0.2\textwidth}	
				\begin{tabular}{l}
					Output  \\
					\hline
					$1$ $0$ \\
					$0$ $2$ \\
					\\
					\\
					\\
				\end{tabular} \\
			\end{minipage} 
			\begin{minipage}{0.2\textwidth}	
				\begin{tabular}{l}
					Output  \\
					\hline
					$0$ $1$ \\
					$2$ $0$ \\
					\\
					\\
					\\
				\end{tabular} \\
			\end{minipage} \\
		
		En este caso sólo se pueden colocar las piezas intercaladas. Se podrían tomar $2$ piezas cualesquiera para la solución (Ej: piezas $3$ y $4$). \\
		
		\item El tablero no se completa pero se llena más que el mínimo posible. \\
		
			\begin{minipage}{0.2\textwidth}
				\begin{tabular}{l}
					Input  \\
					\hline
					$2$ $2$ $3$     \\
					$1$ $2$ $1$ $1$ \\
					$1$ $1$ $1$ $2$ \\
					$2$ $2$ $3$ $2$ \\
					$3$ $2$ $3$ $3$ \\
				\end{tabular} \\  
			\end{minipage}
			\begin{minipage}{0.2\textwidth}	
				\begin{tabular}{l}
					Output  \\
					\hline
					$1$ $2$ \\
					$0$ $3$ \\
					\\
					\\
					\\
				\end{tabular} \\
			\end{minipage} \\
	\end{itemize}
	
\item Existe alguna pieza diferente al resto: 
     \begin{itemize}
		\item El tablero se completa. \\
		
			\begin{minipage}{0.2\textwidth}
				\begin{tabular}{l}
					Input  \\
					\hline
					$2$ $2$ $2$     \\
					$2$ $2$ $1$ $2$ \\
					$2$ $2$ $1$ $2$ \\
					$2$ $1$ $2$ $2$ \\
					$2$ $1$ $2$ $2$ \\
				\end{tabular} \\  
			\end{minipage}
			\begin{minipage}{0.2\textwidth}	
				\begin{tabular}{l}
					Output  \\
					\hline
					$1$ $3$ \\
					$2$ $4$ \\
					\\
					\\
					\\
				\end{tabular} \\
			\end{minipage} \\
		
		\item El tablero tiene la mínima cantidad de piezas (($n \times m)/2$). \\
		
			\begin{minipage}{0.2\textwidth}
				\begin{tabular}{l}
					Input  \\
					\hline
					$2$ $2$ $3$     \\
					$2$ $2$ $1$ $2$ \\
					$3$ $3$ $3$ $1$ \\
					$3$ $2$ $1$ $1$ \\
					$2$ $2$ $1$ $2$ \\
				\end{tabular} \\  
			\end{minipage}
			\begin{minipage}{0.2\textwidth}	
				\begin{tabular}{l}
					Output  \\
					\hline
					$1$ $0$ \\
					$0$ $4$ \\
					\\
					\\
					\\
				\end{tabular} \\
			\end{minipage} 
			\begin{minipage}{0.2\textwidth}	
				\begin{tabular}{l}
					Output  \\
					\hline
					$0$ $1$ \\
					$2$ $0$ \\
					\\
					\\
					\\
				\end{tabular} \\
			\end{minipage} \\
		
		En este caso sólo se pueden colocar las piezas intercaladas. Se podrían tomar $2$ piezas cualesquiera para la solución (Ej: piezas $3$ y $4$). \\
		
		\item El tablero no se completa pero se llena más que el mínimo posible. \\
		
			\begin{minipage}{0.2\textwidth}
				\begin{tabular}{l}
					Input  \\
					\hline
					$2$ $2$ $3$     \\
					$2$ $2$ $1$ $2$ \\
					$3$ $2$ $3$ $1$ \\
					$3$ $1$ $3$ $2$ \\
					$2$ $2$ $1$ $2$ \\
				\end{tabular} \\  
			\end{minipage}
			\begin{minipage}{0.2\textwidth}	
				\begin{tabular}{l}
					Output  \\
					\hline
					$1$ $3$ \\
					$0$ $4$ \\
					\\
					\\
					\\
				\end{tabular} \\
			\end{minipage} \\
		
	\end{itemize}
		
\end{itemize}

Ejecutamos el programa con los distintos ejemplos y se llegó a la solución esperada. Por lo tanto, podemos concluir que el comportamiento del programa es correcto. 

\subsubsection{Medición empírica de la performance}

% -----------------------------------------------
\end{document}
