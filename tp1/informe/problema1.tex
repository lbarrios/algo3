% ------ headers globales y begin ---------------
\documentclass[11pt, a4paper, twoside]{article}
\usepackage{header_tp1}

\begin{document}{}
% -----------------------------------------------

\subsubsection{Descripción}
Un inspector de un puesto de control de camiones quiere contratar a un experto por $D$ días consecutivos para que éste pueda detectar si algún camión de la empresa \textit{il Ravioli} está transportando sustancias ilegales. Lo que se quiere es maximizar la cantidad de camiones inspeccionados, es decir que la mayor cantidad de camiones tienen que estar pasando por el puesto durante el período de contración del experto. Se conocen los días en que cada uno de los $n$ camiones de la empresa estarán pasando, pero esta información podría no estar dada en orden cronológico. Además, el problema se deberá resolver con una complejidad estrictamente mejor que $\mathcal{O}(n^{2})$, siendo $n$ la cantidad total de camiones. El resultado que se deberá obtener es el día inicial en que convendría contratar al experto y el total de camiones que serán inspeccionados por el mismo.\\
\\
Por ejemplo: 
\\
Si se contrata al experto por $D=3$ días consecutivos y en total hay $n=20$ camiones sospechosos ($c1,c2,...,c20$) que pasan por el puesto según se muestra a continuación: \\ 

\begin{itemize}
	\item Primer caso:
\end{itemize} 

\begin{tabular}{|l|l|l|l|l|l|l|l|l|l|l|}
	\hline
	Día          &  $1$  & $2$   & $3$   & $4$   & $5$   & $6$ & $7$ & $8$ & $9$   & $10$  \\
	\hline
	Camiones     &  $c5$ & $c10$ & $c14$ & $c4$  & $c15$ & $c11$ &   &     & $c3$  & $c1$  \\
				 &  $c6$ &       &       &       & $c16$ & $c12$ &   &     & $c20$ & $c2$  \\    
				 &	$c7$ &       &       &       & $c17$ & $c13$ &   &     &       &       \\  
				 &	$c8$ &       &       &       & $c18$ &       &   &     &       &       \\
				 &	$c9$ &       &       &       & $c19$ &       &   &     &       &       \\
	\hline
\end{tabular} \\


El primer día del período de contratación será $d=4$ y habrán sido inspeccionados, durante los días Día4, Día5 y Día6, un total de $c=9$ camiones. \\

\begin{itemize}
	\item Segundo caso:
\end{itemize} 

\begin{tabular}{|l|l|l|l|l|l|l|l|}
	\hline
	Día          &  $1$  & $2$   & $3$   & $4$    & $5$ & $6$ & $7$   \\
	\hline
	Camiones     &  $c5$ &       &       & $c15$  &     &     & $c11$ \\
				 &  $c6$ &       &       & $c16$  &     &     & $c12$ \\    
				 &	$c7$ &       &       & $c17$  &     &     & $c13$ \\  
				 &	$c8$ &       &       & $c18$  &     &     & $c3$  \\
				 &	$c9$ &       &       & $c19$  &     &     & $c20$ \\
				 &	$c10$&       &       &        &     &     &       \\
				 &	$c14$&       &       &        &     &     &       \\
				 &	$c4$ &       &       &        &     &     &       \\
				 &	$c1$ &       &       &        &     &     &       \\
				 &	$c2$ &       &       &        &     &     &       \\
	\hline
\end{tabular} \\

Para este caso, el primer día del período de contratación será $d=1$ y habrán sido inspeccionados, durante los días Día1, Día2 y Día3, un total de $c=10$ camiones. 


\subsubsection{Hipótesis de resolución}
Consideremos los días de llegada ordenados en forma creciente: $d_1 \le d_2 \le \dots \le d_n$ son enteros \underline{positivos} al igual que $D$. \\
Definimos P(x), donde $x \in \mathbb{N}$, como $\sum_{i=1}^{n} I(d_i)_{[x,x+D-1]}$. \\
Buscamos $d$ tal que $(\forall x)$ P(x) $\le$ P(d). \\
Veamos que $d \le d_n$: \\
P$(d_n)=1$ pues $d_n \in [d_n,d_n + D - 1]$ pero $\forall x > d_n$, P(x)$=0$ pues $d_i < x, \forall i \in 1...n$.\\
Luego, si $d > d_n, d$ no es óptimo. \\
Veamos que existe un $d$ óptimo tal que $d=d_i$ para algún $i \in 1...n$. \\
Sea $d'$ óptimo. \\
Sea $d=min_{i \in 1...n}$ $d_i \big| d_i \ge d'$. \\
P(d)$\ge$ P(d') pues $\forall i$ tal que $d_i \in [d',d'+D-1], d_i \in [d,d+D-1]$ ya que $\not\exists$ $j$ tal que $d' \le d_j < d$ y pues $d'+D-1 \le d+D-1$. \\
Luego, hemos reducido el espacio de solución a $\{$ $d_1,...,d_n$$\}$.

\subsubsection{Justificación formal de correctitud}
Sabemos que la soluci\'on pertenece al conjunto {d1, ... , dn}. Nuestro algoritmo itera sobre cada elemento di del conjunto calculando P(di), y guardando el elemento e con P(e) m\'aximo encontrado hasta el momento. Al final de las iteraciones, tenemos guardado el elemento e con P(e) maximo en todo el conjunto. Devolvemos e y P(e).
\subsubsection{Cota de complejidad temporal}
El algoritmo a analizar es el siguiente:

\begin{algorithm}
\caption{Algoritmo Camiones Sospechosos}\label{logpascual}
\footnotesize\begin{algorithmic}[1]
	\Require
		\Statex $intervaloInspector \gets$ \Call{dameIntervaloInspector}{} \Comment{$integer$}
		\Statex $cantidadDeCamiones \gets$ \Call{dameCantidadCamiones}{} \Comment{$integer$}
		\Statex $fechasCamiones \gets$ \Call{dameFechasCamiones}{} \Comment{$arreglo\langle integer \rangle$}
	\Ensure
		\Statex \Call{Fecha Óptima Inspector}{} \Comment{$integer$}
		\Statex \Call{Cantidad De Camiones Analizados}{} \Comment{$integer$}
	\Statex
	
  \State \Call{ordenar}{fechasCamiones} \Comment{\bigO{n.\log n}}
  \State $inicioInter \gets 0$ \Comment{\bigO{1}}
  \State $finInter \gets 0$ \Comment{\bigO{1}}
  \State $mDia \gets 0$ \Comment{\bigO{1}}
  \State $mCantidadCamiones \gets 0$ \Comment{\bigO{1}}

  \While {$finInter < cantidadDeCamiones$} \Comment{ \texttt{Como máximo n interaciones}}
    \While {$ \Call{DiferenciaMenorAInterInspector}{inicioInter, finInter} $}
      \State $ finInter \gets finInter+1$
    \EndWhile 
  \If {$\Call{CantCamionesEn}{inicioInter, finInterl} < mCantidadCamiones$} \Comment{\bigO{1}}
    \State $ mDia \gets inicioInter$ \Comment{\bigO{1}}
    \State $ mCantidadCamiones \gets \Call{CantCamionesEn}{inicioInter,finInter}$ \Comment{\bigO{1}}
  \EndIf {}
  \State $inicioInter \gets inicioInter +1$
  \EndWhile \Comment{\texttt{ciclo} \bigO{n}}
  \State \Return {$mDia, mCantCamiones$}
  \State


	\Function{CantCamionesEn}{$inicioInter, finInter$}
  \State \Return { $finInter - inicioInter$ } \Comment{\bigO{1}}
	\EndFunction \Comment{\texttt{final} \bigO{1}}
  \State

	\Function{DiferenciaMenorAInterInspector}{$inicioInter, finInter$}
    \State $diferencia \gets fechasCamiones_{finInter} - fechasCamiones_{inicioInter}$ \Comment{\bigO{1}}
    \State \Return {$ finInter < cantidadDeCamiones \texttt{ and } diferencia < intervaloInspector $} \Comment{\bigO{1}}
	\EndFunction \Comment{\texttt{final} \bigO{1}}

	\Statex{}
	
\end{algorithmic}
\end{algorithm}


Consideremos la complejidad del ciclo for. A lo largo del ciclo se van actualizando izq y der, y se va guardando el maximo que lleva  en si O(1).
¿Cuántas veces se actualizan izq y der? Pues, como van en forma creciente de 1 a n, exactamente n veces cada una. El ciclo for tiene complejidad O(2n) = O(n)
La complejidad del algoritmo es finalmente O(n log n).

Como se puede ver el algorítmo tiene 2 partes bien diferenciadas: Primero 
tiene un ordenamiento de un arreglo unidimensional. Para eso se usa el algorítmo de ordenamiento
que brinda a librería standard de \textit{c++}. En la documentación de la misma se puede apreciar
que su complijidad en el peor caso es de \bigO{n. \log n}\footnote{http://www.cplusplus.com/reference/algorithm/sort/}
donde \textit{n} es la distancia que hay entre el primer y el último elemento que se quieren ordenar.
En este caso se necesita ordenar todo el arreglo, por lo que termina siendo \bigO{n.\log n} con respecto
al tamaño de la entrada.

Luego hay un bucle while. El bucle \texttt{mientras} se repite mientras que el fin del intervalo a analizar
se encuentre dentro del arreglo de camiones.

Apenas se entra al ciclo \texttt{mientras} hay otro ciclo, el cuál tiene una complejidad de peor caso de \bigO{n}.
Este ciclo interno se encarga de hacer avanzar el puntero \textit{finInter} hasta que la diferencia entre
de las fechas entre el principio del intervalo a analizar y el final del intervalo a analizar sea menor a
\textit{intervaloInspector}. El peor caso posible en este ciclo sería que la diferencia entre el la fecha mas próxima
y la fecha mas lejana sea menor a intervalo inspector. En ese caso sale del bucle porque se pasó del máximo. Sim embargo
si eso ocurre también se deja de cumplir la condición de la guarda externa, por lo tanto el bucle externo también termina.

Es decir, ambas guardas dependen de la misma comparación entre \textit{finInter} y \textit{cantCamiones}, por lo tanto cuando
termine una va a terminar la otra. Por otra parte \textit{finInter} siempre avanza. No hay ningún caso en el cuál retroceda.
Esto significa que siempre va a recorrer exactamente \textit{cantCamiones}, es decir que en el peor de los casos va a haber
\texttt{n} iteraciones del ciclo. Como el resto de las operaciones son todas \bigO{1} podemos concluir que el ciclo completo
tiene una complejidad de \bigO{n}

De esta manera el algorítmo teermina teniendo una complejidad de \bigO{n.\log n + n} = \bigO{n. \log n}. De esta manera
la complejidad es estrictamente menor que $\bigO{n^{2}}$.









\subsubsection{Verificación mediante casos de prueba}

A continuación presentamos distintas instancias que sirven para verificar que el programa funciona correctamente. \\ 
\\
Input: [D n $d_1$...$d_n$], Output: [d c] \\
\\
D: cant. de días de contratación del experto \\
n: cant.  total de camiones que pasan por el puesto \\
$d_i$ con $1 \le i \le n$: días de llegada de los camiones \\
d: día inicial del período de contratación del experto \\
c: cant. de camiones inspeccionados por el experto \\  
\\
Sea $L$ el intervalo de días en que llegan los camiones. \\
$L$ $=$ (máx $d_i$ $-$ mín $d_i$ $+ 1$) con $1 \le i \le n$. \\

De los datos de entrada y salida sabemos que: 

\begin{itemize}
    \item $1$ $\le$ D, n, $d_i$ (con $1$ $\le$ i $\le$ n).
	\item $1 \le $ c $\le$ n 
	\item $1 \le $ mín $d_i$ $\le$ d $\le$ máx $d_i$ 
\end{itemize}

Entonces, podemos separar el conjunto de soluciones en los siguientes casos: \\

\begin{tabular}{|l|l|l|l|}
	\hline
	Caso &  Condición  						              & Ej.Input      & Ej.Output \\
	\hline
	$1$  &  D$=$L, d$=$mín $d_i$, c$=$n    				  & 3 3 1 2 3     & 1 3 \\	
	$2$  &	D$>$L, d$=$mín $d_i$, c$=$n     		      & 5 3 1 2 3     & 1 3 \\
	\hline	
	$3$  &	D$<$L, d$=$mín $d_i$, c$<$n     			  & 3 4 1 2 5 1   & 1 3 \\
	$4$  &	D$<$L, mín $d_i$ $<$ d $<$ máx $d_i$, c$<$n   & 2 5 1 3 4 7 4 & 3 3 \\
	$5$  &	D$=1$, d$=$máx $d_i$, c$<$n     			  & 1 4 1 5 5 5   & 5 3 \\	
	\hline
\end{tabular} \\

Si c$=$n podemos deducir que sólo es posible que d$=$mín $d_i$ y que D$\ge$L (casos $1,2$). De no ser así, quedarían camiones sin inspeccionar y eso estaría contradiciendo que c$=$n. En cambio si c$<$n, por lo que mencionamos anteriormente, sólo nos queda que el período de contratación del experto sea menor al intervalo $L$ (casos $3,4,5$). \\
Estos $5$ casos cubrirían todo el espacio de soluciones. Ejecutamos distintos ejemplos correspondientes a cada uno de ellos y obtuvimos la respuesta esperada. Por lo tanto, podemos concluir que el comportamiento del programa es correcto. 

\subsubsection{Medición empírica de la performance}

% -----------------------------------------------
\end{document}
