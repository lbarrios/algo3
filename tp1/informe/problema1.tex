% ------ headers globales y begin ---------------
\documentclass[11pt, a4paper, twoside]{article}
\usepackage{header_tp1}

\begin{document}{}
% -----------------------------------------------

\subsubsection{Descripción}
Un inspector de un puesto de control de camiones quiere contratar a un experto por $D$ d\'ias consecutivos para que \'este pueda detectar si alg\'un cami\'on de la empresa $il Ravioli$ est\'a transportando sustancias ilegales. Lo que se quiere es maximizar la cantidad de camiones inspeccionados, es decir que la mayor cantidad de camiones tienen que estar pasando por el puesto durante el per\'iodo de contraci\'on del experto. Se conocen los d\'ias en que cada uno de los $n$ camiones de la empresa estar\'an pasando, pero esta informaci\'on podr\'ia no estar dada en orden cronol\'ogico. Adem\'as, el problema se deber\'a resolver con una complejidad estrictamente mejor que $\mathcal{O}(n^{2})$, siendo $n$ la cantidad total de camiones. El resultado que se deber\'a obtener es el d\'ia inicial en que convendr\'ia contratar al experto y el total de camiones que ser\'an inspeccionados por el mismo.\\
\\
Por ejemplo: 
\\
Si se contrata al experto por $D=3$ d\'ias consecutivos y en total hay $n=20$ camiones sospechosos ($c1,c2,...,c20$) que pasan por el puesto seg\'un se muestra a continuaci\'on: \\

Primer caso: 

\begin{itemize}
\item D\'ia1: $c5$, $c6$, $c7$, $c8$, $c9$. 
\item D\'ia2: $c10$. 
\item D\'ia3: $c14$.
\item D\'ia4: $c4$.
\item D\'ia5: $c15$, $c16$, $c17$, $c18$, $c19$. 
\item D\'ia6: $c11$, $c12$, $c13$.
\item D\'ia7: 
\item D\'ia8: 
\item D\'ia9: $c3$, $c20$.
\item D\'ia10: $c1$, $c2$ .
\end{itemize}  

El primer d\'ia del per\'iodo de contrataci\'on ser\'a $d=4$ y habr\'an sido inspeccionados, durante los d\'ias D\'ia4, D\'ia5 y D\'ia6, un total de $c=9$ camiones. \\

Segundo caso:

\begin{itemize}
\item D\'ia1: $c5$, $c6$, $c7$, $c8$, $c9$, $c10$, $c14$, $c4$, $c1$, $c2$. 
\item D\'ia2:  
\item D\'ia3: 
\item D\'ia4: $c15$, $c16$, $c17$, $c18$, $c19$. 
\item D\'ia5: 
\item D\'ia6: 
\item D\'ia7: $c11$, $c12$, $c13$, $c3$, $c20$.
\end{itemize}  

Para este caso, el primer d\'ia del per\'iodo de contrataci\'on ser\'a $d=1$ y habr\'an sido inspeccionados, durante los d\'ias D\'ia1, D\'ia2 y D\'ia3, un total de $c=10$ camiones. 

\subsubsection{Hipótesis de resolución}

\subsubsection{Justificación formal de correctitud}

\subsubsection{Cota de complejidad temporal}

\subsubsection{Verificación mediante casos de prueba}

\subsubsection{Medición empírica de la performance}

% -----------------------------------------------
\end{document}
