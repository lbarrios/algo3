% ------ headers globales y begin ---------------
\documentclass[11pt, a4paper, twoside]{article}
\usepackage{header_tp1}
\begin{document}{}
% -----------------------------------------------

\subsubsection{Descripción}


Dada una cantidad $n$ de piezas $i$ de orfebrería, cada una de las cuales tienen una pérdida diaria de ganancia $d_i$ y una cantidad de días $t_i$ que lleva su fabricación, se quiere saber el orden en que el joyero Frank O. Yar tendría que fabricar y vender las mismas para minimizar las pérdidas. Cada pieza no entregada dará como mínimo una pérdida de $d_i$ por $t_i$, y a esto se le sumará $d_i$ por la cantidad de días que Frank haya estado fabricando otras piezas, ya que él sólo puede trabajar de a una pieza a la vez. El problema se deberá resolver con una complejidad temporal de $\mathcal{O}(n^{2})$, siendo $n$ la cantidad de piezas a fabricar. En el resultado se mostrarán: $P$, el monto total perdido de la ganancia, y las piezas $i1,...,in$ ordenadas. \\

\begin{itemize}
	\item Ejemplo $1$: 
\end{itemize}

Frank tiene que entregar $n=3$ piezas en total ($i:1,2,3$). Los datos de cada pieza $i$ son los siguientes: \\

\begin{tabular}{|l|l|l|}
	\hline
	Pieza &  Pérdida diaria & $\#$ días de fabr.\\
	\hline
	$1$   &     $1$         & $3$               \\
	\hline 
	$2$   &     $2$         & $3$               \\
	\hline 
	$3$   &     $3$         & $3$               \\
	\hline 
\end{tabular} \\ 
    
La fabricación de las $3$ piezas le llevará $9$ días en total y durante estos días se irán sumando las pérdidas diarias de ganancia correspondiente a cada pieza sin terminar. 

\begin{itemize}
	\item D\'ia1: $3+2+1$. 
	\item D\'ia2: $3+2+1$. 
	\item D\'ia3: $3+2+1$ $\rightarrow$ entrega pieza $3$.
	\item D\'ia4: $2+1$.
	\item D\'ia5: $2+1$. 
	\item D\'ia6: $2+1$ $\rightarrow$ entrega pieza $2$.
	\item D\'ia7: $1$
	\item D\'ia8: $1$
	\item D\'ia9: $1$ $\rightarrow$ entrega pieza $1$.
\end{itemize}  	
	
Como resultado se obtendría el orden $3,2,1$ y $P:30$, el monto total perdido de la ganancia. \\

\begin{itemize}
	\item Ejemplo $2$: 
\end{itemize}

Frank tiene que entregar $n=2$ piezas en total ($i:1,2$). Los datos de cada pieza $i$ son los siguientes: \\

\begin{tabular}{|l|l|l|}
	\hline
	Pieza &  Pérdida diaria & $\#$ días de fabr.\\
	\hline
	$1$   &     $1$         & $3$               \\
	\hline 
	$2$   &     $3$         & $1$               \\
	\hline 
\end{tabular} \\ 
    
La fabricación de las $2$ piezas le llevará $4$ días en total y durante estos días se irán sumando las pérdidas diarias de ganancia correspondiente a cada pieza sin terminar. 

\begin{itemize}
	\item Día1: $3+1$ $\rightarrow$ entrega pieza $2$. 
	\item Día2: $1$. 
	\item Día3: $1$.
	\item Día4: $1$ $\rightarrow$ entrega pieza $1$.
\end{itemize}  	
	
Como resultado se obtendría el orden $2,1$ y $P:7$, el monto total perdido de la ganancia. 

\subsubsection{Hipótesis de resolución}

Contamos con $n \in \mathbb{N}$, donde cada collar queda determinado por una $3$-upla: (p,d,t) donde \\
$p \in \{1,...,n\}$ es el número de collar $(p_i \neq p_j)$, \\
$d \in \mathbb{N}$ es la pérdida diaria de ganancia del collar, y \\
$t \in \mathbb{N}$ es el tiempo de fabricación (en días) del collar. \\
Se pide devolver el orden de fabricación de los collares que minimice la pérdida total. \\
El espacio de soluciones es luego, todos los vectores que son permutaciones de los primeros $n$ naturales. \\
\\
Sea $v = (v_1,v_2,...,v_n)$ y entiéndase, \\
$d[V_i]$ como la segunda componente del collar que tiene $p=v_i$ \\
$t[V_i]$ como la tercer componente del collar que tiene $p=v_i$. \\
Sea $\displaystyle{C(V)= \sum_{i=1}^{n} ((\sum_{j=1}^{i} t[v_i])d[v_i])}$ la función que queremos minimizar sobre el espacio de soluciones. La \underline{solución} $v$ cumple $(\forall V')$ C(v) $\le$ C(v').\\
Para cada $i \in 1,...n$ consideramos $\displaystyle{ \pi(i)= \frac{d[v_i]}{t[v_i]}}$. \\
Veamos que $v$ cumple $\displaystyle{(\forall i \in 1...n-1) \pi(i) \ge \pi(i+1)}$. \\
\\
Supongo que no, es decir, $\displaystyle{(\exists X \in 1...n-1) \pi(X) < \pi(X+1)}$. \\
\\
Sea V'$=(v_1,v_2,...,v_{X-1},v_{X+1},v_X,v_{X+2},...,v_n)$, es decir, se construye a partir de V con las posiciones X e X$+1$ intercambiadas. \\
Veamos que C(v')$<$C(v). \\
\\
C(v)$-$C(v')$= \displaystyle {\sum_{i=1}^{n} (d[v_i](\sum_{j=1}^{i} t[v_j]))- \sum_{i=1}^{n}(d[v'_i](\sum_{j=1}^{i} t[v'_j]))} = $\\

$ \underbrace{\sum_{i=1}^{x-1} (d[v_i]\sum_{j=1}^{i} t[v_j]-\sum_{i=1}^{x-1}d[v'_i]\sum_{j=1}^{i} t[v'_j]}_{0,\textnormal{ pues} \forall i \in 1...x-1, v_i=v'_i}$ + 
$\displaystyle{d[v_x]\sum_{j=1}^{x} t[v_j] - d[v'_x]\sum_{j=1}^{x}t[v'_j] +}$ \\ 
$\displaystyle{d[v_{x+1}]\sum_{j=1}^{x+1} t[v_j] - d[v'_{x+1}]\sum_{j=1}^{x+1}t[v'_j] + }$ 
$\underbrace{\sum_{i=x+2}^{n} (d[v_i]\sum_{j=1}^{i} t[v_j]-\sum_{i=x+2}^{n}d[v'_i]\sum_{j=1}^{i} t[v'_j]}_{0,\textnormal{ pues} \forall i \in x+2...n, v_i=v'_i \textnormal{y} \sum_{j=1}^{i}t[v_j]=\sum_{j=1}^{i}t[v'_j]} =$ \\
$\displaystyle{d[v_x](\sum_{j=1}^{x-1} t[v_j]+ t[v_x]) - d[v_{x+1}](\sum_{j=1}^{x-1} t[v_j] + t[v_{x+1}]) + }$ \\
$\displaystyle{d[v_{x+1}](\sum_{j=1}^{x-1} t[v_j] + t[v_x] + t[v_{x+1}]) - d[v_x](\sum_{j=1}^{x-1} t[v_j] + t[v_{x+1}] + t[v_x])=} $\\
$\displaystyle{-d[v_x] t[v_{x+1}] + d[v_{x+1}] t[v_x] > 0 \Leftrightarrow \frac{d[v_{x+1}]}{t[v_{x+1}]} > \frac{d[v_x]}{t[v_x]} } \Leftrightarrow $ \\
$\displaystyle{\pi(x+1) > \pi(x) }$ que asumimos como verdadero. \\
Abs! Pues $v$ era óptimo. Luego, $(\forall i \in 1...n-1) \pi(i) \ge \pi(i+1)$. \\
\\
Sabemos pues $v$ óptimo $\Rightarrow v$ ordenado no ascendentemente por $\pi$. \\
Sabemos que existe $v*$ óptima, luego existte $v*$ óptima y ordenada. \\
Veamos que $v$ ordenada $\Rightarrow v$ óptima, es decir que no existe $v$ ordenada y no óptima. \\
\\
C(v)$=$C(v*) pues uno siempre puede ir intercambiando dos posiciones $x$ y $x+1$ que mantengan el ordenamiento (es decir $\pi(x)=\pi(x+1))$ las veces que sea necesario hasta que $v=v*$ y luego, C(v)$=$C(v*) y por lo tanto, $v$ óptima. \\
\\
Veamos que se pueden intercambiar $x$ y $x+1$: \\
$v = (v_1,...,v_x,v_{x+1},...,v_n)$ \\
$v'= (v_1,...,v_{x+1},v_x,...,v_n)$ \\
\\
Desarrollando como hecho previamente, \\
$\displaystyle{C(v) - C(v') = -d[v_x]t[v_{x+1}] + d[v_{x+1}]t[v_x]=0 \Leftrightarrow }$\\
$\displaystyle{\frac{d[v_{x+1}]}{t[v_{x+1}]} = \frac{d[v_x]}{t[v_x]} \Leftrightarrow \pi(x+1) = \pi(x)}$ que asumimos verdadero. \\
\\
Luego, $v$ ordenado $\Rightarrow v$ óptimo. \\ 


\subsubsection{Justificación formal de correctitud}
Nuestro algoritmo ordena de forma decreciente por el cociente $\pi$, luego devolvemos un vector v óptimo. Iteramos sobre v con el indice i, acumulando la pérdida del subvector v[1..i]. Al final de las iteraciones tenemos entonces la pérdida completa correspondiente al vector v.

\newpage

\subsubsection{Cota de complejidad temporal}

\begin{algorithm}
\caption{La joya del Río de la plata}\label{logpascual}
\footnotesize\begin{algorithmic}[1]
	\Require
		\Statex $cantidadDePiezas \gets$ \Call{dameCantidadDePiezas}{} \Comment{$integer$}
		\Statex $listaDePiezas \gets$ \Call{dameListaDePiezas}{} \Comment{$vector<Pieza>$}
	\Ensure
		\Statex \Call{Orden Óptimo Piezas}{} \Comment{$vector<Pieza>$}
	\Statex
	
  \State $sumaTotal \gets 0$ \Comment{\bigO{1}}
  \State $tiempoAcum \gets 0$ \Comment{\bigO{1}}
  \State \Call{ordenar}{listaDePiezas, \textsc{compararPiezasMayorQue} } \Comment{\bigO{n.\log n}}
  \ForAll {$pieza$ \texttt{en} $listaDePiezas$} \Comment {\texttt{n} iteraciones}
    \State $tiempoAcum \gets tiempoAcum + pieza.tiempo$ \Comment{ \bigO{1}}
    \State $sumaTotal \gets sumaTotal + pieza.devaluaci\acute{o}n * tiempoAcum $ \Comment{ \bigO{1}}
    \EndFor \Comment{\texttt{ciclo}\bigO{n}}
  \State \Return $listaDePiezas, sumaTotal$
  \State

  \Statex{}
  \Function{compararPiezasMayorQue}{$pieza1, pieza2$} \Comment{\bigO{1}}
  \If {$(pieza1.devaluaci\acute{o}n/pieza1.tiempo) > (pieza2.devaluaci\acute{o}n/pieza2.tiempo$} \Comment{\bigO{1}}
    \State \Return {1} \Comment{\bigO{1}}
  \Else
    \State \Return {0 \Comment{\bigO{1}}}
  \EndIf
	\EndFunction \Comment{\texttt{final} \bigO{1}}
  \State

  \Statex
  \State Definición \textbf{Pieza}:
  \State \texttt{ integer } tiempo \Comment{tiempo de fabricación}
  \State \texttt{ integer } devaluación \Comment{cantidad de dinero perdido por unidad de tiempo}
	\Statex{}
	
\end{algorithmic}
\end{algorithm}



  Como se puede ver en el pseudo código la complejidad está dada por un Ordenamiento.
El ordenamiento se realiza utilizando un criterio específico de comparación que consiste
en comparar el cociente entre la devaluación diaria de una pieza y el tiempo que toma
fabricar esa pieza. La comparación entonces tiene una complejidad de \bigO{1}, por lo que
un ordenamiento basado en comparaciones tiene complejidad de \bigO{n.\log n}.

  Luego hay un ciclo \textbf{para cada} que itera una por cada elemento en \textit{listaDePiezas},
por lo tanto el ciclo se realiza \textit{cantidadDePiezas} veces. Cada iteración del ciclo realiza
sólo acciones \bigO{1}, por lo tanto el ciclo en si es \bigO{cantidadDePiezas}. La cantidad de piezas
es proporcional al tamaño de la entrada, por lo tanto el ciclo es \bigO{n}.

  En conclusión el algorítmo tiene una complejidad temporal \bigO{n + n. \log n}=\bigO{n. \log n}

  




\subsubsection{Verificación mediante casos de prueba}

A continuación presentamos distintas instancias que sirven para verificar que el programa funciona correctamente: \\ 

\begin{minipage}{0.2\textwidth}
	\begin{tabular}{ll}
		Input  \\
		\hline
		n &  \\
		$d_1$ & $t_1$ \\
		\vdots & \vdots \\
		$d_n$ & $t_n$ 
	\end{tabular} \\ 
\end{minipage}
\begin{minipage}{0.2\textwidth}
	\begin{tabular}{ll}
		Output  \\
		\hline
		$i_1$ \dots $i_n$ & P \\
		 \\
		 \\
		 \\
	\end{tabular} \\ 
\end{minipage}  \\
\\
n: cant. total de piezas  \\
$d_i$: cant. de pérdida diaria de la pieza $i$ \\
$t_i$: cant. de días de fabricación ($1 \le i \le n$) \\
$i_1$ \dots $i_n$: n piezas ordenadas \\
P: monto total perdido de la ganancia \\

Podemos separar el conjunto de soluciones en los siguientes casos: 
\begin{enumerate}
    \item $\frac{d_i}{t_i}$ es igual  $\forall$ $i$. 
	
		\begin{minipage}{0.2\textwidth}
			\begin{tabular}{l}
				Input  \\
				\hline
				$3$   \\
				$2$ $1$ \\
				$4$ $2$ \\
				$8$ $4$ 
			\end{tabular} \\ 
		\end{minipage}
		\begin{minipage}{0.2\textwidth}
			\begin{tabular}{ll}
				Output  \\
				\hline
				$1$   $2$  $3$ & $70$ \\
				$1$   $3$  $2$ & $70$ \\
				\vdots         & \vdots\\
				$3$   $2$  $1$ & $70$ \\
			\end{tabular} \\ 
		\end{minipage}  \\
		\textnormal{Cualquier permutación de las piezas $1,2,3$ es una solución válida para este caso. $P=70$ en los $6$ posibles outputs. Esto es porque el cociente entre la pérdida diaria y el tiempo de fabricación es igual para las $3$ piezas, $\frac{d_i}{t_i}=2, \forall i = 1,2,3$.} \\

	\item $\exists$ $j$ tal que $\frac{d_i}{t_i}$ $\ne$ $\frac{d_j}{t_j}$, j $\ne$ i, con j, i $=$ $1,...,n$. 
	
	\begin{itemize}
		\item Fijo $t_i$ y varío $d_i$, $\forall$ $i$. Entonces el orden va a estar determinado por la pérdida diaria $d_i$.
		
		\begin{minipage}{0.2\textwidth}
			\begin{tabular}{l}
				Input  \\
				\hline
				$5$   \\
				$1$ $2$ \\
				$2$ $2$ \\
				$3$ $2$ \\
				$5$ $2$ \\
				$4$ $2$ \\
			\end{tabular} \\ 
		\end{minipage}
		\begin{minipage}{0.3\textwidth}
			\begin{tabular}{ll}
				Output  \\
				\hline
				$4$ $5$ $3$ $2$ $1$ & $70$ \\
				\\
				\\
				\\
			    \\
				\\
			\end{tabular} \\ 
		\end{minipage}  
		\begin{minipage}{0.2\textwidth}
			\begin{tabular}{l|l}
				Pieza i & $d_i$  \\
				\hline
				$1$ & $1$  \\
				$2$ & $2$ \\
				$3$ & $3$ \\
				$4$ & $5$ \\
				$5$ & $4$ \\
                        \\
			\end{tabular} \\ 
		\end{minipage}
		
		\item Fijo $d_i$ y varío $t_i$ $\forall$ $i$. El orden depende del cociente entre $d_i$ y $t_i$.
		
		\begin{minipage}{0.2\textwidth}
			\begin{tabular}{l}
				Input  \\
				\hline
				$3$   \\
				$2$ $1$ \\
				$2$ $2$ \\
				$2$ $3$ \\
			\end{tabular} \\ 
		\end{minipage}
		\begin{minipage}{0.3\textwidth}
			\begin{tabular}{ll}
				Output  \\
				\hline
				$1$ $2$ $3$ & $20$ \\
				\\
				\\
				\\
			\end{tabular} \\ 
		\end{minipage}  
		\begin{minipage}{0.2\textwidth}
			\begin{tabular}{l|l}
				Pieza i & $\frac{d_i}{t_i}$  \\
				\hline
				$1$     & $2$ \\
				$2$     & $1$ \\
				$3$     & $0,67$ \\
				\\
			\end{tabular} \\ 
		\end{minipage}  \\
		
		\item Varío $d_i$ y $t_i$ $\forall$ $i$ pero sin que todos los $\frac{d_i}{t_i}$ sean iguales. El orden depende del cociente entre $d_i$ y $t_i$. 
		
		\begin{minipage}{0.2\textwidth}
			\begin{tabular}{l}
				Input  \\
				\hline
				$3$   \\
				$3$ $1$ \\
				$1$ $2$ \\
				$2$ $3$ \\
			\end{tabular} \\ 
		\end{minipage}
		\begin{minipage}{0.3\textwidth}
			\begin{tabular}{ll}
				Output  \\
				\hline
				$1$ $3$ $2$ & $14$ \\
				\\
				\\
				\\
			\end{tabular} \\ 
		\end{minipage}  
        	\begin{minipage}{0.2\textwidth}
			\begin{tabular}{l|l}
				Pieza i & $\frac{d_i}{t_i}$  \\
				\hline
				$1$     & $3$ \\
				$2$     & $0,5$ \\
				$3$     & $0,67$ \\
				\\
			\end{tabular} \\ 
		\end{minipage}  \\		
		
	\end{itemize}
	
\end{enumerate}

Estos casos cubrirían todo el espacio de soluciones. Ejecutamos distintos ejemplos correspondientes a cada uno de ellos y obtuvimos la respuesta esperada. Por lo tanto, podemos concluir que el comportamiento del programa es correcto. 

\subsubsection{Medición empírica de la performance}

% -----------------------------------------------
\end{document}
