% ------ headers globales y begin ---------------
\documentclass[11pt, a4paper, twoside]{article}
\usepackage{header_tp1}
\begin{document}{}
% -----------------------------------------------

\subsubsection{Descripción}

Dada una cantidad $n$ de piezas $i$ de orfebrería, cada una de las cuales tienen una pérdida diaria de ganancia $d_i$ y una cantidad de días $t_i$ que lleva su fabricación, se quiere saber el orden en que el joyero Frank O. Yar tendría que fabricar y vender las mismas para minimizar las pérdidas. Cada pieza no entregada dará como mínimo una pérdida de $d_i$ por $t_i$, y a esto se le sumará $d_i$ por la cantidad de días que Frank haya estado fabricando otras piezas, ya que él sólo puede trabajar de a una pieza a la vez. El problema se deberá resolver con una complejidad temporal de $\mathcal{O}(n^{2})$, siendo $n$ la cantidad de piezas a fabricar. En el resultado se mostrarán: $P$, el monto total perdido de la ganancia, y las piezas $i1,...,in$ ordenadas. \\

\begin{itemize}
	\item Ejemplo $1$: 
\end{itemize}

Frank tiene que entregar $n=3$ piezas en total ($i:1,2,3$). Los datos de cada pieza $i$ son los siguientes: \\

\begin{tabular}{|l|l|l|}
	\hline
	Pieza &  Pérdida diaria & $\#$ días de fabr.\\
	\hline
	$1$   &     $1$         & $3$               \\
	\hline 
	$2$   &     $2$         & $3$               \\
	\hline 
	$3$   &     $3$         & $3$               \\
	\hline 
\end{tabular} \\ 
    
La fabricación de las $3$ piezas le llevará $9$ días en total y durante estos días se irán sumando las pérdidas diarias de ganancia correspondiente a cada pieza sin terminar. 

\begin{itemize}
	\item D\'ia1: $3+2+1$. 
	\item D\'ia2: $3+2+1$. 
	\item D\'ia3: $3+2+1$ $\rightarrow$ entrega pieza $3$.
	\item D\'ia4: $2+1$.
	\item D\'ia5: $2+1$. 
	\item D\'ia6: $2+1$ $\rightarrow$ entrega pieza $2$.
	\item D\'ia7: $1$
	\item D\'ia8: $1$
	\item D\'ia9: $1$ $\rightarrow$ entrega pieza $1$.
\end{itemize}  	
	
Como resultado se obtendría el orden $3,2,1$ y $P:30$, el monto total perdido de la ganancia. \\

\begin{itemize}
	\item Ejemplo $2$: 
\end{itemize}

Frank tiene que entregar $n=2$ piezas en total ($i:1,2$). Los datos de cada pieza $i$ son los siguientes: \\

\begin{tabular}{|l|l|l|}
	\hline
	Pieza &  Pérdida diaria & $\#$ días de fabr.\\
	\hline
	$1$   &     $1$         & $3$               \\
	\hline 
	$2$   &     $3$         & $1$               \\
	\hline 
\end{tabular} \\ 
    
La fabricación de las $2$ piezas le llevará $4$ días en total y durante estos días se irán sumando las pérdidas diarias de ganancia correspondiente a cada pieza sin terminar. 

\begin{itemize}
	\item Día1: $3+1$ $\rightarrow$ entrega pieza $2$. 
	\item Día2: $1$. 
	\item Día3: $1$.
	\item Día4: $1$ $\rightarrow$ entrega pieza $1$.
\end{itemize}  	
	
Como resultado se obtendría el orden $2,1$ y $P:7$, el monto total perdido de la ganancia. 

\subsubsection{Hipótesis de resolución}
Esta en el jpg.
\subsubsection{Justificación formal de correctitud}
Nuestro algoritmo ordena de forma decreciente por el cociente $\pi$, luego devolvemos un vector v \'optimo. Iteramos sobre v con el indice i, acumulando la p\'erdida del subvector v[1..i]. Al final de las iteraciones tenemos entonces la p\'erdida completa correspondiente al vector v.
\subsubsection{Cota de complejidad temporal}
sort(arreglo, $\pi$) //O(n log (n))

perdida = 0
dias = 0

for i=1 to n    //O(n)
    dias = dias + ti    //O(1)
    perdida = perdida + dias * di   //O(1)

La complejidad del algoritmo es O(n log (n)).


\subsubsection{Verificación mediante casos de prueba}

A continuación presentamos distintas instancias que sirven para verificar que el programa funciona correctamente: \\ 

\begin{minipage}{0.2\textwidth}
	\begin{tabular}{ll}
		Input  \\
		\hline
		n &  \\
		$d_1$ & $t_1$ \\
		\vdots & \vdots \\
		$d_n$ & $t_n$ 
	\end{tabular} \\ 
\end{minipage}
\begin{minipage}{0.2\textwidth}
	\begin{tabular}{ll}
		Output  \\
		\hline
		$i_1$ \dots $i_n$ & P \\
		 \\
		 \\
		 \\
	\end{tabular} \\ 
\end{minipage}  \\
\\
n: cant. total de piezas  \\
$d_i$: cant. de pérdida diaria de la pieza $i$ \\
$t_i$: cant. de días de fabricación ($1 \le i \le n$) \\
$i_1$ \dots $i_n$: n piezas ordenadas \\
P: monto total perdido de la ganancia \\

Podemos separar el conjunto de soluciones en los siguientes casos: 
\begin{enumerate}
    \item $\frac{d_i}{t_i}$ es igual  $\forall$ $i$. 
	
		\begin{minipage}{0.2\textwidth}
			\begin{tabular}{l}
				Input  \\
				\hline
				$3$   \\
				$2$ $1$ \\
				$4$ $2$ \\
				$8$ $4$ 
			\end{tabular} \\ 
		\end{minipage}
		\begin{minipage}{0.2\textwidth}
			\begin{tabular}{ll}
				Output  \\
				\hline
				$1$   $2$  $3$ & $70$ \\
				$1$   $3$  $2$ & $70$ \\
				\vdots         & \vdots\\
				$3$   $2$  $1$ & $70$ \\
			\end{tabular} \\ 
		\end{minipage}  \\
		\textnormal{Cualquier permutación de las piezas $1,2,3$ es una solución válida para este caso. $P=70$ en los $6$ posibles outputs. Esto es porque el cociente entre la pérdida diaria y el tiempo de fabricación es igual para las $3$ piezas, $\frac{d_i}{t_i}=2, \forall i = 1,2,3$.} \\

	\item $\exists$ $j$ tal que $\frac{d_i}{t_i}$ $\ne$ $\frac{d_j}{t_j}$, j $\ne$ i, con j, i $=$ $1,...,n$. 
	
	\begin{itemize}
		\item Fijo $t_i$ y varío $d_i$, $\forall$ $i$. Entonces el orden va a estar determinado por la pérdida diaria $d_i$.
		
		\begin{minipage}{0.2\textwidth}
			\begin{tabular}{l}
				Input  \\
				\hline
				$5$   \\
				$1$ $2$ \\
				$2$ $2$ \\
				$3$ $2$ \\
				$5$ $2$ \\
				$4$ $2$ \\
			\end{tabular} \\ 
		\end{minipage}
		\begin{minipage}{0.3\textwidth}
			\begin{tabular}{ll}
				Output  \\
				\hline
				$4$ $5$ $3$ $2$ $1$ & $70$ \\
				\\
				\\
				\\
			    \\
				\\
			\end{tabular} \\ 
		\end{minipage}  
		\begin{minipage}{0.2\textwidth}
			\begin{tabular}{l|l}
				Pieza i & $d_i$  \\
				\hline
				$1$ & $1$  \\
				$2$ & $2$ \\
				$3$ & $3$ \\
				$4$ & $5$ \\
				$5$ & $4$ \\
                        \\
			\end{tabular} \\ 
		\end{minipage}
		
		\item Fijo $d_i$ y varío $t_i$ $\forall$ $i$. El orden depende del cociente entre $d_i$ y $t_i$.
		
		\begin{minipage}{0.2\textwidth}
			\begin{tabular}{l}
				Input  \\
				\hline
				$3$   \\
				$2$ $1$ \\
				$2$ $2$ \\
				$2$ $3$ \\
			\end{tabular} \\ 
		\end{minipage}
		\begin{minipage}{0.3\textwidth}
			\begin{tabular}{ll}
				Output  \\
				\hline
				$1$ $2$ $3$ & $20$ \\
				\\
				\\
				\\
			\end{tabular} \\ 
		\end{minipage}  
		\begin{minipage}{0.2\textwidth}
			\begin{tabular}{l|l}
				Pieza i & $\frac{d_i}{t_i}$  \\
				\hline
				$1$     & $2$ \\
				$2$     & $1$ \\
				$3$     & $0,67$ \\
				\\
			\end{tabular} \\ 
		\end{minipage}  \\
		
		\item Varío $d_i$ y $t_i$ $\forall$ $i$ pero sin que todos los $\frac{d_i}{t_i}$ sean iguales. El orden depende del cociente entre $d_i$ y $t_i$. 
		
		\begin{minipage}{0.2\textwidth}
			\begin{tabular}{l}
				Input  \\
				\hline
				$3$   \\
				$3$ $1$ \\
				$1$ $2$ \\
				$2$ $3$ \\
			\end{tabular} \\ 
		\end{minipage}
		\begin{minipage}{0.3\textwidth}
			\begin{tabular}{ll}
				Output  \\
				\hline
				$1$ $3$ $2$ & $14$ \\
				\\
				\\
				\\
			\end{tabular} \\ 
		\end{minipage}  
        	\begin{minipage}{0.2\textwidth}
			\begin{tabular}{l|l}
				Pieza i & $\frac{d_i}{t_i}$  \\
				\hline
				$1$     & $3$ \\
				$2$     & $0,5$ \\
				$3$     & $0,67$ \\
				\\
			\end{tabular} \\ 
		\end{minipage}  \\		
		
	\end{itemize}
	
\end{enumerate}

Estos casos cubrirían todo el espacio de soluciones. Ejecutamos distintos ejemplos correspondientes a cada uno de ellos y obtuvimos la respuesta esperada. Por lo tanto, podemos concluir que el comportamiento del programa es correcto. 

\subsubsection{Medición empírica de la performance}

% -----------------------------------------------
\end{document}
