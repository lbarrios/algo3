% ------ headers globales y begin ---------------
\documentclass[11pt, a4paper, twoside]{article}
\usepackage{header_tp1}
\begin{document}{}
% -----------------------------------------------

\subsubsection{Descripción}

Dada una cantidad $n$ de piezas $i$ de orfebrer\'ia, cada una de las cuales tienen una p\'erdida diaria de ganancia $d_i$ y una cantidad de d\'ias $t_i$ que lleva su fabricaci\'on, se quiere saber el orden en que el joyero Frank O. Yar tendr\'ia que fabricar y vender las mismas para minimizar las p\'erdidas. Cada pieza no entregada dar\'a como m\'inimo una p\'erdida de $d_i$ por $t_i$, y a esto se le sumar\'a $d_i$ por la cantidad de d\'ias que Frank haya estado fabricando otras piezas, ya que \'el s\'olo puede trabajar de a una pieza a la vez. El problema se deber\'a resolver con una complejidad temporal de $\mathcal{O}(n^{2})$, siendo $n$ la cantidad de piezas a fabricar. En el resultado se mostrar\'an: $P$, el monto total perdido de la ganancia, y las piezas $i1,...,in$ ordenadas. \\

\begin{itemize}
	\item Ejemplo $1$: 
\end{itemize}

Frank tiene que entregar $n=3$ piezas en total ($i:1,2,3$). Los datos de cada pieza $i$ son los siguientes: \\

\begin{tabular}{|l|l|l|}
	\hline
	Pieza &  Pérdida diaria & $\#$ días de fabr.\\
	\hline
	$1$   &     $1$         & $3$               \\
	\hline 
	$2$   &     $2$         & $3$               \\
	\hline 
	$3$   &     $3$         & $3$               \\
	\hline 
\end{tabular} \\ 
    
La fabricaci\'on de las $3$ piezas le llevar\'a $9$ d\'ias en total y durante estos d\'ias se ir\'an sumando las p\'erdidas diarias de ganancia correspondiente a cada pieza sin terminar. 

\begin{itemize}
	\item D\'ia1: $3+2+1$. 
	\item D\'ia2: $3+2+1$. 
	\item D\'ia3: $3+2+1$ $\rightarrow$ entrega pieza $3$.
	\item D\'ia4: $2+1$.
	\item D\'ia5: $2+1$. 
	\item D\'ia6: $2+1$ $\rightarrow$ entrega pieza $2$.
	\item D\'ia7: $1$
	\item D\'ia8: $1$
	\item D\'ia9: $1$ $\rightarrow$ entrega pieza $1$.
\end{itemize}  	
	
Como resultado se obtendr\'ia el orden $3,2,1$ y $P:30$, el monto total perdido de la ganancia. \\

\begin{itemize}
	\item Ejemplo $2$: 
\end{itemize}

Frank tiene que entregar $n=2$ piezas en total ($i:1,2$). Los datos de cada pieza $i$ son los siguientes: \\

\begin{tabular}{|l|l|l|}
	\hline
	Pieza &  Pérdida diaria & $\#$ días de fabr.\\
	\hline
	$1$   &     $1$         & $3$               \\
	\hline 
	$2$   &     $3$         & $1$               \\
	\hline 
\end{tabular} \\ 
    
La fabricaci\'on de las $2$ piezas le llevar\'a $4$ d\'ias en total y durante estos d\'ias se ir\'an sumando las p\'erdidas diarias de ganancia correspondiente a cada pieza sin terminar. 

\begin{itemize}
	\item D\'ia1: $3+1$ $\rightarrow$ entrega pieza $2$. 
	\item D\'ia2: $1$. 
	\item D\'ia3: $1$.
	\item D\'ia4: $1$ $\rightarrow$ entrega pieza $1$.
\end{itemize}  	
	
Como resultado se obtendr\'ia el orden $2,1$ y $P:7$, el monto total perdido de la ganancia. 

\subsubsection{Hipótesis de resolución}
Esta en el jpg.
\subsubsection{Justificación formal de correctitud}
Nuestro algoritmo ordena de forma decreciente por el cociente $\pi$, luego devolvemos un vector v \'optimo. Iteramos sobre v con el indice i, acumulando la p\'erdida del subvector v[1..i]. Al final de las iteraciones tenemos entonces la p\'erdida completa correspondiente al vector v.
\subsubsection{Cota de complejidad temporal}
sort(arreglo, $\pi$) //O(n log (n))

perdida = 0
dias = 0

for i=1 to n    //O(n)
    dias = dias + ti    //O(1)
    perdida = perdida + dias * di   //O(1)

La complejidad del algoritmo es O(n log (n)).


\subsubsection{Verificación mediante casos de prueba}

\subsubsection{Medición empírica de la performance}

% -----------------------------------------------
\end{document}
