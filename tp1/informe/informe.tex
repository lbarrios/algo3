% ------ headers globales -------------
\documentclass[11pt, a4paper, twoside]{article}
\usepackage{header_tp1}
\includeonly{instrucciones, codigo-fuente, conclusiones, introduccion, problema1, problema2, problema3} %arreglar y agregar problema3
\begin{document}{}
% -------------------------------------
% -- Carátula --
\newpage{\pagestyle{empty}% parametros para la caratula (caratula.sty)

\materia{Algoritmos y Estructuras de Datos III}
\submateria{Entrega de TP}
\titulo{Trabajo Práctico 1}
\subtitulo{Técnicas de Diseño de Algoritmos}
\fecha{Viernes 11 de Abril de 2014}
\integrante{Barrios, Leandro E.}{404/11}{ezequiel.barrios@gmail.com}
\integrante{Benegas, Gonzalo}{958/12}{gsbenegas@gmail.com}
\integrante{Duarte, Miguel}{904/11}{miguelfeliped@gmail.com}
\integrante{Niikado, Marina}{711/07}{mariniik@yahoo.com.ar}
\grupo{Grupo \%NUMERO\_DE\_GRUPO\%}

\maketitle
\cleardoublepage}
  % -- nota:
  %       si no esta dentro de \newpage{} se rompe la numeracion y el encabezadp
  %       \cleardoublepage es para que el indice no se imprima atras de la caratula
%~ \pagestyle{empty}% parametros para la caratula (caratula.sty)

\materia{Algoritmos y Estructuras de Datos III}
\submateria{Entrega de TP}
\titulo{Trabajo Práctico 1}
\subtitulo{Técnicas de Diseño de Algoritmos}
\fecha{Viernes 11 de Abril de 2014}
\integrante{Barrios, Leandro E.}{404/11}{ezequiel.barrios@gmail.com}
\integrante{Benegas, Gonzalo}{958/12}{gsbenegas@gmail.com}
\integrante{Duarte, Miguel}{904/11}{miguelfeliped@gmail.com}
\integrante{Niikado, Marina}{711/07}{mariniik@yahoo.com.ar}
\grupo{Grupo \%NUMERO\_DE\_GRUPO\%}

\maketitle
 
\setcounter{page}{1}

%-- Índice --
\newpage{\pagestyle{empty}\tableofcontents\cleardoublepage}

%-- Dentro de TP1 redefino ciertos comandos para que se pueda compilar todo individualmente --
\begin{TP1}
%-- Introduccion --
\section{Introducción}\label{sec:introduccion}
% ------ headers globales y begin ---------------
\documentclass[11pt, a4paper, twoside]{article}
\usepackage{header_tp1}
\begin{document}{}
% -----------------------------------------------

\subsection{Objetivos}
Mediante la realización de este trabajo práctico se pretende realizar una introducción a la implementación y el análisis de las técnicas algorítmicas básicas para resolución de problemas. 

Se analizan en particular las técnicas de \textit{algoritmos golosos}, y de \textit{backtracking}.

\subsection{Pautas de trabajo}
Se brindan tres problemas, escritos en términos coloquiales, en donde para cada uno de ellos se requiere \textbf{encontrar un algoritmo} que brinde una \textbf{solución particular}, acotado por una determinada \textbf{complejidad temporal}. El algoritmo debe, además, ser \textbf{implementado} en un lenguaje de programación a elección. Los datos son proporcionados, y deben ser devueltos bajo formatos específicos de \textit{input} y de \textit{output}.

Posteriormente se deben realizar análisis teóricos y empíricos tanto de la de \textbf{correctitud} como de la \textbf{complejidad temporal} para cada una de las soluciones propuestas.

\subsection{Metodología utilizada}
Para cada ejercicio, se brinda primeramente una \textbf{descripción} del problema planteado, a partir de la cual se realiza una \textbf{abstracción} hacia un \textbf{modelo formal}, que permite tener un \textbf{entendimiento preciso} de las pautas requeridas. 

Se expone, cuando las hay, una \textbf{enumeración de las características} elementales del problema; estas son aquellas que permiten \textbf{encuadrarlo} dentro de una \textbf{familia de problemas} típicos.

Se desarrolla posteriormente un análisis del conjunto \textbf{universo de posibles soluciones} (o factibles), caracterizando matemáticamente el concepto de \textbf{solución correcta} y, en los casos en que se solicita \textbf{optimización}\footnote{Es decir, que la solución pertenezca al \textit{subconjunto de soluciones que \textbf{maximicen} o \textbf{minimicen} una determinada función}}, se definen las condiciones que dan forma ya sea a todo el subconjunto de \textbf{soluciones óptimas} que se encuadran dentro de las pretenciones del problema, o a una \textbf{solución particular} dentro del mismo (la cual denominamos \textit{mejor solución}).

Luego de caracterizar para todo conjunto posible de entradas <<\textit{cómo se compone el conjunto solución}>> correspondiente, se desarrolla un \textbf{pseudocódigo} en el que se expone <<\textit{cómo llegar a ese conjunto}>>\footnote{Una explicación coloquial, obviando detalles puramente implementativos: arquitectura, lenguaje, etc.}.

Habiendo planteado la \textbf{hipótesis de resolución} se demuestra, de manera informal o mediante inferencias matemáticas según sea necesario, que el \textbf{algoritmo propuesto} realmente permite obtener la \textbf{solución correcta}\footnote{En caso de existir más de una solución correcta, se demuestra que el algoritmo obtiene al menos una de ellas o, dicho de otro modo, para problemas de optimización, se demuestra que ninguna del resto de las soluciones correctas es mejor que la solución propuesta por nuestro algoritmo}.

Después de demostrar la \textbf{correctitud de la solución}, y su \textbf{optimalidad} en caso de existir varias soluciones correctas, se realiza un análisis teórico de la \textbf{complejidad temporal} en donde se estima el comportamiento del algoritmo en términos de tiempo. Este análisis en particular se realiza con el objetivo de obtener una \textit{cota superior asintótica}\footnote{Aunque se mencionan, sobre todo en el caso del \textit{Algoritmo de Backtracking}, algunas <<familias de entrada>> particulares bajo las cuales el algoritmo propuesto presenta un comportamiento mucho mejor al peor caso.}.

Luego de calcular la \textbf{cota de complejidad temporal}, se realiza una \textbf{verificación} empírica, junto con una \textbf{exposición gráfica} de los resultados obtenidos, mediante la combinación de técnicas básicas de medición y análisis de datos.


% -----------------------------------------------
\end{document}

\newpage
%-- Instrucciones --
\section{Instrucciones de uso}\label{sec:instrucciones}
% ------ headers globales y begin ---------------
\documentclass[11pt, a4paper, twoside]{article}
\usepackage{header_tp1}
\begin{document}{}
% -----------------------------------------------

\subsection{Herramientas utilizadas}\label{subsec:instrucciones-herramientas}

Para la realización de este trabajo se utilizaron un conjunto de herramientas, las cuales se enumeran a continuación:

\begin{itemize}
  \item \texttt{C++} como lenguaje de programación
    \begin{itemize}
      \item \texttt{gcc} como compilador de C++
    \end{itemize}
  \item \texttt{python} y \texttt{bash} para la realización de scripts
    \begin{itemize}
      \item \texttt{python} para generar casos de prueba
      \item \texttt{bash} para automatizar las mediciones
      \item \texttt{python/matplotlib} para plotear los gráficos
    \end{itemize}
  \item \LaTeX\ para la redacción de este documento
  \item Se testeó Sistemas Operativos \hfill
    \begin{itemize}
      \item \texttt{Debian GNU/Linux}
      \item \texttt{Ubuntu}
      \item \texttt{FreeBSD}, compilando a través de \texttt{gmake}
      \item \texttt{Windows}, a través de \texttt{cygwin}
    \end{itemize}
\end{itemize}

% -----------------------------------------------
\end{document}
\newpage

%-- Desarrollo --
\section{Desarrollo del TP}\label{sec:desarrollo}
	
	%-- Problema 1 --
	\subsection{Problema 1: Camiones sospechosos}\label{subsec:problema1}
  No se reentregará este problema.
	%% ------ headers globales y begin ---------------
\documentclass[11pt, a4paper, twoside]{article}
\usepackage{header_tp1}

\begin{document}{}
% -----------------------------------------------

\subsubsection{Descripción}
Un inspector de un puesto de control de camiones quiere contratar a un experto por $D$ d\'ias consecutivos para que \'este pueda detectar si alg\'un cami\'on de la empresa $il Ravioli$ est\'a transportando sustancias ilegales. Lo que se quiere es maximizar la cantidad de camiones inspeccionados, es decir que la mayor cantidad de camiones tienen que estar pasando por el puesto durante el per\'iodo de contraci\'on del experto. Se conocen los d\'ias en que cada uno de los $n$ camiones de la empresa estar\'an pasando, pero esta informaci\'on podr\'ia no estar dada en orden cronol\'ogico. Adem\'as, el problema se deber\'a resolver con una complejidad estrictamente mejor que $\mathcal{O}(n^{2})$, siendo $n$ la cantidad total de camiones. El resultado que se deber\'a obtener es el d\'ia inicial en que convendr\'ia contratar al experto y el total de camiones que ser\'an inspeccionados por el mismo.\\
\\
Por ejemplo: 
\\
Si se contrata al experto por $D=3$ d\'ias consecutivos y en total hay $n=20$ camiones sospechosos ($c1,c2,...,c20$) que pasan por el puesto seg\'un se muestra a continuaci\'on: \\

Primer caso: 

\begin{itemize}
\item D\'ia1: $c5$, $c6$, $c7$, $c8$, $c9$. 
\item D\'ia2: $c10$. 
\item D\'ia3: $c14$.
\item D\'ia4: $c4$.
\item D\'ia5: $c15$, $c16$, $c17$, $c18$, $c19$. 
\item D\'ia6: $c11$, $c12$, $c13$.
\item D\'ia7: 
\item D\'ia8: 
\item D\'ia9: $c3$, $c20$.
\item D\'ia10: $c1$, $c2$ .
\end{itemize}  

El primer d\'ia del per\'iodo de contrataci\'on ser\'a $d=4$ y habr\'an sido inspeccionados, durante los d\'ias D\'ia4, D\'ia5 y D\'ia6, un total de $c=9$ camiones. \\

Segundo caso:

\begin{itemize}
\item D\'ia1: $c5$, $c6$, $c7$, $c8$, $c9$, $c10$, $c14$, $c4$, $c1$, $c2$. 
\item D\'ia2:  
\item D\'ia3: 
\item D\'ia4: $c15$, $c16$, $c17$, $c18$, $c19$. 
\item D\'ia5: 
\item D\'ia6: 
\item D\'ia7: $c11$, $c12$, $c13$, $c3$, $c20$.
\end{itemize}  

Para este caso, el primer d\'ia del per\'iodo de contrataci\'on ser\'a $d=1$ y habr\'an sido inspeccionados, durante los d\'ias D\'ia1, D\'ia2 y D\'ia3, un total de $c=10$ camiones. 

\subsubsection{Hipótesis de resolución}

\subsubsection{Justificación formal de correctitud}

\subsubsection{Cota de complejidad temporal}

\subsubsection{Verificación mediante casos de prueba}

\subsubsection{Medición empírica de la performance}

% -----------------------------------------------
\end{document}

	\newpage
	
	%-- Problema 2 --
	\subsection{Problema 2: La joya del Río de la Plata}\label{subsec:problema2}
  No se reentregará este problema.
	%% ------ headers globales y begin ---------------
\documentclass[11pt, a4paper, twoside]{article}
\usepackage{header_tp1}
\begin{document}{}
% -----------------------------------------------

\subsubsection{Descripción}

Dada una cantidad $n$ de piezas $i$ de orfebrería, cada una de las cuales tienen una pérdida diaria de ganancia $d_i$ y una cantidad de días $t_i$ que lleva su fabricación, se quiere saber el orden en que el joyero Frank O. Yar tendría que fabricar y vender las mismas para minimizar las pérdidas. Cada pieza no entregada dará como mínimo una pérdida de $d_i$ por $t_i$, y a esto se le sumará $d_i$ por la cantidad de días que Frank haya estado fabricando otras piezas, ya que él sólo puede trabajar de a una pieza a la vez. El problema se deberá resolver con una complejidad temporal de $\mathcal{O}(n^{2})$, siendo $n$ la cantidad de piezas a fabricar. En el resultado se mostrarán: $P$, el monto total perdido de la ganancia, y las piezas $i1,...,in$ ordenadas. \\

\begin{itemize}
	\item Ejemplo $1$: 
\end{itemize}

Frank tiene que entregar $n=3$ piezas en total ($i:1,2,3$). Los datos de cada pieza $i$ son los siguientes: \\

\begin{tabular}{|l|l|l|}
	\hline
	Pieza &  Pérdida diaria & $\#$ días de fabr.\\
	\hline
	$1$   &     $1$         & $3$               \\
	\hline 
	$2$   &     $2$         & $3$               \\
	\hline 
	$3$   &     $3$         & $3$               \\
	\hline 
\end{tabular} \\ 
    
La fabricación de las $3$ piezas le llevará $9$ días en total y durante estos días se irán sumando las pérdidas diarias de ganancia correspondiente a cada pieza sin terminar. 

\begin{itemize}
	\item D\'ia1: $3+2+1$. 
	\item D\'ia2: $3+2+1$. 
	\item D\'ia3: $3+2+1$ $\rightarrow$ entrega pieza $3$.
	\item D\'ia4: $2+1$.
	\item D\'ia5: $2+1$. 
	\item D\'ia6: $2+1$ $\rightarrow$ entrega pieza $2$.
	\item D\'ia7: $1$
	\item D\'ia8: $1$
	\item D\'ia9: $1$ $\rightarrow$ entrega pieza $1$.
\end{itemize}  	
	
Como resultado se obtendría el orden $3,2,1$ y $P:30$, el monto total perdido de la ganancia. \\

\begin{itemize}
	\item Ejemplo $2$: 
\end{itemize}

Frank tiene que entregar $n=2$ piezas en total ($i:1,2$). Los datos de cada pieza $i$ son los siguientes: \\

\begin{tabular}{|l|l|l|}
	\hline
	Pieza &  Pérdida diaria & $\#$ días de fabr.\\
	\hline
	$1$   &     $1$         & $3$               \\
	\hline 
	$2$   &     $3$         & $1$               \\
	\hline 
\end{tabular} \\ 
    
La fabricación de las $2$ piezas le llevará $4$ días en total y durante estos días se irán sumando las pérdidas diarias de ganancia correspondiente a cada pieza sin terminar. 

\begin{itemize}
	\item Día1: $3+1$ $\rightarrow$ entrega pieza $2$. 
	\item Día2: $1$. 
	\item Día3: $1$.
	\item Día4: $1$ $\rightarrow$ entrega pieza $1$.
\end{itemize}  	
	
Como resultado se obtendría el orden $2,1$ y $P:7$, el monto total perdido de la ganancia. 

\subsubsection{Hipótesis de resolución}
Esta en el jpg.
\subsubsection{Justificación formal de correctitud}
Nuestro algoritmo ordena de forma decreciente por el cociente $\pi$, luego devolvemos un vector v \'optimo. Iteramos sobre v con el indice i, acumulando la p\'erdida del subvector v[1..i]. Al final de las iteraciones tenemos entonces la p\'erdida completa correspondiente al vector v.
\subsubsection{Cota de complejidad temporal}
sort(arreglo, $\pi$) //O(n log (n))

perdida = 0
dias = 0

for i=1 to n    //O(n)
    dias = dias + ti    //O(1)
    perdida = perdida + dias * di   //O(1)

La complejidad del algoritmo es O(n log (n)).


\subsubsection{Verificación mediante casos de prueba}

A continuación presentamos distintas instancias que sirven para verificar que el programa funciona correctamente: \\ 

\begin{minipage}{0.2\textwidth}
	\begin{tabular}{ll}
		Input  \\
		\hline
		n &  \\
		$d_1$ & $t_1$ \\
		\vdots & \vdots \\
		$d_n$ & $t_n$ 
	\end{tabular} \\ 
\end{minipage}
\begin{minipage}{0.2\textwidth}
	\begin{tabular}{ll}
		Output  \\
		\hline
		$i_1$ \dots $i_n$ & P \\
		 \\
		 \\
		 \\
	\end{tabular} \\ 
\end{minipage}  \\
\\
n: cant. total de piezas  \\
$d_i$: cant. de pérdida diaria de la pieza $i$ \\
$t_i$: cant. de días de fabricación ($1 \le i \le n$) \\
$i_1$ \dots $i_n$: n piezas ordenadas \\
P: monto total perdido de la ganancia \\

Podemos separar el conjunto de soluciones en los siguientes casos: 
\begin{enumerate}
    \item $\frac{d_i}{t_i}$ es igual  $\forall$ $i$. 
	
		\begin{minipage}{0.2\textwidth}
			\begin{tabular}{l}
				Input  \\
				\hline
				$3$   \\
				$2$ $1$ \\
				$4$ $2$ \\
				$8$ $4$ 
			\end{tabular} \\ 
		\end{minipage}
		\begin{minipage}{0.2\textwidth}
			\begin{tabular}{ll}
				Output  \\
				\hline
				$1$   $2$  $3$ & $70$ \\
				$1$   $3$  $2$ & $70$ \\
				\vdots         & \vdots\\
				$3$   $2$  $1$ & $70$ \\
			\end{tabular} \\ 
		\end{minipage}  \\
		\textnormal{Cualquier permutación de las piezas $1,2,3$ es una solución válida para este caso. $P=70$ en los $6$ posibles outputs. Esto es porque el cociente entre la pérdida diaria y el tiempo de fabricación es igual para las $3$ piezas, $\frac{d_i}{t_i}=2, \forall i = 1,2,3$.} \\

	\item $\exists$ $j$ tal que $\frac{d_i}{t_i}$ $\ne$ $\frac{d_j}{t_j}$, j $\ne$ i, con j, i $=$ $1,...,n$. 
	
	\begin{itemize}
		\item Fijo $t_i$ y varío $d_i$, $\forall$ $i$. Entonces el orden va a estar determinado por la pérdida diaria $d_i$.
		
		\begin{minipage}{0.2\textwidth}
			\begin{tabular}{l}
				Input  \\
				\hline
				$5$   \\
				$1$ $2$ \\
				$2$ $2$ \\
				$3$ $2$ \\
				$5$ $2$ \\
				$4$ $2$ \\
			\end{tabular} \\ 
		\end{minipage}
		\begin{minipage}{0.3\textwidth}
			\begin{tabular}{ll}
				Output  \\
				\hline
				$4$ $5$ $3$ $2$ $1$ & $70$ \\
				\\
				\\
				\\
			    \\
				\\
			\end{tabular} \\ 
		\end{minipage}  
		\begin{minipage}{0.2\textwidth}
			\begin{tabular}{l|l}
				Pieza i & $d_i$  \\
				\hline
				$1$ & $1$  \\
				$2$ & $2$ \\
				$3$ & $3$ \\
				$4$ & $5$ \\
				$5$ & $4$ \\
                        \\
			\end{tabular} \\ 
		\end{minipage}
		
		\item Fijo $d_i$ y varío $t_i$ $\forall$ $i$. El orden depende del cociente entre $d_i$ y $t_i$.
		
		\begin{minipage}{0.2\textwidth}
			\begin{tabular}{l}
				Input  \\
				\hline
				$3$   \\
				$2$ $1$ \\
				$2$ $2$ \\
				$2$ $3$ \\
			\end{tabular} \\ 
		\end{minipage}
		\begin{minipage}{0.3\textwidth}
			\begin{tabular}{ll}
				Output  \\
				\hline
				$1$ $2$ $3$ & $20$ \\
				\\
				\\
				\\
			\end{tabular} \\ 
		\end{minipage}  
		\begin{minipage}{0.2\textwidth}
			\begin{tabular}{l|l}
				Pieza i & $\frac{d_i}{t_i}$  \\
				\hline
				$1$     & $2$ \\
				$2$     & $1$ \\
				$3$     & $0,67$ \\
				\\
			\end{tabular} \\ 
		\end{minipage}  \\
		
		\item Varío $d_i$ y $t_i$ $\forall$ $i$ pero sin que todos los $\frac{d_i}{t_i}$ sean iguales. El orden depende del cociente entre $d_i$ y $t_i$. 
		
		\begin{minipage}{0.2\textwidth}
			\begin{tabular}{l}
				Input  \\
				\hline
				$3$   \\
				$3$ $1$ \\
				$1$ $2$ \\
				$2$ $3$ \\
			\end{tabular} \\ 
		\end{minipage}
		\begin{minipage}{0.3\textwidth}
			\begin{tabular}{ll}
				Output  \\
				\hline
				$1$ $3$ $2$ & $14$ \\
				\\
				\\
				\\
			\end{tabular} \\ 
		\end{minipage}  
        	\begin{minipage}{0.2\textwidth}
			\begin{tabular}{l|l}
				Pieza i & $\frac{d_i}{t_i}$  \\
				\hline
				$1$     & $3$ \\
				$2$     & $0,5$ \\
				$3$     & $0,67$ \\
				\\
			\end{tabular} \\ 
		\end{minipage}  \\		
		
	\end{itemize}
	
\end{enumerate}

Estos casos cubrirían todo el espacio de soluciones. Ejecutamos distintos ejemplos correspondientes a cada uno de ellos y obtuvimos la respuesta esperada. Por lo tanto, podemos concluir que el comportamiento del programa es correcto. 

\subsubsection{Medición empírica de la performance}

% -----------------------------------------------
\end{document}

	\newpage
	
	%-- Problema 3 --
	\subsection{Problema 3: Rompecolores}
	% ------ headers globales y begin ---------------
\documentclass[11pt, a4paper, twoside]{article}
\usepackage{header_tp1}
\begin{document}{}
% -----------------------------------------------

\subsubsection{Descripción}

Este problema consiste en ubicar en un tablero la mayor cantidad de piezas posibles siguiendo ciertas reglas.  

\begin{itemize}
\item El tablero contiene $n \times m$ casilleros cuadrados, $n$ filas y $m$ columnas. 
\item Piezas existentes: $1,...,n \times m$. (Cantidad total de piezas: $n \times m$).     
\item Una pieza es cuadrada y puede tener de $1$ a $4$ colores distintos. A cada lado $(sup, izq, der, inf)$ le corresponde un color. 
\item Las piezas no se pueden rotar.
\item Colores posibles: $1,...,c$ ($c$ entero positivo). 
\item 2 piezas pueden ubicarse en casilleros adyacentes sólo si sus lados adyacentes tienen el mismo color. Podría ocurrir que no sea posible llenar completamente el tablero con las piezas existentes. 
\item Cantidad mínima de piezas que se pueden colocar en el tablero: $(n \times m)\div 2$. (Se intercalan las piezas en el tablero).  
\item El contenido final de una casilla podría ser $1,...,n \times m$, si se pudo colocar alguna ficha, o $0$ si quedara vacía. 

\end{itemize} 

El problema se deberá resolver utilizando la técnica de $Backtracking$ eligiendo algunas podas para mejorar los tiempos de ejecución del programa.

% Me falta agregar algún ejemplo...  

\subsubsection{Hipótesis de resolución}

\subsubsection{Justificación formal de correctitud}

\subsubsection{Cota de complejidad temporal}

\subsubsection{Verificación mediante casos de prueba}

\subsubsection{Medición empírica de la performance}

% -----------------------------------------------
\end{document}
\label{subsec:problema3}
	\newpage

%-- Apéndices --
\section{Apéndices}
	
	\subsection{Código Fuente (resumen)}
	% ------ headers globales y begin ---------------
\documentclass[11pt, a4paper, twoside]{article}
\usepackage{header_tp2}
\begin{document}{}
% -----------------------------------------------
%á

% -----------------------------------------------
\end{document}
	\clearpage
	
  %
  % - Para la reentrega -
  %
  \subsection{Informe de Modificaciones}
	% ------ headers globales y begin ---------------
\documentclass[11pt, a4paper, twoside]{article}
\usepackage{header_tp2}
\begin{document}{}
% -----------------------------------------------
%á


% -----------------------------------------------
\end{document}
	\clearpage
  
  %
  % - Reentrega
  %
  \subsection{Ejercicios Adicionales (reentrega)}
	% ------ headers globales y begin ---------------
\documentclass[11pt, a4paper, twoside]{article}
\usepackage{header_tp1}
\begin{document}{}
% -----------------------------------------------

\nota{GONZA}

% -----------------------------------------------
\end{document}

	\clearpage
	
	%-- Bibliografia --
	%\subsection{Bibliografía}
	%\begin{thebibliography}{99}
%	
%		\bibitem{lib:brassard} G. Brassard, P. Bratley, \textit{Fundamental of Algorithmics}, Prentice Hall,  1996, Chapter 6.2, \enquote{General characteristics of greedy algoritmhs}, Chapter 9.6, \enquote{Backtracking}, 
%		\bibitem{lib:cormen} Cormen, Leiserson, Rivest \textit{Introduction to Algorithms}, 2001, Chapter 16 \enquote{Greedy Algorithms}.
%		\bibitem{stl:stl} \texttt{http://www.cplusplus.com/reference/stl/}
%		\bibitem{stl:set} \texttt{http://www.cplusplus.com/reference/set/set/}
%		\bibitem{stl:vector} \texttt{http://www.cplusplus.com/reference/vector/vector/}
%		\bibitem{stl:pqueue} \texttt{http://www.cplusplus.com/reference/queue/priority\_queue/}
%		\bibitem{wiki:greedy} \texttt{http://en.wikipedia.org/wiki/Greedy\_algorithm\#Specifics}
%		\bibitem{wiki:back} \texttt{http://en.wikipedia.org/wiki/Backtracking\#Usage\_considerations}

	%\end{thebibliography}

\end{TP1}
\end{document}
