% ------ headers globales y begin ---------------
\documentclass[11pt, a4paper, twoside]{article}
\usepackage{header_tp1}
\begin{document}{}
% -----------------------------------------------

Se realizaron las siguientes modificaciones:

\begin{itemize}
\item Se cambiaron los nombres de las secciones ``Hipótesis de Resolución''
por ``Planteamiento de Resolución''; el contenido sigue siendo el mismo.
\end{itemize}

\secref{introduccion}:
\begin{itemize}[leftmargin=+4em]
  \item Correcciones y aclaraciones mínimas en el texto.
    
    \item Se creó \secref{instrucciones}
    en donde se aclaran detalles relativos al código 
    y/o scripts provistos en el paquete de entrega, 
    y se movió el apartado de ``Herramientas Utilizadas'' 
    a \subsecref{instrucciones-herramientas}.
\end{itemize}

% \subsecref{problema1}:
% \begin{itemize}[leftmargin=+4em]

    % \item Se eliminó un error ocasionado por un ``\textit{mergeo incorrecto}'' 
    % en el primer párrafo de \subsubsecref{problema1}{descripcion}.
    % También se corrigieron detalles varios, sintácticos y de estilo.
    
    % \item Se agregaron subtítulos a los distintos párrrafos con el
    % objetivo de permitir al lector una rápida y mejor
    % categorización y comprensión del texto.
    
    % \item Se reformuló en gran parte el texto de \subsubsecref{problema1}{resolucion},
    % incluyendo las correcciones indicadas por el profesor.
    
% \end{itemize}

\subsecref{problema3}:

\begin{itemize}[leftmargin=+4em]

    \item Se realizaron las correcciones marcadas por el profesor en
\subsubsecref{problema1}{descripcion}.

    \item Para la primer parte\footnote{ Consta del análisis del tipo de
problema brindado, del universo de soluciones posible, y la formalización de la
función objetivo.} del contenido que figuraba en la sección ``Hipótesis de
Resolución'', se agregaron aclaraciones a algunas de las correciones hechas, se
eliminó/reformuló por completo la abstracción formal matemática\footnote{ A
causa de la notación utilizada, que era incorrecta, y de que llegamos a la
conclusión de que no era necesario ni deseable ese nivel de ``formalidad'' en la
notación, cuando las mismas ideas pueden bien ser explicadas, sin perder su
formalidad, en lenguaje coloquial.}. Además, se trasladó todo el contenido de
estos párrafos a \subsubsecref{problema1}{descripcion}, ya que consideramos que
el análisis realizado en esos párrafos es más propio del modelado/descripción
del problema; es decir que es independiente de la resolución que hayamos ideado.
    
    \item El texto que figuraba en la segunda parte del contenido de la sección
``Hipótesis de Resolución'' se eliminó/reformuló por completo, haciendo un
análisis y descripción más detallados de los objetivos del algoritmo y las podas
implementadas. No se hizo demasiado hincapié en los detalles estrictamente
implementativos, sino en las nociones o ideas generales que inspiraron el
código.

    \item El código del problema 3 fue reescrito por completo, utilizando una
versión con clases, permitiendo así realizar diversas abstracciones que
permitieron (a nuestro gusto) una mejor organización y comprensión del
algoritmo. Por ejemplo, se implementó la clase IndiceDePiezas, la cual contiene
todo el comportamiento relativo al índice con el que se eligen las piezas que
serán utilizadas, y la clase abstracta IteradorIndiceDePiezas, cuyas clases
heredadas (IteradorSecuencial, IteradorColores) son instanciadas por el Índice
de Piezas, y contienen un comportamiento común a ambas, y un comportamiento
propio que depende de la posición del tablero en que se encuentre el iterador.

\end{itemize}

% -----------------------------------------------
\end{document}
